%!TEX root = Ausarbeitung-Thoma.tex
Sowohl das Problem der Knotenklassifikation, als auch das der Textklassifikation,
wurde bereits in verschiedenen Kontexten. Jedoch scheien bisher entweder nur die Struktur des zugrundeliegenden Graphen oder nur Eigenschaften der Texte verwendet worden zu sein.

So werden in \cite{bhagat,szummer} unter anderem Verfahren zur Knotenklassifikation
beschrieben, die wie der in \cite{aggarwal2011} vorgestellte DYCOS-Algorithmus,
um den es in dieser Ausarbeitung geht, auch auf Random Walks basieren.

Obwohl es auch zur Textklassifikation einige Paper gibt \cite{Zhu02learningfrom,Jiang2010302}, geht doch keines davon auf den Spezialfall der Textklassifikation
mit einem zugrundeliegenden Graphen ein.

Die vorgestellten Methoden zur Textklassifikation variieren außerdem sehr stark.
Es gibt Verfahren, die auf dem bag-of-words-Modell basieren \cite{Ko:2012:STW:2348283.2348453}
wie es auch im DYCOS-Algorithmus verwendet wird. Aber es gibt auch Verfahren,
die auf dem Expectation-Maximization-Algorithmus basieren \cite{Nigam99textclassification}
oder Support Vector Machines nutzen \cite{Joachims98textcategorization}.

Es wäre also gut Vorstellbar, die Art und Weise wie die Texte in die Klassifikation
des DYCOS-Algorithmus einfließen zu variieren. Allerdings ist dabei darauf hinzuweisen,
dass die im Folgeden vorgestellte Verwendung der Texte sowohl einfach zu implementieren
ist und nur lineare Vorverarbeitungszeit in Anzahl der Wörter des Textes hat,
als auch es erlaubt einzelne
Knoten zu klassifizieren, wobei der Graph nur lokal um den zu klassifizerenden
Knoten betrachten werden muss.