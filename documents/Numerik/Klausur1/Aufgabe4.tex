\section*{Aufgabe 4}
\textbf{Aufgabe}:

\[I(f) = \int_a^b f(x) \mathrm{d}x \]

\begin{enumerate}
    \item Integrand am linken und am rechten Rand interpolieren
	\item Interpolationspolynom mit Quadraturformel integrieren
\end{enumerate}

\textbf{Lösung}:

Nutze Interpolationsformel von Lagrange:

\begin{align}
    L_i &= \frac{\prod_{j=1, j \neq i}^n (x-x_j)}{\prod_{j=1, j \neq i}^n (x_i - x_j)}\\
    p(x) &= \sum_{i=0}^{1} f_i \cdot L_i(x)
\end{align}

Berechne Lagrangepolynome:

\begin{align}
    L_0(x) = \frac{x-b}{a-b} \\
    L_1(x) = \frac{x-a}{b-a}
\end{align}

So erhalten wir:

\begin{align}
    p(x) &= f(a) \frac{x-b}{a-b} + f(b) \frac{x-a}{b-a}\\
    &= \frac{f(a) (b-x) + f(b) (x-a)}{b-a} \\
    &= \frac{f(a)b- f(a)x + f(b) x- f(b)a}{b-a}\\
    &=\frac{x \cdot \left (f(b)-f(a) \right  ) + f(a)b- f(b)a}{b-a}\\
    &= x \cdot \underbrace{\frac{f(b)-f(a)}{b-a}}_{=:r} + \underbrace{\frac{f(a)b - f(b)a}{b-a}}_{=: s}
\end{align}

Nun integrieren wir das Interpolationspolynom:

\begin{align}
    \int_a^b p(x) \mathrm{d} x &= \left [\frac{r}{2} x^2 +  sx \right ]_a^b\\
    &= \left (\frac{a^2 r}{2} + sa \right ) - \left (\frac{b^2 r}{2} + sb \right )\\
    &= a\left (\frac{a r}{2} + s \right ) - b \left (\frac{b r}{2} + s \right )\\
    &= a\left (\frac{a \frac{f(b)-f(a)}{b-a}}{2} + \frac{f(a)b - f(b)a}{b-a} \right ) - b \left (\frac{b \frac{f(b)-f(a)}{b-a}}{2} + \frac{f(a)b - f(b)a}{b-a} \right )\\
    &= a\left (\frac{-a f(a)+2b f(a)-a f(b)}{2 \cdot(b-a)}\right ) - b \left (\frac{bf(b) + b f(a) - 2 a f(b)}{2 \cdot (b-a)} \right )\\
    & \dots \text{theoretisch sollte das zu } (b-a)(\frac{f(a)}{2} + \frac{f(b)}{2}) \text{ zu vereinfachen sein}
\end{align}

Alternativer Rechenweg

\[ \int_a^b p(x)dx = \int_a^b f(a) \frac{x-b}{a-b}dx + \int_a^b f(b) \frac{x-a}{b-a}dx \]
\[ = \int_a^b \frac{f(a) \cdot x}{a-b}dx - \int_a^b \frac{f(a) \cdot b}{a-b}dx + \int_a^b \frac{f(b) \cdot x}{b-a}dx - \int_a^b \frac{f(b) \cdot a}{b-a}dx \]
\[ = \frac{1}{2} \cdot \frac{f(a) \cdot b^2}{a-b} - \frac{1}{2} \cdot \frac{f(a) \cdot a^2}{a-b} - \frac{f(a) \cdot b^2}{a-b} + \frac{f(a) \cdot b \cdot a}{a-b} + \frac{1}{2} \cdot \frac{f(b) \cdot b^2}{b-a} \]
\[ - \frac{1}{2} \cdot \frac{f(b) \cdot a^2}{b-a} - \frac{f(b) \cdot a \cdot b}{b-a} + \frac{f(b) \cdot a^2}{b-a}\]
\[=(b-a)\cdot(\frac{f(a)}{2} + \frac{f(b)}{2})\]

Betrachtet man nun die allgemeine Quadraturformel,
\[
\int_a^b f(x)dx \approx (b-a) \sum_{i=1}^s b_i f(a+c_i(b-a))
\]
so gilt für die hergeleitete Quadraturformel also $s=2$, $c_1=0, c_2=1$ und $b_1 = b_2 = \frac{1}{2}$. Sie entspricht damit der Trapezregel.

\subsection*{Teilaufgabe b)}
Sei nun $f(x) = x^2$ und $a = 0$ sowie $b = 4$. Man soll die ermittelte
Formel zwei mal auf äquidistanten Intervallen anwenden.

\textbf{Lösung:}

\begin{align}
	\int_0^4 p(x) \mathrm{d}x &= \int_0^2 p(x)\mathrm{d}x + \int_2^4 p(x)\mathrm{d}x \\
    &= (2-0)\cdot \left (\frac{0}{2} + \frac{4}{2} \right ) + (4-2) \cdot \left (\frac{4}{2} + \frac{16}{2} \right )\\
    &= 2 \cdot 2 + 2 \cdot (2+8)\\
    &= 24
\end{align}
