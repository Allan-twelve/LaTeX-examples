\section*{Aufgabe 5}
\subsection*{Teilaufgabe a}
Eine Quadraturformel $(b_i, c_i)_{i=1, \dots, s}$ hat die Ordnung
$p$, falls sie exakte Lösungen für alle Polynome vom Grad $\leq p -1$
liefert.

\subsection*{Teilaufgabe b}
Für die ersten 3. Ordnungsbedingungen gilt:

\begin{align*}
	1 = \sum_{i = 0}^{s} b_i \\
 	\frac{1}{2} = \sum_{i = 0}^{s} b_i \cdot c_i \\
 	\frac{1}{3} = \sum_{i = 0}^{s} b_i \cdot c_i^2
\end{align*}

\subsection*{Teilaufgabe c}
Wähle die Simpson-Regel, also $c_1=0, c_2 = \frac{1}{2}, c_3 = 1$ und
$b_1 = b_3 = \frac{1}{6}$ und $b_2 = \frac{4}{6}$.

Überprüfe nun Ordnungsbedingungen 1-4 $\Rightarrow$ Simpson-Regel hat Ordnung 4
Überprüfe Ordnungsbedingung 5 $\Rightarrow$ Simpson-Regel hat nicht Ordnung 5. %TODO ausführlicher
