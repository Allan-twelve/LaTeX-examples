\documentclass[a4paper]{scrartcl}
\usepackage{amssymb, amsmath} % needed for math
\usepackage[utf8]{inputenc} % this is needed for umlauts
\usepackage[ngerman]{babel} % this is needed for umlauts
\usepackage[T1]{fontenc}    % this is needed for correct output of umlauts in pdf
\usepackage{pdfpages}       % Signatureinbingung und includepdf
\usepackage{geometry}       % [margin=2.5cm]layout
\usepackage[pdftex]{hyperref}       % links im text
\usepackage{color}
\usepackage{framed}
\usepackage{enumerate}      % for advanced numbering of lists
\usepackage{marvosym}       % checkedbox
\usepackage{wasysym}
\usepackage{braket}         % for \Set{}
\usepackage{pifont}% http://ctan.org/pkg/pifont
\usepackage{gauss}
\usepackage{algorithm,algpseudocode}
\usepackage{parskip}
\usepackage{lastpage}
\allowdisplaybreaks

\newcommand{\cmark}{\ding{51}}%
\newcommand{\xmark}{\ding{55}}%

\title{Numerik Klausur 4 - Musterlösung}
\makeatletter
\AtBeginDocument{
	\hypersetup{
	  pdfauthor   = {Martin Thoma, Peter, Felix},
	  pdfkeywords = {Numerik, KIT, Klausur},
	  pdftitle    = {\@title}
  	}
	\pagestyle{fancy}
	\lhead{\@title}
	\rhead{Seite \thepage{} von \pageref{LastPage}}
}
\makeatother

\usepackage{fancyhdr}
\fancyfoot[C]{}

\begin{document}
	\section*{Aufgabe 1}
\subsection*{Teilaufgabe a}
\textbf{Gegeben:}

\[A =
\begin{pmatrix}
    3 & 15 & 13 \\
    6 & 6  & 6  \\
    2 & 8  & 19
\end{pmatrix}\]

\textbf{Aufgabe:} LR-Zerlegung von $A$ mit Spaltenpivotwahl

\textbf{Lösung:}

\begin{align*}
	&
	&
    A^{(0)} &= \begin{gmatrix}[p]
		3 & 15 & 13 \\
		6 & 6  & 6  \\
		2 & 8  & 19
	 \rowops
	 \swap{0}{1}
	\end{gmatrix}
	&\\
    P^{(1)} &= \begin{pmatrix}
		0 & 1 & 0\\
		1 & 0 & 0\\
     	0 & 0 & 1
	\end{pmatrix},
	&
    A^{(1)} &= \begin{gmatrix}[p]
		6 & 6  & 6  \\
		3 & 15 & 13 \\
		2 & 8  & 19
	 \rowops
	 \add[\cdot (-\frac{1}{2})]{0}{1}
	 \add[\cdot (-\frac{1}{3})]{0}{2}
	\end{gmatrix}
	&\\
	L^{(1)} &= \begin{pmatrix}
		1 & 0 & 0\\
		-\frac{1}{2} & 1 & 0\\
     	-\frac{1}{3} & 0 & 1
	\end{pmatrix},
	&
    A^{(2)} &= \begin{gmatrix}[p]
		6 & 6  & 6  \\
		0 & 12 & 10 \\
		0 & 6  & 17
	 \rowops
	 \add[\cdot (-\frac{1}{2})]{1}{2}
	\end{gmatrix}
	&\\
	L^{(2)} &= \begin{pmatrix}
		1 & 0 & 0\\
		0 & 1 & 0\\
     	0 & -\frac{1}{2} & 1
	\end{pmatrix},
	&
    A^{(3)} &= \begin{gmatrix}[p]
		6 & 6  & 6  \\
		0 & 12 & 10 \\
		0 & 0  & 12
	\end{gmatrix}
\end{align*}

Es gilt:

\begin{align}
	L^{(2)} \cdot L^{(1)} \cdot \underbrace{P^{(1)}}_{=: P} \cdot A^{0} &= \underbrace{A^{(3)}}_{=: R}\\
	\Leftrightarrow P A &= (L^{(2)} \cdot L^{(1)})^{-1} \cdot R \\
	\Rightarrow L &= (L^{(2)} \cdot L^{(1)})^{-1}\\
	&= \begin{pmatrix}
		1 & 0 & 0\\
		\frac{1}{2} & 1 & 0\\
		\frac{1}{3} & \frac{1}{2} & 1
	\end{pmatrix}
\end{align}

Nun gilt: $P A = L R = A^{(1)}$ (Kontrolle mit \href{http://www.wolframalpha.com/input/?i=%7B%7B1%2C0%2C0%7D%2C%7B0.5%2C1%2C0%7D%2C%7B1%2F3%2C0.5%2C1%7D%7D*%7B%7B6%2C6%2C6%7D%2C%7B0%2C12%2C10%7D%2C%7B0%2C0%2C12%7D%7D}{Wolfram|Alpha})

\subsection*{Teilaufgabe b}

\textbf{Gegeben:}

\[A =
\begin{pmatrix}
    9 & 4 & 12 \\
    4 & 1  & 4 \\
   12 & 4  & 17
\end{pmatrix}\]

\textbf{Aufgabe:} $A$ auf positive Definitheit untersuchen, ohne Eigenwerte zu berechnen.

\textbf{Vorüberlegung:}
Eine Matrix $A \in \mathbb{R}^{n \times n}$ heißt positiv definit $\dots$
\begin{align*}
  \dots & \Leftrightarrow \forall x \in \mathbb{R}^n \setminus \Set{0}: x^T A x > 0\\
	& \Leftrightarrow \text{Alle Eigenwerte sind größer als 0}
\end{align*}

Falls $A$ symmetrisch ist, gilt:
\begin{align*}
 \text{$A$ ist positiv definit} & \Leftrightarrow \text{alle führenden Hauptminore von $A$ sind positiv}\\
	& \Leftrightarrow \text{es gibt eine Cholesky-Zerlegung $A=GG^T$}\\
\end{align*}

\subsubsection*{Lösung 1: Hauptminor-Kriterium}

\begin{align}
	\det(A_1) &= 9 > 0\\
	\det(A_2) &=
		\begin{vmatrix}
			9 & 4 \\
			4 & 1 \\
		\end{vmatrix} = 9 - 16 < 0\\
	&\Rightarrow \text{$A$ ist nicht positiv definit}
\end{align}

\subsubsection*{Lösung 2: Cholesky-Zerlegung}
\begin{align}
	l_{11} &= \sqrt{a_{11}} = 3\\
	l_{21} &= \frac{a_{21}}{l_{11}} = \frac{4}{3}\\
	l_{31} &= \frac{a_{31}}{l_{11}} = \frac{12}{3} = 4\\
	l_{22} &= \sqrt{a_{22} - {l_{21}}^2} = \sqrt{1 - \frac{16}{9}}= \sqrt{-\frac{7}{9}} \notin \mathbb{R}\\
 & \Rightarrow \text{Es ex. keine Cholesky-Zerlegung, aber $A$ ist symmetrisch}\\
 & \Rightarrow \text{$A$ ist nicht positiv definit}
\end{align}

	\section*{Aufgabe 2}
Zeige die Aussage für $2\times2$ Matrizen durch Gauß-en mit
Spaltenpivotwahl.

\subsection*{Lösung}
\subsubsection*{Behauptung:}
Für alle tridiagonalen Matrizen gilt:
\begin{enumerate}
    \item[(i)] Die Gauß-Elimination erhält die tridiagonale Struktur
    \item[(ii)] $\rho_n(A) := \frac{\alpha_\text{max}}{\max_{i,j} |a_{ij}|} \leq 2$
\end{enumerate}

\subsubsection*{Beweis:}
\paragraph{Teil 1: (i)}
\begin{align}
    A &= \begin{gmatrix}[p]
        * & *       &        & \\
        * & \ddots  & \ddots & \\
          & \ddots  & \ddots &  * \\
          &         &   *    & *
        \rowops
            \add[\cdot \frac{-a_{21}}{a_{11}}]{0}{1}
        \end{gmatrix}
\end{align}

Offensichtlich ändert diese Operation nur Zeile 2. $a_{21}$ wird zu 0,
$a_{22}$ ändert sich irgendwie, alles andere bleibt unverändert.
Die gesammte Matrix ist keine tridiagonale Matrix mehr, aber die
um Submatrix  in $R^{(n-1) \times (n-1)}$ ist noch eine.

Muss man zuvor Zeile 1 und 2 tauschen (andere Zeilen kommen nicht in
Frage), so ist später die Stelle $a_{21} = 0$, $a_{22}$ ändert sich
wieder irgendwie und $a_{23}$ ändert sich auch. Dies ändert aber nichts
an der tridiagonalen Struktur der Submatrix.

\paragraph{Teil 2: (ii) für $A \in \mathbb{R}^{2 \times 2}$}
Sei $\begin{pmatrix}a_{11} & a_{12}\\a_{21} & a_{22} \end{pmatrix} \in \mathbb{R}^{2 \times 2}$
beliebig.

O.B.d.A sei die Spaltenpivotwahl bereits durchgeführt, also $|a_{11}| \geq |a_{21}|$.

Nun folgt:

\begin{align}
    \begin{gmatrix}[p]
        a_{11} & a_{12}\\
        a_{21} & a_{22}
        \rowops
        \add[\cdot \frac{-a_{21}}{a_{11}}]{0}{1}
    \end{gmatrix}\\
    \leadsto
    \begin{gmatrix}[p]
        a_{11} & a_{12}\\
        0      & a_{22} - \frac{a_{12} \cdot a_{21}}{a_{11}}
    \end{gmatrix}
\end{align}

Wegen $|a_{11}| \geq |a_{21}|$ gilt:
\begin{align}
    \|\frac{a_{21}}{a_{11}}\| \leq 1
\end{align}

Also insbesondere

\begin{align}
    \underbrace{a_{22} - a_{12} \cdot \frac{a_{21}}{a_{11}}}_{\leq \alpha_\text{max}} \leq 2 \cdot \max_{i,j}|a_{ij}|
\end{align}

Damit ist Aussage (ii) für $A \in \mathbb{R}^{2 \times 2}$ gezeigt.

\paragraph{Teil 3: (ii) für allgemeinen Fall}

Aus Teil 2 folgt die Aussage auch direkt für größere Matrizen.
Der worst case ist, wenn man beim Addieren einer Zeile auf eine
andere mit $\max_{i,j}|a_{ij}|$ multiplizieren muss um das erste nicht-0-Element
der Zeile zu entfernen und das zweite auch $\max_{i,j}|a_{ij}|$ ist.
Dann muss man aber im nächsten schritt mit einem Faktor $\leq \frac{1}{2}$
multiplizieren, erhält also nicht einmal mehr $2 \cdot \max_{i,j}|a_{ij}|$.

	\section*{Aufgabe 3}
Die Jacobi-Matrix von $f$ lautet:
\[f' (x,y) = \begin{pmatrix}
	3     & \cos y\\
	3 x^2 & e^y
\end{pmatrix}\]
Hierfür wurde in in der ersten Spalte nach $x$ abgeleitet und in der
zweiten Spalte nach $y$.

Eine Iteration des Newton-Verfahren ist durch
\begin{align}
x_{k+1}&=x_{k}\underbrace{-f'(x_k)^{-1}\cdot f(x_k)}_{\Delta x}
\end{align}
gegeben (vgl. Skript, S. 35).

Zur praktischen Durchführung lösen wir
\begin{align}
    f'(x_0, y_0)\Delta x &= -f(x_0,y_0)\\
    L \cdot \underbrace{R \cdot \Delta x}_{=: c} &= -f(x_0, y_0)
\end{align}
mit Hilfe der LR Zerlegung nach $\Delta x$ auf.

\subsection*{Lösungsvorschlag 1 (Numerische Lösung)}
\begin{align}
%
	f'(x_0,y_0)	&= L \cdot R \\
	\Leftrightarrow f'(-\nicefrac{1}{3}, 0)	&= L \cdot R \\
	\Leftrightarrow \begin{pmatrix}
		3     & 1\\
		\frac{1}{3} & 1
	\end{pmatrix}
	&=
	\overbrace{\begin{pmatrix}
		1      & 0\\
		\frac{1}{9} & 1
	\end{pmatrix}}^{=: L} \cdot
	\overbrace{\begin{pmatrix}
		3 & 1\\
		0      & \frac{8}{9}
	\end{pmatrix}}^{=: R}\\
%
	L \cdot c	&= -f(x_0,y_0) \\
	\Leftrightarrow
	\begin{pmatrix}
		1      & 0\\
		\frac{1}{9} & 1
	\end{pmatrix}
	\cdot c
	&= -
		\begin{pmatrix}
		2\\
		\frac{26}{27}
	\end{pmatrix}\\
	\Rightarrow
	c &=		\begin{pmatrix}
		-2\\
		-\frac{20}{27}
	\end{pmatrix}\footnotemark\\
%
	R\cdot \Delta x &= c\\
	\Leftrightarrow
	\begin{pmatrix}
		3 & 1\\
		0      & \frac{8}{9}
	\end{pmatrix}
	\cdot \Delta x &=
	\begin{pmatrix}
		-2\\
		-\frac{20}{27}
	\end{pmatrix}\\
	\Rightarrow \Delta x &= \frac{1}{18}
	\begin{pmatrix}
		-7\\
		-15
	\end{pmatrix}
\end{align}
\footnotetext{Dieser Schritt wird durch Vorwärtssubsitution berechnet.}

Anschließend berechnen wir
\begin{align}
	\begin{pmatrix}
		x_1\\
		y_1
	\end{pmatrix} &=
	\begin{pmatrix}
		x_0\\
		y_0
	\end{pmatrix}+\Delta x \\
	\Leftrightarrow\begin{pmatrix}
		x_1\\
		y_1
	\end{pmatrix} &=
	\begin{pmatrix}
		-\frac{1}{3}\\
		0
	\end{pmatrix} +
    \frac{1}{18}
	\begin{pmatrix}
		-7\\
		-15
	\end{pmatrix} \\
	\Leftrightarrow\begin{pmatrix}
		x_1\\
		y_1
	\end{pmatrix} &=
	\begin{pmatrix}
		-\nicefrac{13}{18}\\
		-\nicefrac{15}{18}
	\end{pmatrix}
\end{align}


\subsection*{Lösungsvorschlag 2 (Analytische Lösung)}
LR-Zerlegung für $f'(x, y)$ kann durch scharfes hinsehen durchgeführt
werden, da es in $L$ nur eine Unbekannte links unten gibt. Es gilt
also ausführlich:

\begin{align}
	\begin{pmatrix}
		3     & \cos y\\
		3 x^2 & e^y
	\end{pmatrix}
	&=
	\overbrace{\begin{pmatrix}
		1      & 0\\
		l_{12} & 1
	\end{pmatrix}}^L \cdot
	\overbrace{\begin{pmatrix}
		r_{11} & r_{12}\\
		0      & r_{22}
	\end{pmatrix}}^R\\
	\Rightarrow r_{11} &= 3\\
	\Rightarrow r_{12} &= \cos y\\
	\Rightarrow \begin{pmatrix}
		3     & \cos y\\
		3 x^2 & e^y
	\end{pmatrix}
	&=
	\begin{pmatrix}
		1      & 0\\
		l_{12} & 1
	\end{pmatrix} \cdot
	\begin{pmatrix}
		3 & \cos y\\
		0 & r_{22}
	\end{pmatrix}\\
	\Rightarrow 3x^2 &\stackrel{!}{=} l_{12} \cdot 3 + 1 \cdot 0\\
	\Leftrightarrow l_{12} &= x^2\\
	\Rightarrow e^y &\stackrel{!}{=} x^2 \cdot \cos y + 1 \cdot r_{22}\\
	\Leftrightarrow r_{22} &= -x^2 \cdot \cos y + e^y\\
	\Rightarrow \begin{pmatrix}
		3     & \cos y\\
		3 x^2 & e^y
	\end{pmatrix}
	&=
	\begin{pmatrix}
		1   & 0\\
		x^2 & 1
	\end{pmatrix} \cdot
	\begin{pmatrix}
		3 & \cos y\\
		0 & -x^2 \cdot \cos y + e^y
	\end{pmatrix}\\
	P &= I_2
\end{align}

	\section*{Aufgabe 4}
\subsection*{Teilaufgabe a)}
\begin{enumerate}
    \item Ordnung 3 kann durch geschickte Gewichtswahl erzwungen werden.
    \item Ordnung 4 ist automatisch gegeben, da die QF symmetrisch sein soll.
    \item Aufgrund der Symmetrie gilt Äquivalenz zwischen Ordnung 5 und 6.
          Denn eine hätte die QF Ordnung 5, so wäre wegen der
          Symmetrie Ordnung 6 direkt gegeben. Ordnung 6 wäre aber
          bei der Quadraturformel mit 3 Knoten das Maximum, was nur
          mit der Gauß-QF erreicht werden kann. Da aber $c_1 = 0$ gilt,
          kann es sich hier nicht um die Gauß-QF handeln. Wegen
          erwähnter Äquivalenz kann die QF auch nicht Ordnung 5 haben.
\end{enumerate}

Da $c_1 = 0$ gilt, muss $c_3 = 1$ sein (Symmetrie). Und dann muss $c_2 = \frac{1}{2}$
sein. Es müssen nun die Gewichte bestimmt werden um Ordnung 3 zu
garantieren mit:

\begin{align}
    b_i &= \int_0^1 L_i(x) \mathrm{d}x\\
    b_1 &= \frac{1}{6},\\
    b_2 &= \frac{4}{6},\\
    b_3 &= \frac{1}{6}
\end{align}

\subsection*{Teilaufgabe b)}
Als erstes ist festzustellen, dass es sich hier um die Simpsonregel handelt und die QF
\begin{align}
    \int_a^b f(x) \mathrm{d}x &= (b-a) \cdot \frac{1}{6} \cdot \left ( f(a) + 4 \cdot f(\frac{a+b}{2}) + f(b) \right )
\end{align}

ist. Wenn diese nun auf $N$ Intervalle aufgepflittet wird gilt folgendes:
\begin{align}
	h &= \frac{(b-a)}{N} \\
	\int_a^b f(x) \mathrm{d}x &= h \cdot \frac{1}{6} \cdot \left [ f(a) + f(b) + 2 \cdot \sum_{i=1}^{N-1} f(a + i \cdot h) + 4 \cdot \sum_{l=0}^{N-1} f(a + \frac{1}{2} \cdot h + l \cdot h)\right ]
\end{align}

$\sum_{i=1}^{N-1} f(a + i \cdot h)$ steht für die Grenzknoten
 (deshalb werden sie doppelt gezählt). Von den Grenzknoten gibt es
insgesamt $N-2$ Stück, da die tatsächlichen Integralgrenzen $a$ und $b$
nur einmal in die Berechnung mit einfließen.

$\sum_{l=0}^{N-1} f(a + \frac{1}{2} \cdot h + l \cdot h)$ sind die jeweiligen
mittleren Knoten der Intervalle. Davon gibt es $N$ Stück.

\subsection*{Teilaufgabe c)}
Diese Aufgabe ist nicht relevant, da Matlab nicht Klausurrelevant ist.
\clearpage
	\section*{Aufgabe 5}
\subsection*{Teilaufgabe a}
Eine Quadraturformel $(b_i, c_i)_{i=1, \dots, s}$ hat die Ordnung
$p$, falls sie exakte Lösungen für alle Polynome vom Grad $\leq p -1$
liefert.

\subsection*{Teilaufgabe b}
Die Ordnungsbedingungen, mit denen man zeigen kann, dass eine Quadraturformel
mindestens Ordnung $p$ hat, lautet:
\[\forall p \in \Set{1, \dots, p}: \sum_{i=1}^s b_i c_i^{q-1} = \frac{1}{q}\]

\subsection*{Teilaufgabe c}
\paragraph{Aufgabe} Bestimmen Sie zu den Knoten $c_1 = 0$ und $c_2 = \frac{2}{3}$ Gewichte, um eine Quadraturformel
maximaler Ordnung zu erhalten. Wie hoch ist die Ordnung?

\paragraph{Lösung}

Nach VL kann bei Vorgabe von $s$ Knoten auch die Ordnung $s$ durch
geschickte Wahl der Gewichte erreicht werden. Nach Satz 27 ist diese
Wahl eindeutig.
Also berechnen wir die Gewichte, um die Ordnung $p=2$ zu sichern.

Dazu stellen wir zuerst die Lagrange-Polynome auf:

\begin{align}
	L_1(x) &= \frac{x-x_2}{x_1 - x_2} = \frac{x-c_2}{c_1-c_2} = \frac{x-\nicefrac{2}{3}}{-\nicefrac{2}{3}} = -\frac{3}{2} x + 1\\
    L_2(x) &= \frac{x-x_1}{x_2 - x_1} = \frac{x-c_1}{c_2-c_1} = \frac{x}{\nicefrac{2}{3}} = \frac{3}{2} x
\end{align}

Nun gilt für die Gewichte:
\begin{align}
	b_i &= \int_0^1 L_i(x) \mathrm{d}x\\
	b_1 &= \int_0^1 -\frac{3}{2} x + 1 \mathrm{d}x = \left [ -\frac{3}{4}x^2 + x \right ]_0^1 = \frac{1}{4}\\
	b_2 &= \frac{3}{4}
\end{align}

Nun sind die Ordnungsbedingungen zu überprüfen:
\begin{align}
    \nicefrac{1}{1} &\stackrel{?}{=} b_1 c_1^0 + b_2 c_2^0 = \nicefrac{1}{4} + \nicefrac{3}{4} \text{\;\;\cmark}\\
    \nicefrac{1}{2} &\stackrel{?}{=} b_1 c_1^1 + b_2 c_2^1 = \frac{1}{4} \cdot 0 + \frac{3}{4} \cdot \frac{2}{3} \text{\;\;\cmark}\\
    \nicefrac{1}{3} &\stackrel{?}{=} b_1 c_1^2 + b_2 c_2^2 = \frac{1}{4} \cdot 0 + \frac{3}{4} \cdot \frac{4}{9} \text{\;\;\cmark}\\
    \nicefrac{1}{4} &\stackrel{?}{=} b_1 c_1^3 + b_2 c_2^3 = \frac{1}{4} \cdot 0 + \frac{3}{4} \cdot \frac{8}{27} \text{\;\;\xmark}\\
\end{align}

Die Quadraturformel mit den Knoten $c_1 = 0$, $c_2 = \nicefrac{2}{3}$ sowie
den Gewichten $b_1 = \nicefrac{1}{4}$, $b_2 = \nicefrac{3}{4}$ erfüllt
also die 1., 2. und 3. Ordnungsbedingung, nicht jedoch die 4.
Ordnungsbedingung. Ihre maximale Ordnung ist also $p=3$.

\textbf{Anmerkungen:} Da $c_1 = 0$ kann es sich nicht um die Gauß-QF handeln.
Somit können wir nicht Ordnung $p=4$ erreichen.

Bei der Suche nach den Gewichten hätte man alternativ auch das folgende
LGS lösen können:

\begin{align}
    \begin{pmatrix}
        c_1^0 & c_2^0\\
        c_1^1 & c_2^1
    \end{pmatrix}
    \cdot x
    =
    \begin{pmatrix}
        1\\
        \nicefrac{1}{2}
    \end{pmatrix}
\end{align}

\end{document}
