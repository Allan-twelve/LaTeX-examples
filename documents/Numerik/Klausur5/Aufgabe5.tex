\section*{Aufgabe 5}
\subsection*{Aufgabe}
Bestimme alle Quadraturformeln mit $s=3$ und Knoten
$0 = c_1 < c_2, c_3$ und Ordnung $p \geq 4$.

Schreiben Sie ein Programm in Pseudocode, welches zu vorgegebenem
$c_2$ den Knoten $c_3$ und die Gewichte $b_i$ möglichst effizient
berechnet.

Wie viele symmetrische Quadraturformeln gibt es mit diesen Eigneschaften?

\subsection*{Lösung}
Da $c_1 = 0$ kann es keine Gauß-Quadraturformel sein. Daher kann
die Ordnung nicht $2 \cdot s = 6$ sein. Interessant sind also
\begin{itemize}
    \item[(A)] Symmetrische Quadraturformeln der Ordnung 4
    \item[(B)] Unsymmetrische Quadraturformeln der Ordnung 4
    \item[(C)] Unsymmetrische Quadraturformeln der Ordnung 5
\end{itemize}

Die Simpson-Regel mit $c_1 = 0, c_2 = \frac{1}{2}$ und $c_3 = 1$
mit $b_1 = b_3 = \frac{1}{6}$ und $b_2 = \frac{4}{6}$ ist die einzige
symmetrische Quadraturformel in (A).

Für (B) müssen die Ordnungsbedingungen gelten:
\begin{align}
    \nicefrac{1}{1} &\stackrel{!}{=} b_1 + b_2 + b_3\label{eq:i}\\
    \nicefrac{1}{2} &\stackrel{!}{=} b_2 \cdot c_2 + b_3 c_3\label{eq:ii}\\
    \nicefrac{1}{3} &\stackrel{!}{=} b_2 \cdot c_2^2 + b_3 c_3^2\label{eq:iii}\\
    \nicefrac{1}{4} &\stackrel{!}{=} b_2 \cdot c_2^3 + b_3 c_3^3\label{eq:iv}\\
    \stackrel{\ref{eq:ii}}{\Rightarrow} \frac{\nicefrac{1}{2} - b_3 c_3}{c_2}&\stackrel{!}{=} b_2 \label{eq:ii2}\\
    \stackrel{\ref{eq:ii2} \text{ in } \ref{eq:iii}}{\Rightarrow} \nicefrac{1}{3} &\stackrel{!}{=} (\nicefrac{1}{2} - b_3 c_3) \cdot c_2 + b_3 c_3^2\label{eq:iii2}\\
    \Leftrightarrow \nicefrac{1}{3} - \nicefrac{1}{2} \cdot c_2 &\stackrel{!}{=} b_3 (- c_3 c_2 + c_3^2)\\
    \Leftrightarrow \frac{\nicefrac{1}{3} - \nicefrac{1}{2} \cdot c_2}{c_3^2 - c_3 c_2} &\stackrel{!}{=} b_3 \label{eq:iii3}\\
    \stackrel{\ref{eq:ii2} \text{ in } \ref{eq:iv}}{\Rightarrow} \nicefrac{1}{4} &\stackrel{!}{=} (\nicefrac{1}{2} - b_3 c_3) \cdot c_2^2 + b_3 c_3^3\label{eq:iv2}\\
    \Leftrightarrow \nicefrac{1}{4} - \nicefrac{1}{2} c_2^2&\stackrel{!}{=} b_3 (c_3^3 - c_3 \cdot c_2^2) \label{eq:iv2}\\
    \stackrel{\ref{eq:iii3} \text{ in } \ref{eq:iv2}}{\Rightarrow} \nicefrac{1}{4} - \nicefrac{1}{2} c_2^2&\stackrel{!}{=} \frac{\nicefrac{1}{3} - \nicefrac{1}{2} \cdot c_2}{c_3^2 - c_3 c_2} (c_3^3 - c_3 \cdot c_2^2)\\
    \Leftrightarrow \frac{1}{2} \left (\frac{1}{2} - c_2^2 \right ) &= \frac{(\nicefrac{1}{3} - \nicefrac{1}{2} \cdot c_2)(c_3^2 - c_2^2)}{c_3 - c_2}\\
    \Leftrightarrow \frac{1}{2} - c_2^2 &= (\nicefrac{2}{3} - c_2)(c_3 + c_2)\\
    \Leftrightarrow \frac{1}{2} - c_2^2 &= \nicefrac{2}{3} c_3 + \nicefrac{2}{3} c_2 - c_2 c_3 - c_2^2\\
    \Leftrightarrow \frac{1}{2} - \frac{2}{3} c_2 &= \nicefrac{2}{3} c_3- c_2 c_3\\
    \Leftrightarrow \frac{1}{2} - \frac{2}{3} c_2 &= c_3(\nicefrac{2}{3}- c_2)\\
    \Leftrightarrow c_3 &= \frac{\frac{1}{2} - \frac{2}{3} c_2}{\nicefrac{2}{3}- c_2} = \frac{2c_2-\nicefrac{3}{2}}{3c_2 - 2}\\
\end{align}

Für (C) muss zusätzlich gelten:
\begin{align}
    \nicefrac{1}{5} &\stackrel{!}{=} b_2 \cdot c_2^4 + b_3 c_3^4
\end{align}

TODO
