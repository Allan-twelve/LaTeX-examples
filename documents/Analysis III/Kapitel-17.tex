In diesem Kapitel sei $\emptyset \ne X \in \fb_d, f: X \to \MdC$ eine Funktion, $ u:= \Re(f), v:= \Im(f)$, also: $u,v: X \to \MdR, f= u+iv$.

Wir versehen $\MdC$ mit der $\sigma$-Algebra $\fb_2$ (wir identifizieren $\MdC$ mit $\mdr^2$).

\begin{definition}
\index{messbar}
$f$ heißt (Borel-)\textbf{messbar}, genau dann wenn gilt: $f$ ist $\fb_d$-$\fb_2$-messbar.
\end{definition}

Aus 3.2 folgt: $f$ ist messbar genau dann, wenn $u$ und $v$ messbar sind.

\begin{definition}
\index{integrierbar}\index{Integral}
Sei $f$ messbar. $f$ heißt \textbf{integrierbar} (ib.) genau dann, wenn $u$ und $v$ integrierbar sind.
In diesem Fall setze
\[ \int_X f \text{ d}x := \int_X u \text{ d}x + i\int_X v \text{ d}x \quad ( \in \MdC) \]
\end{definition}

Es gilt: $|u|, |v| \leq |f| \leq |u| + |v|$ auf $X$.
Hieraus und aus 4.9 folgt: $f$ ist integrierbar genau dann, wenn $|f|$ integrierbar ist.

\begin{definition}
\[ \fl^p(X, \MdC) := \{ f : X \to \MdC | f  \text{ ist messbar und } \int_X |f|^p \text{ d}x < \infty  \} \]
(Achtung: mit den Betragsstrichen in ob. Integral ist der komplexe Betrag gemeint!)
\[ \cn := \{ f: X \to \MdC | f \text{ ist messbar und } f = 0 \text{ f.ü.} \} \]
$\fl^p(X,\MdC )$ ist ein komplexer Vektorraum (siehe 17.1) und $\cn$ ist ein Untervektorraum von $\fl^p(X,\MdC )$.
\[ L^p(X,\MdC ) := \fl^p(X,\MdC)\diagup\cn \]
\end{definition}

\begin{definition}
\index{orthogonal}
Für $f,g \in L^2(X,\MdC )$ setze
\[(f | g) := \int_X f(x) \overline{g(x)} \text{ d}x\]
sowie
\[f \bot g :\Longleftrightarrow (f | g) = 0 \quad \text{ ($f$ und $g$ sind \textbf{orthogonal}).} \]
( $\overline{z}$ bezeichne hierbei die komplex Konjugierte von $z$, vgl. Lineare Algebra).
\end{definition}

\textbf{Klar:} \begin{enumerate}
\item $L^p(X,\MdC )$ ist mit $\| f \|_p := (\int_X |f|^p \text{ d}x )^{\frac{1}{p}}$ ein komplexer normierter Raum (NR).
\item $(f | g)$ definiert ein Skalarprodukt auf $L^2(X,\MdC)$. Es ist
\[(f | g) = \overline{(g | f)}, \]
\[ (f | f) = \int_X f(x) \overline{f(x)} \text{ d}x = \int_X |f(x)|^2 \text{ d}x = \| f \|_2^2 \text{, also:} \]
\[ \| f\|_2 = \sqrt{(f|f)} \quad (f,g \in L^2(X,\MdC )) \]
(Beachte: es ist $z \cdot \overline{z} = |z|^2$ für $z \in \MdC$).
\end{enumerate}

\textbf{Inoffizielle Anmerkung:} Dieses Skalarprodukt ist auf $\MdC$ nur linear in der ersten Komponente! Wenn man einen $\MdC$-Skalar aus der zweiten Komponente rausziehen möchte, muss man diesen komplex konjugieren:
\begin{align*}
\alpha \in \MdC:\quad &(f|\alpha g) = \overline{\alpha} (f|g)\\
&(\alpha f|g) = \alpha (f | g)
\end{align*}

\begin{satz}
\label{Satz 17.1}
\begin{enumerate}
	\item 	Seien \(f,g\colon X\to\mdc\) integrierbar und \(\alpha,\beta\in\mdc\). Dann gelten:
	\begin{enumerate}
		\item[(i)] 	\(\alpha f+\beta g\) ist integrierbar und
				\[\int_X(\alpha f+\beta g)\,dx = \alpha\int_Xf\,dx+\beta\int_Xg\,dx\]
		\item[(ii)]	\(\text{Re}\left(\int_Xf\,dx\right) = \int_X\text{Re}(f)\,dx\ \) und
				\(\ \text{Im}\left(\int_Xf\,dx\right) = \int_X\text{Im}(f)\,dx\)
		\item[(iii)]	\(\overline f\) ist integrierbar und
				\[\int_X\overline f\,dx=\overline{\int_Xf\,dx}\]
	\end{enumerate}
	\item 	Die Sätze \ref{Satz 16.1} bis \ref{Satz 16.3} und das Beispiel \ref{Beispiel 16.6} gelten in
			\(L^p(X,\mdc)\).
	\item 	\(L^p(X,\mdc)\) ist ein komplexer Banachraum, \(L^2(X,\mdc)\) ist ein komplexer
			Hilbertraum.
\end{enumerate}
\end{satz}

\begin{wichtigesbeispiel}
\label{Beispiel 17.2}
Sei \(X=[0,2\pi]\). Für \(k\in\MdZ\) und \(t\in\mdr\) setzen wir
\begin{align*}
	e_k(t):=e^{ikt}=\cos(kt)+i\sin(kt) && \text{ und } && b_k:=\frac1{\sqrt{2\pi}}e_k
\end{align*}
Dann gilt: \(b_k,e_k\in L^2([0,2\pi],\mdc)\) und \[\int_0^{2\pi}e_0(t)\,dt=2\pi\]
Für \(k\in\MdZ\) und \(k\neq0\) ist
\begin{align*}
	\int_0^{2\pi}e_k(t)\,dt=\left.\frac1{ik}e^{ikt}\right\rvert_0^{2\pi}
	= \frac1{ik}\left(e^{2\pi ki}-1\right)=0
\intertext{Damit ist}
	(b_k\mid b_l) = \int^{2\pi}_0 b_k\overline{b_l}\,dt = \frac1{2\pi}\int_0^{2\pi}e^{ikt}e^{-ilt}\,dt
	= \frac1{2\pi}\int_0^{2\pi}e^{i(k-l)t}\,dt =
	\begin{cases}
		1 ,\text{falls } k=l\\
		0 ,\text{falls }k\neq l
	\end{cases}
\end{align*}
Insbesondere ist \(\| b_k\|_2=1\). Das heißt \(\{b_k\mid k\in\MdZ\}\) ist ein
\textbf{Orthonormalsystem} in \(L^2([0,2\pi],\mdc)\).
Zur Übung: \(\{b_k\mid k\in\MdZ\}\) ist linear unabhängig in \(L^2([0,2\pi],\mdc)\).
\end{wichtigesbeispiel}

\begin{definition}
Sei \((\alpha_k)_{k\in\MdZ}\) eine Folge in \(\mdc\) und \((f_k)_{k\in\MdZ}\) eine Folge in
\(L^2(X,\mdc)\).
\begin{enumerate}
	\item 	Für \(n\in\mdn_0\) setze
			\[s_n:=\sum^n_{k=-n}\alpha_k = \sum_{\lvert k\rvert\leq n}\alpha_k
			=\alpha_{-n}+\alpha_{-(n-1)}+\dots+\alpha_0+\alpha_1+\dots+\alpha_n\]
			Existiert \(\lim_{n\to\infty}s_n\) in \(\mdc\), so schreiben wir
			\(\sum_{k\in\MdZ}\alpha_k:=\lim_{n\to\infty}s_n\)
	\item 	Für \(n\in\mdn_0\) setze
			\[\sigma_n:=\sum^n_{k=-n}f_k=\sum_{\lvert k\rvert\leq n}f_k\]
			Gilt für ein \(f\in L^2(X,\mdc)\):
			\(\| f-\sigma_n\|_2\overset{n\to\infty}\longrightarrow 0\), so schreiben
			wir \[f\overset{\|\cdot\|_2}=\sum_{k\in\MdZ}f_k \ \ \
			\left(=\lim_{n\to\infty}\sigma_n \text{ im Sinne der } L^2\text{-Norm}\right)\]
\end{enumerate}
\end{definition}

\begin{definition}
\index{Orthonormalbasis}
Sei \(\{b_k\mid k\in\MdZ\}\) wie in \ref{Beispiel 17.2}. \(\{b_k\mid k\in\MdZ\}\) heißt eine
\textbf{Orthonormalbasis (ONB)} von \(L^2([0,2\pi],\mdc)\) genau dann, wenn es zu jedem
\(f\in L^2([0,2\pi],\mdc)\) eine Folge \[(c_k)_{k\in\MdZ}=(c_k(f))_{k\in\MdZ}\] gibt, mit
\[(\ast)\ \ \ \ \ \ \ \ \ f\overset{\|\cdot\|_2}=\sum_{k\in\MdZ}c_kb_k \]
\textbf{Frage:} Ist \(\{b_k\mid k\in\MdZ\}\) eine ONB von \(L^2([0,2\pi],\mdc)\)?\\
\textbf{Antwort:} Ja! In \ref{Satz 18.5} werden wir sehen, dass \((\ast)\) gilt mit
\(c_k=(f\mid b_k)\).
\end{definition}

\chapter{Fourierreihen}
\label{Kapitel 18}

In diesem Kapitel sei stets \(X=[0,2\pi]\), \(L^2:=L^2([0,2\pi],\mdc)\) und
\(L^2_\mdr:=L^2([0,2\pi],\mdr)\).  Weiter sei \(\{b_k\mid k\in\MdZ\}\) wie in \ref{Beispiel 17.2}.

\begin{satz}
\label{Satz 18.1}
Ist \(f\in L^2\) und gilt mit einer Folge \((c_k)_{k\in\MdZ}\) in \(\mdc\):
\(f\overset{\|\cdot\|_2}=\sum_{k\in\MdZ}c_kb_k \), so gilt:
\[c_k=(f\mid b_k) \text{ für alle } k\in\MdZ\]
\end{satz}

\begin{beweis}
Für \(n\in\mdn_0\) setze \[\sigma_n:=\sum_{\lvert k\rvert\leq n}c_kb_k\] Aus der Voraussetzung folgt
\(\| \sigma_n-f\|_2\to 0\) für \(n\to\infty\). Sei \(j\in\MdZ\) und \(n\in\mdn\) mit
\(n\geq \lvert j\rvert\). Es gilt einerseits
\[(\sigma_n\mid b_j) = \sum_{\lvert k\rvert\leq n}c_k(b_k\mid b_j)=c_j, \text{ da gilt: }
(b_k\mid b_j)=
\begin{cases}
0, \text{ falls } k\neq j\\
1, \text{ falls } k= j
\end{cases}\]
Andererseits: \((\sigma_n\mid b_j)\to(f\mid b_j)\) für \(n\to\infty\) wegen \ref{Beispiel 16.6}(3). Daraus
folgt \(c_j=(f\mid b_j)\)
\end{beweis}

\begin{definition}
\index{Fourier ! -sche Partialsumme}
\index{Fourier ! -koeffizient}
\index{Fourier ! -reihe}
Sei \(f\in L^2\), \(n\in\mdn_0\) und \(k\in\MdZ\).
\begin{enumerate}
\item	\(S_nf:=\sum_{\lvert k\rvert\leq n}(f\mid b_k)b_k\) heißt
	\textbf{n-te Fouriersche Partialsumme}. Also gilt:
	\[f\overset{\|\cdot\|_2}
	=\sum_{k\in\MdZ}(f\mid b_k)b_k\gdw\| f-S_nf\|_2
	\to0\]
\item	\((f\mid b_k)\) heißt \textbf{k-ter Fourierkoeffizient von f}.
\item	\(\sum_{k\in\MdZ}(f\mid b_k)b_k\) heißt \textbf{Fourierreihe von f}.
\item	Für \(n_0\in\mdn_0\) setze
	\(E_n:=[b_{-n},b_{-(n-1)},\dots,b_0,b_1,\dots,b_n]\)
	(lineare Hülle). Es ist dann \[\dim E_n=2n+1\]
	\textbf{Beachte: } Für \(v\in E_n\) gilt \(v(0)=v(2\pi)\).
\end{enumerate}
\end{definition}

\begin{satz}
\label{Satz 18.2}
\index{Besselsche Ungleichung}
\index{Ungleichung ! Besselsche}
Seien \(f_1,\dots,f_n,f\in L^2\).
\begin{enumerate}
\item	Gilt \(f_\mu\perp f_\nu\) für \(\mu\neq\nu\) (\(\mu,\nu=1,\dots,n\)),
	so gilt der Satz des Pythagoras
	\[\| f_1+\dots+f_n\|^2_2=
	\| f_1\|^2_2+\dots+
	\| f_n\|^2_2\]
\item	Die Abbildung \[S_n\colon
	\begin{cases}
		L^2\to E_n\\
		S_nf:=\sum_{\lvert k\rvert\leq n}(f\mid b_k)b_k
	\end{cases}\]
	ist linear und für jedes \(v\in E_n\) gilt \(S_nv=v\) und
	\((f-S_nf)\perp v\) mit \(f\in L^2\).
\item 	Die \textbf{Besselsche Ungleichung} lautet:
	\[\| S_nf\|^2_2
	=\sum_{\lvert k\rvert\leq n}\lvert(f\mid b_k)\rvert^2
	=\| f\|_2^2-\|(f-S_nf)\|^2_2
	\leq\| f\|^2_2\]
\item	Für alle \(v\in E_n\) gilt:
	\[\| f-S_nf\|_2\leq\| f-v\|_2
	\]
\end{enumerate}
\end{satz}

\begin{beweis}
\begin{enumerate}
\item	Es genügt den Fall \(n=2\) zu betrachten, der Rest folgt induktiv.
	\begin{align*}
	\| f_1+f_2\|_2^2
	&= (f_1+f_2\mid f_1+f_2)						\\
	&= (f_1\mid f_1)+(f_1\mid f_2)+(f_2\mid f_1)+(f_2\mid f_2)		\\
	&= (f_1\mid f_1)+(f_2\mid f_2)						\\
	&=\| f_1\|^2_2+\| f_2\|^2_2
	\end{align*}
\item	Übung!
\item	Es gilt
	\begin{align*}
	\| S_nf\|^2_2
	&= \left\lvert\left\lvert\sum_{\lvert k\rvert\leq n}(f\mid b_k)b_k\right\rvert
		\right\rvert^2_2
	\overset{(1)}=
		\sum_{\lvert k\rvert\leq n}\|(f\mid b_k)b_k\rvert
		\rvert^2_2
	= \sum_{\lvert k\rvert\leq n}\lvert(f\mid b_k)\rvert^2\| b_k\rvert
		\rvert^2_2
	= \sum_{\lvert k\rvert\leq n}\lvert(f\mid b_k)\rvert^2
	\end{align*}
	und
	\begin{align*}
	\| f\|^2_2
	= \|\underbrace{(f-S_nf)}_{\underset{(2)}\perp E_n}
		+\underbrace{S_nf}_{\in E_n}\|^2_2
	= \| f-S_nf\|^2_2 + \| S_nf\|^2_2
	\end{align*}
\item	Sei \(v\in E_n\). Dann gilt:
	\begin{align*}
	\| f-v\|^2_2
	&= \|\underbrace{(f-S_nf)}_{\perp E_n}
		+\underbrace{(S_nf-v)}_{\in E_n}\|^2_2		\\
	&\overset{(1)}=
		\| f-S_nf\|^2_2
		+\| S_nf-v\|^2_2				\\
	&\geq \| f-S_nf\|^2_2
	\end{align*}
\end{enumerate}
\end{beweis}

\begin{wichtigebemerkung}
\label{Bemerkung 18.3}
Es sei \(\mdk\in\{\mdr,\mdc\},\,a,b\in\mdr,\,I:=[a,b]\,(a<b)\) und \(f_{n},\,f,\,g\in C(I,\mdk)\); es war
\(\lVert f\rVert_{\infty}:=\max_{t\in I}\lvert f(t)\rvert\).
\begin{enumerate}
\item \((f_{n})\) konvergiert auf \(I\) gleichmäßig gegen \(f\) genau dann, wenn
    \(\lVert f_{n}-f\rVert_{\infty}\to 0\,(n\to\infty)\) (vgl. Analysis I/II).
\item \(f\in\mathrm{L}^{p}(I,\mdk)\) und \(\lVert f\rVert_{p}\leq(b-a)^{\frac{1}{p}}\lVert f\rVert_{\infty}\) (siehe \ref{Satz 16.2}).
\item Gilt \(f=g\) fast überall, so ist \(f=g\) auf \(I\).
\begin{beweis}
Es existiert eine Nullmenge \(N\subseteq I:\,f(x)=g(x)\,\forall x\in I\setminus N\).\\
Sei \(x_{0}\in\mdn\). Für \(\ep>0\) gilt: \(U_{\ep}(x_{0})\cap I\not\subseteq N\) (andernfalls:
    \(\lambda_{1}(N)\geq\lambda_{1}(U_{\ep}(x_{0})\cap I)>0\)). Das heißt, es existiert ein
    \(x_{\ep}\in U_{\ep}(x_{0})\cap I:\,x_{\ep}\not\in N\). Also:
    \(\forall n\in\mdn\,\exists x_{n}\in U_{\frac{1}{n}}(x_{0})\cap I:\, x_{n}\not\in N\). Also: \(x_{n}\to x_{0}\).\\
Dann: \(f(x_{0})=\lim_{n\to\infty}f(x_{n})=\lim_{n\to\infty}g(x_{n})=g(x_{0})\)
\end{beweis}
\end{enumerate}
\end{wichtigebemerkung}

\begin{satz}[Approximationssatz von Weierstraß]
\label{Satz 18.4}
Es sei \(I=[a,b]\) wie in \ref{Bemerkung 18.3} und \(\mdk\in\{\mdr, \mdc\}\).
\begin{enumerate}
\item Ist \(f\in C(I,\mdk)\) und \(\ep>0\), so existiert ein Polynom \(p\) mit Koeffizienten in \(\mdk\) mit:
\[
\lVert f-p\rVert_{\infty}<\ep
\]
\item Ist \(a=0,\,b=2\pi,\,f\in C(I,\mdk),\,f(0)=f(2\pi)\) und \(\ep>0\), so existiert ein \(n\in\mdn\) und ein
    \(v\in\mathrm{E}_{n}\) mit:
\[
\lVert f-v\rVert_{\infty}<\ep
\]
\end{enumerate}
\end{satz}

\begin{satz}
\label{Satz 18.5}
Sei \(f\in\mathrm{L}^{2}\). Dann gilt: \(f\overset{\lVert\cdot\rVert_{2}}{=}\sum_{k\in\mdz}{(f\mid b_{k})b_{k}}\) und
\[\lVert f\rVert_{2}^{2}=\sum_{k\in\mdz}{\lvert(f\mid b_{k})\rvert^{2}}\quad\text{(\textbf{Parsevalsche Gleichung})}\] Insbesondere gilt:
\((f\mid b_{k})\to 0\quad(\lvert k\rvert\to\infty)\).
\end{satz}

\begin{beweis}
Zu zeigen: \(\lVert f-S_{n}f\rVert_{2}\to0\,(n\to\infty)\). Die Parsevalsche Gleichung folgt dann aus \ref{Satz 18.2}.\\
Sei \(\ep>0\). Wende \ref{Satz 16.8}(2) auf \(\Re f\) und \(\Im f\) an. Dies liefert eine stetige Funktion
\(g:\,(0,2\pi)\to\mdc\) mit: \(K:=\supp(g)\subseteq(0,2\pi)\), \(K\) kompakt und \(\lVert f-g\rVert_{2}<\ep\).\\
Setze \(g(0):=g(2\pi):=0\). Dann ist \(g\) stetig auf \([0,2\pi]\). Satz \ref{Satz 18.4} liefert nun:
    \(\exists n\in\mdn\exists v\in\mathrm{E}_{n}:\,\lVert g-v\rVert_{\infty}<\ep\).\\
Damit: \(\lVert g-v\rVert_{2}\leq\sqrt{2\pi}\lVert g-v\rVert_{\infty}<\sqrt{2\pi}\ep\). Somit:
\begin{align*}
\lVert f-S_{n}f\rVert_{2}&=\lVert f-g+g-S_{n}g+S_{n}g-S_{n}f\rVert_{2}\\
    &\leq\underbrace{\lVert f-g\rVert_{2}}_{<\ep}
		+\underbrace{\lVert g-S_{n}g\rVert_{2}}_{\overset{18.2(4)}{\leq}\lVert g-v\lVert_2}
		+\underbrace{\lVert S_{n}(g-f)\rVert_{2}}_{\overset{18.2(3)}{\leq}\lVert g-f\lVert_2}\\
    &<2\ep+\sqrt{2\pi}\ep=\ep(2+\sqrt{2\pi})
\end{align*}
Sei \(m\geq n\). Dann gilt: \(\mathrm{E}_{n}\subseteq\mathrm{E}_{m}\), also \(w:=S_{n}f\in\mathrm{E}_{m}\). Damit:
\[
\lVert f-S_{m}f\rVert_{2}\leq\lVert f-w\rVert_{2}=\lVert f-S_{n}f\rVert_{2}<\ep(2+\sqrt{2\pi})
\]
\end{beweis}

\subsubsection*{Reelle Version}
Sei \(f\in\mathrm{L}^{2}_{\mdr}\).
Es gelten die folgenden Bezeichnungen:
\begin{enumerate}
\item Für \(k\in\mdn\) bezeichnen wir die Funktionen \(t\mapsto\cos(kt)\) und \(t\mapsto\sin(kt)\) mit \(\cos(k\cdot)\) bzw.
    \(\sin(k\cdot)\).
\item Für \(k\in\mdn_{0}:\,\alpha_{k}:=\frac{1}{\pi}\int_{0}^{2\pi}{f(t)\cos(kt)\mathrm{d}t}=\frac{1}{\pi}\Re(f\mid e_{k})\).\\
Für \(k\in\mdn:\,\beta_{k}:=\frac{1}{\pi}\int_{0}^{2\pi}{f(t)\sin(kt)\mathrm{d}t}=\frac{1}{\pi}\Im(f\mid e_{k}),\,\beta_{0}:=0\).
\end{enumerate}

\begin{definition}
\index{gerade Funktion}
\index{ungerade Funktion}
\(f\) heißt \textbf{gerade} (bezüglich \(\pi\)) genau dann, wenn gilt: \(f(t)=f(2\pi-t)\) für fast alle \(t\in[0,2\pi]\).\\
\(f\) heißt \textbf{ungerade} (bezüglich \(\pi\)) genau dann, wenn gilt: \(f(t)=-f(2\pi-t)\) für fast alle \(t\in[0,2\pi]\).\\
% Bild nicht vergessen
\end{definition}

\begin{satz}
\label{Satz 18.6}
(Dieser Satz folgt aus \ref{Satz 18.5} und ``etwas'' rechnen)\\
Sei \(f\in\mathrm{L}^{2}_{\mdr}\) und \(n\in\mdn_{0}\).
\begin{enumerate}
\item \(S_{n}f=\frac{\alpha_{0}}{2}+\sum_{k=1}^{n}{(\alpha_{k}\cos(k\cdot)+\beta_{k}\sin(k\cdot))}\)
\item \(f\overset{\lVert\cdot\rVert_{2}}{=}\frac{\alpha_{0}}{2}+\sum_{k=1}^{\infty}{(\alpha_{k}\cos(k\cdot)+\beta_{k}\sin(k\cdot))}\)
\item \(\frac{1}{\pi}\lVert f\rVert_{2}^{2}=\frac{\alpha_{0}^{2}}{2}+\sum_{k=1}^{\infty}{(\alpha_{k}^{2}+\beta_{k}^{2})}\quad\)
    (Parsevalsche Gleichung)\\
Insbesondere gilt: \(\alpha_{k}\to0,\,\beta_{k}\to0\quad(k\to\infty)\)
\item Ist \(f\) gerade, so sind alle \(\beta_{k}=0\) und \(\alpha_{k}=\frac{2}{\pi}\int_{0}^{\pi}{f(t)\cos(kt)\mathrm{d}t}\). Die
Fourierreihe von \(f\) ist eine \textbf{Cosinusreihe}.\\
Ist \(f\) ungerade, so sind alle \(\alpha_{k}=0\) und \(\beta_{k}=\frac{2}{\pi}\int_{0}^{\pi}{f(t)\sin(kt)\mathrm{d}t}\). Die
Fourierreihe von \(f\) ist eine \textbf{Sinusreihe}.
\end{enumerate}
\end{satz}

\begin{beispiele}
\begin{enumerate}
\item \(f(t):=\begin{cases}1,&0\leq t\leq\pi\\-1,&\pi<t\leq 2\pi\end{cases}\)

\(f\) ist ungerade, also \(\alpha_{k}=0\,\forall k\in\mdn_{0}\). Es ist
\(\beta_{k}=\frac{2}{\pi}\int_{0}^{\pi}{\sin(kt)\mathrm{d}t}=\begin{cases}0,&k\text{ gerade}\\\frac{4}{k\pi},&k\text{ ungerade}\end{cases}\).\\
Damit:
\[
f\overset{\lVert\cdot\rVert_{2}}{=}\frac{4}{\pi}\sum_{j=0}^{\infty}{\frac{\sin((2j+1)\cdot)}{2j+1}}
\]
Beachte: \((S_{n}f)(0)=0\to 0\neq1=f(0)\) und \((S_{n}f)(2\pi)=0\to 0\neq -1=f(2\pi)\).
\item \(f(t):=\begin{cases}t,&0\leq t\leq\pi\\2\pi-t,&\pi\leq t\leq 2\pi\end{cases}\)\\
\(f\) ist gerade, das heißt \(\beta_{k}=0\,\forall k\in\mdn\) und \(\alpha_{k}=\frac{2}{\pi}\int_{0}^{\pi}{t\cos(kt)\mathrm{d}t},\,\alpha_{0}=\pi\).\\
Für \(k\geq 1:\quad\alpha_{k}=\begin{cases}0,&k\text{ gerade}\\-\frac{4}{\pi k^{2}},&k\text{ ungerade}\end{cases}\).\\
Damit:
\[
f\overset{\lVert\cdot\rVert_{2}}{=}\frac{\pi}{2}-\frac{4}{\pi}\sum_{j=0}^{\infty}{\frac{\cos((2j+1)\cdot)}{(2j+1)^{2}}}
\]
\end{enumerate}
\end{beispiele}
% Ende der reellen Version

\begin{satz}
\label{Satz 18.7}
Sei $f \in L^2$ und $\sum_{k \in \MdZ} |(f|b_k)| < \infty$. Dann:
\begin{enumerate}
\item Die Reihe $\sum_{k \in \MdZ} (f\mid b_k) b_k(t)$ konvergiert auf $[0, 2 \pi ]$ absolut und gleichmäßig.
Setzt man $g(t) := \sum_{k \in \MdZ} (f\mid b_k)b_k(t)$ für $t \in [0, 2\pi ]$, so ist $g$ stetig, $g(0)=g(2\pi )$ und $f=g$ f.ü. auf $[0,2 \pi ]$.
\item Ist $f$ stetig, so gilt $f=g$ auf $[0,2\pi ]$, also:
\begin{equation*}
\label{Gleichung 2, Satz 18.7}
f(t)=\sum_{k\in\MdZ}(f\mid b_k)b_k(t)\quad\forall t\in[0,2\pi]
\end{equation*}
Insbesondere: $f(0)=f(2\pi)$
\end{enumerate}
\end{satz}

\begin{beweis}
\begin{enumerate}
\item $f_k(t) := (f\mid b_k)b_k(t)$;
\[
\lvert f_k(t)\rvert=\lvert(f\mid b_k)\rvert\cdot\lvert b_k(t)\rvert=\frac{1}{\sqrt{2\pi}}\lvert(f\mid b_k)\rvert\quad\forall t \in [0,2\pi ] \forall k \in \MdZ
\]
Aus Analysis I, 19.1(2) (Konvergenzkriterium von Weierstraß) folgt: Die Reihe in (1) konvergiert auf $[0,2\pi]$ absolut und gleichmäßig.
Aus Analysis I, 19.2 folgt: $g$ ist stetig.
Klar: $g(0) = g(2\pi )$.
\[ s_n(t) := \sum_{\lvert k\rvert \leq n} f_k(t) \quad (n \in \MdN_0, t \in [0,2\pi ]).\]
Aus \ref{Satz 18.5} folgt: $\| f-s_n \|_2 \to 0 (n\to \infty )$.
$\| g-s_n \|_2 \overset{18.3(2)}{\leq} \| g-s_n \|_\infty \sqrt{2\pi } \to 0 (n\to \infty )$
Also: $\| g -s_n\|_2 \to 0 (n \to \infty)$
Aus \ref{Satz 16.5} folgt: $f=g$ f.ü.
\item $f=g$ f.ü. $\overset{18.3(3)}{\implies}\,f=g$ auf $[0,2\pi]$.
\end{enumerate}
\end{beweis}

\begin{satz}
\label{Satz 18.8}
$f \in L^2_\MdR$ und die Folgen $(\alpha_k )$ und $(\beta_k )$ seien definiert wie im Abschnitt ``Reelle Version''. Weiter gelte: $\sum_{k=1}^\infty\lvert\alpha_k\rvert<\infty$ und $\sum_{k=1}^\infty\lvert\beta_k\rvert<\infty$. Dann gelten die Aussagen in \ref{Satz 18.7} für die Reihen in \ref{Satz 18.6}.
\end{satz}

\begin{satz}
\label{Satz 18.9}
Sei $f:[0,2\pi] \to \MdC$ \textbf{stetig differenzierbar} und $f(0)=f(2\pi)$.
\begin{enumerate}
\item Es ist $(f'\mid b_k)=ik(f\mid b_k)\quad\forall k\in\MdZ$
\item $\sum_{k\in\MdZ}\lvert(f\mid b_k)\rvert<\infty$  (d.h.: die Voraussetzungen von \ref{Satz 18.7} sind erfüllt)
\end{enumerate}
\end{satz}

\begin{beweis}
\begin{enumerate}
\item \begin{align*}
(f'|b_k) &= \frac{1}{\sqrt{2\pi}} \int_0^{2\pi} f'(t)e^{-ikt} \text{ d}t  \\
&\overset{P.I.}{=} \frac{1}{\sqrt{2\pi}} \left[ f(t)e^{-ikt} \right]_0^{2\pi} - \frac{1}{\sqrt{2\pi}}\int_0^{2\pi} f(t)(-ik)e^{-ikt}\text{ d}t \\
&= \frac{1}{\sqrt{2\pi}}(f(2\pi ) - f(0)) + ik(f|b_k).
\end{align*}
\item Setze $\sigma_n := \sum_{|k|\leq n} |(f|b_k)| \quad (n \in \MdN_0)$. Es genügt zu zeigen: $(\sigma_n )$ ist beschränkt. Klar: $0 \leq \sigma_n$.
\begin{align*}
\sigma_n - |(f|b_0)| &= \sum_{0<|k|\leq n} |(f|b_k)| \overset{(1)}{=} \sum_{0<|k|\leq n} \underbrace{\frac{1}{|k|}}_{:= u_k}\underbrace{(f'|b_k)}_{:= v_k} \\
&= \sum_{0<|k|\leq n} u_k v_k \overset{\text{CS-Ugl.}}{\leq} \left( \sum_{0<|k|\leq n} u_k^2 \right)^\frac{1}{2} \left( \sum_{0<|k|\leq n} v_k^2 \right)^\frac{1}{2}\\
&= \left( 2\sum_{k=1}^n u_k^2 \right)^\frac{1}{2} \underbrace{ \left( \sum_{0<|k|\leq n} v_k^2 \right)^\frac{1}{2} }_{ \overset{18.2(3)}{\leq} \|f'\|_2} \\
&\leq \left( 2\sum_{k=1}^\infty u_k^2 \right)^\frac{1}{2} \| f' \|_2
\end{align*}
\end{enumerate}
\end{beweis}

\begin{beispiel}
\begin{enumerate}
\item $f$ sei wie im Beispiel (2) vor \ref{Satz 18.7}. Es war:
\[ f \overset{\| \cdot \|_2}{=} \frac{\pi}{2} - \frac{4}{\pi} \sum_{j=0}^\infty \frac{\cos((2j+1) \cdot )}{(2j+1)^2} \quad \quad \left(\alpha_{2j+1} = \frac{1}{(2j+1)^2}, \alpha_{2j} = 0  \right) \]
Aus \ref{Satz 18.7} bzw. \ref{Satz 18.8} folgt:
\[  f(t) = \frac{\pi}{2} - \frac{4}{\pi} \sum_{j=0}^\infty \frac{\cos((2j+1) t )}{(2j+1)^2} \quad \forall t \in [0,2\pi] \]
Setzt man nun $t=0$, folgt
\[ 0 = \frac{\pi}{2} - \frac{4}{\pi} \sum_{j=0}^\infty \frac{1}{(2j+1)^2} \]
und man erhält durch Umstellen eine Auswertung für diese eigentlich kompliziert wirkende Reihe:
\[ \sum_{j=0}^\infty \frac{1}{(2j+1)^2} = \frac{1}{1^2} + \frac{1}{3^2} + \frac{1}{5^2} + \dots = \frac{\pi^2}{8} \]
(dass diese Reihe konvergiert, ist eine einfache Übung aus Ana I; ihren Wert aber haben wir bislang noch nicht berechnet)

\item $f(t) = (t - \pi)^2 \quad (t \in [0,2\pi])$. $f$ ist gerade bzgl. $\pi$, also ist $\beta_k = 0$. Es ist
\[ \alpha_k = \begin{cases} \frac{2}{3}\pi^2, &k=0\\ \frac{4}{k^2}, &k \geq 1 \end{cases} \quad \text{(nachrechnen!)}\]
Also:
\[ f \overset{\| \cdot \|_2}{=} \frac{\pi^2}{3} + 4 \sum_{j=1}^\infty \frac{\cos(j \cdot)}{j^2} \]
Aus \ref{Satz 18.9} bzw. \ref{Satz 18.7}(2) folgt:
\[ f(t) = \frac{\pi^2}{3} + 4 \sum_{j=1}^\infty \frac{\cos(j t)}{j^2}  \quad \forall t \in [0, 2\pi] \]
Setzt man nun $t=0$, erhält man
\[ \pi^2 = \frac{\pi^2}{3} + 4 \sum_{j=1}^\infty \frac{1}{j^2}, \text{ also } \sum_{j=1}^\infty \frac{1}{j^2} = \frac{\pi^2}{6} \]
Damit erhält man z.B. auch
\[ \sum_{j=1}^\infty \frac{1}{(2j)^2} = \frac{1}{4} \sum_{j=1}^\infty \frac{1}{j^2} = \frac{\pi^2}{24} \]
und damit
\[ \sum_{j=1}^\infty \frac{(-1)^{j+1}}{j^2} = \frac{1}{1^2} - \frac{1}{2^2} + \frac{1}{3^2} - \frac{1}{4^2} \pm \dots = \frac{\pi^2}{8} - \frac{\pi^2}{24} = \frac{\pi^2}{12} \]

\end{enumerate}

\end{beispiel}
