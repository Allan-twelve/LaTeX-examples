Die Bezeichnungen seien wie in den Kapitel 8 und 9.

\begin{satz}[Satz von Tonelli]
\label{Satz 10.1}
Es sei \(f\colon\mdr^d\to[0,+\infty]\) messbar. (Aus \S 8 folgt dann, dass \(f^x,f_y\) messbar sind, wobei klar ist, dass \(f^x,f_y\geq 0\) sind.)\\
Für \(x\in\mdr^k\):
\[F(x):=\int_{\mdr^l}f(x,y)\,dy=\int_{\mdr^l}f^x(y)\,dy\]
Für \(y\in\mdr^l\):
\[G(y):=\int_{\mdr^k}f(x,y)\,dx=\int_{\mdr^k}f_y(x)\,dx\]
Dann sind $F,G$ messbar und
\[\int_{\mdr^d}f(z)\,dz=\int_{\mdr^k}F(x)\,dx=\int_{\mdr^l}G(y)\,dy\]
also
\begin{align*}
\tag{$*$}\int_{\mdr^d}f(x,y)\,d(x,y)=\int_{\mdr^k}\left(\int_{\mdr^l}f(x,y)\,dy\right)dx=\int_{\mdr^l}\left(\int_{\mdr^k}f(x,y)\,dx\right)dy
\end{align*}
\textbf{(iterierte Integrale)}
\end{satz}

\begin{beweis}
\textbf{Fall 1:} Sei \(C\in\fb_d\) und \(f=\mathds{1}_{C}\). Die Behauptungen folgen dann aus \ref{Satz 9.1}.\\
\textbf{Fall 2:} Sei \(f\geq 0\) und einfach. Die Behauptungen folgen aus Fall 1, \ref{Satz 3.6} und \ref{Satz 4.5}.\\
\textbf{Fall 3 - Der allgemeine Fall:}\\
 Sei \((f_n)\) zulässig für $f$, also: \(0\leq f_n\leq f_{n+1}\), \(f_n\) einfach und \(f_n\to f\) auf \(\mdr^d\).
Für \(x\in\mdr^k\) und \(\natn\) gilt:
\[F_n(x):=\int_{\mdr^l}f_n(x,y)\,dy\]
und nach Fall 2 ist \(F_n\) messbar. \\
Aus \(0\leq f_n\leq f_{n+1}\) folgt \(0\leq F_n\leq F_{n+1}\) und \ref{Satz 4.6} liefert \(F_n\to F\) auf \(\mdr^k\). Dann gilt
\[\int_{\mdr^d}f(z)\,dz = \lim \int_{\mdr^d}f_n(z)\,dz \overset{Fall 2}= \lim \int_{\mdr^k}F_n(x)\,dx \overset{\ref{Satz 4.6}}=\int_{\mdr^k}F(x)\,dx\]
Genauso zeigt man
\[\int_{\mdr^d}(f(z)\,dz=\int_{\mdr^l}G(y)\,dy\]
\end{beweis}

\begin{satz}[Satz von Fubini (Version I)]
\label{Satz 10.2}
Es sei \(f\colon\mdr^d\to\imdr\) integrierbar. Dann existieren Nullmengen \(M\subseteq\mdr^k\) und \(N\subseteq\mdr^l\) mit
\begin{align*}
	 f^x\colon\mdr^l\to\imdr \text{ ist integrierbar für jedes } x\in\mdr^k\setminus M \\
	 f_y\colon\mdr^k\to\imdr \text{ ist integrierbar für jedes } y\in\mdr^l\setminus N
\end{align*}
Setze
\begin{align*}
	F(x):=
	\begin{cases}
		\int_{\mdr^l}f^x(y)\,dy=\int_{\mdr^l}f(x,y)\,dy	& \text{, falls } x\in\mdr^k\setminus M \\
		0							& \text{, falls } x\in M
	\end{cases}
\intertext{und}
	G(y):=
	\begin{cases}
		\int_{\mdr^k}f_y(x)\,dx=\int_{\mdr^k}f(x,y)\,dx	& \text{, falls } y\in\mdr^l\setminus N \\
		0							& \text{, falls } y\in N
	\end{cases}
\end{align*}
Dann sind $F$ und $G$ integrierbar und es gelten folgende zwei Gleichungen
\[ \int_{\mdr^d}f(z)\,dz = \int_{\mdr^k}F(x)\,dx = \int_{\mdr^l}G(y)\,dy \]
Es gilt also wieder \((\ast)\) aus \ref{Satz 10.1}.
\end{satz}

\begin{beweis}
Wir zeigen nur die Aussagen über \(f^x\), $F$ und die erste der obigen beiden Gleichungen. Genauso zeigt man die Aussagen über \(f_n, G\) und die zweite Gleichung.\\
Aus \ref{Lemma 8.1} folgt, dass \(f^x\) messbar ist. Definiere
	\begin{align*}
	\Phi(x) := \int_{\mdr^l}\lvert f^x(y)\rvert\,dy
	= \int_{\mdr^l}\lvert f(x,y)\rvert\,dy \ \text{ für } x\in\mdr^k
	\end{align*}
Nach \ref{Satz 10.1} ist \(\Phi\) messbar und
	\begin{align*}
	\int_{\mdr^k}\Phi(x)\,dx
	= \int_{\mdr^k}\left(\int_{\mdr^l}\lvert f(x,y)\rvert\,dy\right)dx \overset{\ref{Satz 10.1}}
	= \int_{\mdr^d}\lvert f(z)\rvert\,dz
	< \infty
	\end{align*}
(denn mit $f$ ist nach \ref{Satz 4.9} auch \(\lvert f\rvert\) integrierbar). Somit ist \(\Phi\) integrierbar.
Setze \(M:=\{\Phi = \infty \}\) was nach \ref{Satz 4.10} eine Nullmenge ist.
Also gilt:
	\begin{align*}
	\int_{\mdr^l}\lvert f^x(y)\rvert\,dy
	= \Phi(x) < \infty \ \text{ für jedes } x\in\mdr^k\setminus M
	\end{align*}
Das heißt, \(\lvert f^x\rvert\) ist für jedes \(x\in\mdr^k\setminus M\) integrierbar und es gilt nach \ref{Satz 4.9} auch
	\begin{align*}
	f^x \text{ ist integrierbar für jedes } x\in\mdr^k\setminus M
	\end{align*}
Aus \ref{Folgerung 9.2} folgt, dass \(M\times\mdr^l\) eine Nullmenge ist.
Setze
	\begin{align*}
	\tilde f(z):=
		\begin{cases}
		f(z)	&\text{, falls } z\in\mdr^d\setminus(M\times\mdr^l)\\
		0	&\text{, falls } z\in M\times\mdr^l
		\end{cases}
	\end{align*}
Aus \ref{Lemma 9.3} folgt, dass \(\tilde f\) messbar ist. Klar ist, dass fast überall \(f=\tilde f\) gilt. Es ist
\[\tilde f^x = \left(\mathds{1}_{(M\times\mdr^l)^C}\cdot f\right)^x\]
Das heißt \(\tilde f^x\) ist integrierbar für jedes \(x\in\mdr^k\). Dann gilt
	\begin{align*}
	 F(x) \overset{\ref{Satz 5.3}}
	= \int_{\mdr^l}\tilde f(x,y)\,dy
	= \underbrace{\int_{\mdr^l}\tilde f_+ (x,y)\,dy}_{=:F^+(x)} - \underbrace{\int_{\mdr^l}\tilde f_- (x,y)\,dy}_{=:F^-(x)}
	\end{align*}
Nach \ref{Satz 10.1} sind \(F^+\) und \(F^-\) messbar. Die Dreiecksungleichung liefert nun
	\begin{align*}
	\lvert F(x)\rvert
	\leq \int_{\mdr^l}\lvert \tilde f(x,y)\rvert\,dy
	\overset{\ref{Satz 5.3}}= \int_{\mdr^l}\lvert f(x,y)\rvert\,dy
	= \Phi(x) \ \text{ für } x\in\mdr^k
	\end{align*}
Also ist \(\lvert F\rvert\leq\Phi\) und \(\Phi\) ist integrierbar. Aus \ref{Satz 4.9} folgt, dass $F$ und \(\lvert F\rvert\) integrierbar sind
und dann sind auch \(F^+\) und \(F^-\) integrierbar (zur Übung). Es folgt
	\begin{align*}
	\int_{\mdr^k}F(x)\,dx
	& = \int_{\mdr^k}F^+(x)\,dx - \int_{\mdr^k}F^-(x)\,dx 											\\
	& = \int_{\mdr^k} \left(\int_{\mdr^l} \tilde f_+(x,y)\,dy\right)dx - \int_{\mdr^k} \left(\int_{\mdr^l}\tilde f(x,y)\,dy\right)dx 		\\
	& \overset{\ref{Satz 10.1}}= \int_{\mdr^d}\tilde f_+(z)\,dz - \int_{\mdr^d}\tilde f_-(z)\,dz 						\\
	& = \int_{\mdr^d}\tilde f(z)\,dz 														\\
	& = \int_{\mdr^d}f(z)\,dz
	\end{align*}
\end{beweis}

\begin{satz}[Satz von Fubini (Version II)]
\label{Satz 10.3}
Sei \(\emptyset\neq X\in\fb_k\), \(\emptyset\neq Y\in\fb_l\) und \(D:=X\times Y\) (nach \S 8 ist \(D\in\fb_d\)).
Es sei \(f\colon D\to\imdr\) messbar.
Ist \(f\geq 0\) auf $D$ oder ist $f$ integrierbar, so gilt
\[ \int_D f(x,y)\,d(x,y) = \int_X\left(\int_Yf(x,y)\,dy\right)dx = \int_Y\left(\int_Xf(x,y)\,dx\right)dy \]
\end{satz}

\begin{beweis}
Definiere \(\tilde f\) wie in \ref{Lemma 9.3} und wende \ref{Satz 10.1} beziehungsweise \ref{Satz 10.2} an.
\end{beweis}

\begin{bemerkung}
\ref{Satz 10.1}, \ref{Satz 10.2} und \ref{Satz 10.3} gelten natürlich auch für mehr als zwei iterierte Integrale.
\end{bemerkung}

\textbf{"'Gebrauchsanweisung"' für Fubini:}\\
Gegeben: \(\emptyset\neq D\subseteq\fb_d\) und messbares \(f\colon D\to\imdr\).
Setze $f$ auf \(\mdr^d\) zu einer messbaren Funktion \(\tilde f\) fort (zum Beispiel wie in \ref{Lemma 9.3}).
Aus \ref{Satz 3.8} folgt dann, dass \(\mathds{1}_{D}\tilde f\) messbar ist und \ref{Satz 10.1} liefert
	\begin{align*}
	\int_{\mdr^d}\lvert \mathds{1}_{D}\tilde f\rvert\,dz
	= \int_{\mdr^k}\left(\int_{\mdr^l}\lvert \mathds{1}_{D}\tilde f\rvert\,dy\right)dx
	= \int_{\mdr^l}\left(\int_{\mdr^k}\lvert \mathds{1}_{D}\tilde f\rvert\,dx\right)dy
	\end{align*}
Ist eines der drei obigen Integrale endlich, so ist \(\lvert \mathds{1}_{D}\tilde f\rvert\) integrierbar und
damit ist nach \ref{Satz 4.9} auch \(\mathds{1}_{D}\tilde f\) integrierbar.\\
Dann ist $f$ integrierbar und es folgt
	\begin{align*}
	\int_Df(z)\,dz
	& = \int_{\mdr^d}\left(\mathds{1}_{D}\tilde f\right)(z)\,dz									\\
	& \overset{\ref{Satz 10.2}}= \int_{\mdr^k}\left(\int_{\mdr^l}\left(\mathds{1}_{D}\tilde f\right)(x,y)\,dy\right)dx 	\\
	& = \int_{\mdr^l}\left(\int_{\mdr^k}\left(\mathds{1}_{D}\tilde f\right)(x,y)\,dx\right)dy
	\end{align*}

\begin{beispiel}
\begin{enumerate}
\item Sei \(D=[a_1,b_1]\times[a_2,b_2]\times\dots\times[a_d,b_d]\) mit \(a_i\leq b_i \ (i=1,\dots,d)\).
Es sei \(f\colon D\to\mdr\) stetig. $D$ ist kompakt, also gilt \(D\in\fb_d\).
Nach \ref{Satz 4.12}(2) ist \(f\in\mathfrak{L}^1(D)\) und aus obiger Bemerkung folgt
	\begin{align*}
	\int_Df(x_1,\dots,x_d)\,d(x_1,\dots,x_d)
	= \int_{a_d}^{b^d} \left(\dots \left( \int_{a_2}^{b^2} \left(\int_{a_1}^{b^1}f(x_1,\dots,x_d)\,dx_1\right)dx_2\right)\dots\right)dx_d
	\end{align*}
Die Reihenfolge der Integrationen darf beliebig vertauscht werden. Aus \ref{Satz 4.13} folgt
\[\int_{a_i}^{b_i}\dots \text{ d}x_i= \text{R-}\int_{a_i}^{b_i}\dots\text{ d}x_i\]

\textbf{Konkretes Beispiel}\\
Sei  \(D:=[a,b]\times[c,d]\subseteq\mdr^2\), \(f\in C([a,b])\) und \(g\in C([c,d])\).
	\begin{align*}
	\int_Df(x)g(y)\,d(x,y)
	& = \int_c^d\left(\int_a^bf(x)g(y)\,dx\right)dy			\\
	& = \int_c^d\left(g(y)\left(\int_a^bf(x)\,dx\right)\right)dy		\\
	&= \left(\int_a^bf(x)\,dx\right) \left(\int_c^dg(y)\,dy\right)
	\end{align*}
\item
	Wir rechtfertigen die "'Kochrezepte"' aus Analysis II, Paragraph 15.
	Seien \(a,b\in\mdr\) mit \(a<b\) und \(I:=[a,b]\). Weiter seien
	\(h_1,h_2\in C(I)\) mit \(h_1\leq h_2\) auf \(I\) und
	\[A:=\{(x,y)\in\mdr^2: x\in I, h_1(x)\leq y\leq h_2(x)\}\]
	Sei \(f\colon A\to\mdr\) stetig. Da \(h_1\) und \(h_2\) stetig
	sind, ist \(A\) kompakt und somit gilt \(A\in\fb_2\). Aus
	\ref{Satz 4.12}(2) folgt dann \(f\in\mathfrak{L}^1(A)\).
	Definiere
	\[\tilde f(x,y)=
		\begin{cases}
		f(x,y) 	&\text{, falls } (x,y)\in A  	\\
		0	&\text{, falls } (x,y)\notin A
		\end{cases}
	\]
	Nach \ref{Lemma 9.3} ist \(\tilde f\) messbar. Setze
	\[M:=\max\{\lvert f(x,y)\rvert:(x,y)\in A\}\]
	Dann gilt \(\lvert\tilde f\rvert \leq M\cdot\mathds{1}_A\).
	Wegen \(\lambda_2(A)<\infty\) ist \(M\cdot\mathds{1}_A\)
	integrierbar und nach \ref{Satz 4.9} ist \(\lvert\tilde f\rvert\)
	und damit auch \(\tilde f\) integrierbar. Dann ist
	\begin{align*}
		\int_A f(x,y)\,d(x,y) &= \int_{\mdr^2}\tilde f(x,y)\,d(x,y) \\
		& \overset{\ref{Satz 10.3}}=
		\int_\mdr\left(\int_\mdr\tilde f	(x,y)\,dy\right)dx \\
		&=\int_a^b\left(\int^{h_2(x)}_{h_1(x)}f(x,y)\,dy\right)dx
	\end{align*}
	Damit ist 15.1 aus Analysis II bewiesen. Genauso zeigt man 15.3.
\item
	Sei \(D:=\{(x,y)\in\mdr^2:x\geq 1, 0\leq y\leq\frac1x\}\) und
	\(f(x,y):=\frac1x\cos(xy)\). $D$ ist abgeschlossen und somit ist
	\(D\in\fb_2\). Außerdem ist $f$ stetig, also messbar. \\
	\textbf{Behauptung: } \[f\in\mathfrak{L}^1(D)\text{ und }\int_Df(x,y)\,d(x,y)=\sin(1)\]
	\textbf{Beweis: } Setze \(X:=(0,\infty)\), \(Y:=[0,\infty)\) und
	\(Q:=X\times Y\). Sei nun \[\tilde f(x,y):=\frac1x\cos(xy) \text{ für }
	(x,y)\in Q\]
	\(\tilde f\) ist eine Fortsetzung von \(f\) auf \(X\times Y\).
	\(\tilde f\) ist also messbar. Es ist
	\begin{align*}
		\int_D\lvert f\rvert\,d(x,y)
		&=\int_Q\mathds{1}_D\cdot\lvert\tilde f\rvert\,d(x,y)	\\
		&\overset{\ref{Satz 10.1}}=
		\int_X\left(\int_Y\mathds{1}_D(x,y)\frac1x\lvert\cos(xy)\rvert
		\,dy\right)dx 						\\
		&\int^\infty_1\left(\int^\frac1x_0 \frac1x\lvert\cos(xy)\rvert
		\,dy\right)dx						\\
		&\leq \int^\infty_1\left(\int^\frac1x_0 \frac1x\,dy\right)dx \\
		&=\int^\infty_1\frac1{x^2}\,dx = 1<\infty
	\end{align*}
	Also ist \(\lvert f\rvert\) integrierbar und dann nach \ref{Satz 4.9}
	auch $f$, also \(f\in\mathfrak{L}^1(D)\). Dann:
	\begin{align*}
		\int_D f\,d(x,y)
		&= \int_X\left(\int_Y\mathds{1}_D(x,y)\frac1x\cos(xy)\,dy\right)
		dx							\\
		&\overset{\text{wie oben}}=
		\int^\infty_1\left(\int^\frac1x_0 \frac1x\cos(xy)\,dy\right)dx\\
		&= \left. \int^\infty_1\left(\frac1x\cdot\frac1x\sin(xy)
		\right\rvert^{y=\frac1x}_{y=0}\right)dx			\\
		&= \int^\infty_1\frac1{x^2}\sin(1)\,dx			\\
		&= \sin(1)
	\end{align*}
\end{enumerate}
\end{beispiel}

\textbf{Vorbemerkung: } Sei \(x>0\). Für \(b>0\) gilt
\begin{align*}
	\int^b_0 e^{-xy}\,dy = \left. -\frac1x e^{-xy}\right\rvert^b_0
	=-\frac1x e^{-xb}+\frac1x
	\overset{b\to\infty}\longrightarrow\frac1x
\end{align*}
und daraus folgt \(\int_0^\infty e^{-xy}\,dy=\frac1x\)

\begin{beispiel}
\begin{enumerate}
\item[(4)]
	Sei
	\[g:=
	\begin{cases}
		\frac{\sin x}{x}	&\text{, falls } x>0	\\
		1			&\text{, falls } x=0
	\end{cases}\]
	$g$ ist stetig auf \([0,\infty)\). Aus Analysis 1 ist bekannt, dass
	\(\int_0^\infty g(x)\,dx\) konvergent, aber \textbf{ nicht }
	absolut konvergent ist. Aus \ref{Satz 4.14} folgt, dass
	\(g\notin\mathfrak{L}^1\left([0,\infty)\right)\)\\
	\textbf{Behauptung: } \(\int^\infty_0 g(x)\,dx = \frac\pi{2}\)\\
	\textbf{Beweis: } Setze \(X:=[0,R]\) mit \(R>0\), \(Y:=[0,\infty)\) und
	\(D:=X\times Y\), sowie
	\[f(x,y):= e^{-xy}\sin x \text{ für } (x,y)\in D\]
	Es ist \(D\in\fb_2\) und $f$ stetig, also messbar. Es ist weiter
	\(f\in\mathfrak{L}^1(D)\) (warum?) und
	\begin{align*}
		\int_D f(x,y)\,d(x,y)
		&\overset{\ref{Satz 10.3}}=
		\int_X\left(\int_Y f(x,y)\,dy\right)dx			\\
		&=\int_0^R\left(\int_0^\infty e^{-xy}\sin x\,dy\right)dx\\
		&=\int^R_0\sin x\left(\int_0^\infty e^{-xy}\,dy\right)dx\\
		&\overset{\text{Vorbemerkung}}=
		\int^R_0\frac{\sin x}{x}\,dx =:I_R
	\end{align*}
	Dann gilt
	\begin{align*}
		I_R
		&\overset{\ref{Satz 10.3}}=
		\int_Y\left(\int_X f(x,y)\,dx\right)dy
		=\int^\infty_0\underbrace{
		\left(\int^R_0 e^{-xy}\sin x\,dx\right)}_{=:\varphi(y)}dy
	\end{align*}
	Zweimalige partielle Integration liefert (nachrechnen!):
	\[\varphi(y)=\frac1{1+y^2}-\frac1{1+y^2}e^{-yR}(y\sin R+\cos R)\]
	Damit gilt
	\begin{align*}
		I_R=
		\int^\infty_0 \frac{dy}{1+y^2}
		-\int^\infty_0\frac1{1+y^2}e^{-yR}(y\sin R+\cos R)\,dy
	\end{align*}
	Aus Analysis 1 ist bekannt, dass das erste Integral gegen
	\(\frac{\pi}2\) konvergiert und das zweite Integral setzen
	wir gleich \(\tilde I_R\).\\
	Es gilt
	\begin{align*}
		\lvert\tilde I_R\rvert
		&\leq \int^\infty_0\frac1{1+y^2}e^{-yR}
		(y\lvert\sin R\rvert + \lvert\cos R\rvert)\,dy	\\
		&\leq \int^\infty_0\frac{y+1}{y^2+1} e^{-yR}\,dy\\
		&\leq 2\int^\infty_0 e^{-yR}\,dy		\\
		&\overset{\text{Vorbemerkung}}=\frac2R
	\end{align*}
	Das heißt also \(\tilde I_R\to 0 \ (R\to\infty)\) und damit folgt
	die Behauptung durch
	\[I_R=\frac{\pi}2-\tilde I_R\to\frac{\pi}2 \ (R\to\infty)\]
\end{enumerate}
\end{beispiel}
