\documentclass[a4paper]{scrartcl}
\usepackage{amssymb, amsmath} % needed for math
\usepackage[utf8]{inputenc} % this is needed for umlauts
\usepackage[english]{babel} % this is needed for umlauts
\usepackage[T1]{fontenc}    % this is needed for correct output of umlauts in pdf
\usepackage[margin=2.5cm]{geometry} %layout
\usepackage{hyperref}   % links im text
\usepackage{color}
\usepackage{framed}
\usepackage{enumerate}  % for advanced numbering of lists
\usepackage{csquotes}
\usepackage{ifxetex,ifluatex}
\usepackage{etoolbox}
\usepackage[svgnames]{xcolor}
\usepackage{tikz}
\usepackage{framed}
\usepackage{parskip}
\usepackage{cite}
\usepackage{mystyle}
\clubpenalty  = 10000   % Schusterjungen verhindern
\widowpenalty = 10000   % Hurenkinder verhindern

\hypersetup{
  pdfauthor   = {Martin Thoma},
  pdfkeywords = {Bachelor proposal: },
  pdftitle    = {Bachelor proposal}
}

%%%%%%%%%%%%%%%%%%%%%%%%%%%%%%%%%%%%%%%%%%%%%%%%%%%%%%%%%%%%%%%%%%%%%

\begin{document}
    \title{Proposal for a Bachelor of Science Thesis:\\Recognition of mathematical formulae in the Context of Lecture Translation}
    \author{Martin Thoma}
    \maketitle
\section{The problem backgound}
    The KIT Lecture Translator, CMUSphinx, Android voice typing and
    many other speech recognition systems have proven that it is possible to
    recognize speech. But at the moment, there seems not to be a single
    system that manages to recognize natural language math speech
    recognition. For example, a term like
    \[\sum_{n=1}^\infty \frac{1}{n^2} \rightarrow \infty \]
    would naturally be spoken as

\begin{shadequote}[l]{}
The sum of one divided by n squared for n from one to infinity diverges to infinity.
\end{shadequote}

    in natural language. Today, speech recognition systems do only
    recognize the words spoken. They don't recognize that it was a
    mathematical term which could and should be expressed with symbols.

    One way to extend an existing speech recognition $A$ systems would be
    by the following steps:
    \begin{enumerate}
        \item $A$ recognizes speech and returns a text $T$. This text
              has to contain anotations that indicate at which time
              in the original recording the various parts of speech
              were detected.
        \item A math detecter parses $T$ and returns the time intervalls $I$
              when math was detected.
        \item A math parser tries to parse speech in $I$. This parser
              can make use of a language model dedicated to math. It
              returns weighted hypotheses which terms might have
              been spoken.
        \item Finally, a program compares the hypotheses with math
              in a formula database. Many formulas might already been
              written in \TeX{}, e.g. on Wikipedia, math.stackexchange.com
              or in freely available \LaTeX{} / \TeX{} files.
    \end{enumerate}
\break

\section{The problem statement}
    The bachelor's thesis at KIT is worth 15 ECTS. It should be
    created within 4 months and at most 450 hours.

    This aim of this bachelor's thesis is to answer the following
    questions:
    \begin{itemize}
        \item \textbf{Representation of Math:} How can math be expressed
              for speech recognition in a textual way?
              Especially:
            \begin{itemize}
                \item What reasons are there to use \TeX{}, which
              reasons are there for MathML?
                \item Are there alternatives?
            \end{itemize}
        \item \textbf{Detection:} How can parts of speech be detected
              that contain math?
            \begin{itemize}
                \item Which keywords indicate mathematics?
                \item Is a keyword-density based approach sufficient?
            \end{itemize}
        \item \textbf{Evalution of math recognition strength}:
            \begin{itemize}
                \item How can speech recognition systems be evaluated
                      for their strength in math recognition?
                \item Is the \textbf{W}ord \textbf{E}rror \textbf{R}ate
                      to measure how well the recognition worked?
            \end{itemize}
        \item \textbf{Literature research:}
            \begin{itemize}
                \item Can \TeX{} be used as a grammar to recognize math speech?
                \item Can MathML be used as a grammar to recognize math speech?
            \end{itemize}
    \end{itemize}

    Follow-up tasks, that will not be part of this bachelor's thesis,
    include:
    \begin{itemize}
        \item \textbf{Other languages}: This thesis will focus on math
            recognition for the English language. Follow-up work might
            try to deal with math independant of the language.
        \item \textbf{Implementation}: The aim of this thesis is not
            to create a working math recognition.
    \end{itemize}

\section{Significance}
This thesis will create a basis for follow-up work in speech recognition
that contains mathematical content. It will enable people to evaluate
various speech2math recognition ideas. Also, it will give an overview
of the current state of art in math speech recognition and which
questions need to be tackled in feature.

\section{Time schedule}
\begin{itemize}
    \item[10h] Research of ways to represent math
    \item[20h] Research ways how \TeX{} deals with math
    \item[20h] Research how MathML deals with math
    \item[50h] Recording math lectures
    \item[100h] Annotating math lectures; writing the best
                representation for mathematical terms contained in
                these lectures
    \item[10h] Finding keywords that indicate mathematical formulas
    \item[5h] Test the keyword-approach with the annotated lectures
\end{itemize}

\renewcommand\refname{Related Literature}
\nocite{*}
\bibliographystyle{itmalpha}
\bibliography{literatur}

\section{Hypotheses}
I think that MathML will be the best way to represent math, because
it was designed to do this. MathML~3.0, the most recent version,
is a W3C recommendation since October 2001.

\TeX{} in contrast is great in rendering mathematical equations,
but it grew over time. It existed even before the web was invented.

Another reason why I think MathML might be favorable for internal
representation is that it was created to be parsed and written by
machines. It is an XML standard and as such you can apply XML tools
and libraries to parse it. \TeX{} on the other hand was created
to be written by humans.

I'm pretty sure that it is hopless to create a grammar for math
in it's general form. But for some areas like boolean logic, arithmetic
or analysis it might work pretty well.

\end{document}
