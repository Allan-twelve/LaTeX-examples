%!TEX root = GeoTopo.tex
%%%%%%%%%%%%%%%%%%%%%%%%%%%%%%%%%%%%%%%%%%%%%%%%%%%%%%%%%%%%%%%%%%%%%
% Mitschrieb vom 30.01.2014                                         %
%%%%%%%%%%%%%%%%%%%%%%%%%%%%%%%%%%%%%%%%%%%%%%%%%%%%%%%%%%%%%%%%%%%%%
\chapter{Krümmung}

\begin{definition}\xindex{Kurve}%
  Sei $f: [a, b] \rightarrow \mdr^n$ eine eine Funktion aus $C^\infty$.
  Dann heißt $f$ \textbf{Kurve}.
\end{definition}

\section{Krümmung von Kurven}\label{sec:Kurvenkrümmung}
\begin{definition}%In Vorlesung: Def.+Bem. 16.1
    Sei $\gamma: I = [a, b] \rightarrow \mdr^n$ eine Kurve.

    \begin{defenum}
        \item Die Kurve $\gamma$ heißt
              \textbf{durch Bogenlänge parametrisiert}\xindex{parametrisiert!durch Bogenlänge},
              wenn gilt:
              \[\|\gamma'(t)\|_2 = 1 \;\;\; \forall t \in I\]
              Dabei ist $\gamma'(t) = \left (\gamma_1'(t), \gamma_2'(t), \dots, \gamma_n'(t) \right)$.
        \item $l(\gamma) = \int_a^b \|\gamma'(t)\| \mathrm{d} t$ heißt
              \textbf{Länge von $\gamma$}\xindex{Kurve!Länge einer}.
    \end{defenum}
\end{definition}

\begin{bemerkung}[Eigenschaften von Kurven I]%In Vorlesung: Def.+Bem. 16.1
    Sei $\gamma: I = [a, b] \rightarrow \mdr^n$ eine $C^\infty$-Funktion.

    \begin{bemenum}
        \item Ist $\gamma$ durch Bogenlänge parametrisiert, so ist $l(\gamma) = b-a$.
        \item \label{bem:16.1d} Ist $\gamma$ durch Bogenlänge parametrisiert, so ist
              $\gamma'(t)$ orthogonal zu $\gamma''(t)$ für alle $t \in I$.
    \end{bemenum}
\end{bemerkung}

\begin{beweis}\leavevmode
    \begin{enumerate}[label=\alph*)]
        \item $l(\gamma) = \int_a^b \|\gamma'(t)\| \mathrm{d} t = \int_a^b 1 \mathrm{d} t = b - a$.
        \item Im Folgenden wird die Aussage nur für $\gamma: [a, b] \rightarrow \mdr^2$ bewiesen.
              Allerdings funktioniert der Beweis im $\mdr^n$ analog. Es muss nur
              die Ableitung angepasst werden.
              \begin{align*}
                            1 &= \|\gamma'(t)\| = \|\gamma'(t)\|^2 = \langle \gamma'(t), \gamma'(t) \rangle\\
                \Rightarrow 0 &= \frac{\mathrm{d}}{\mathrm{d}t} \langle \gamma'(t), \gamma'(t) \rangle\\
                              &= \frac{\mathrm{d}}{\mathrm{d}t} (\gamma_1'(t)\gamma_1'(t) + \gamma_2'(t)\gamma_2'(t))\\
                              &= 2 \cdot (\gamma_1''(t) \cdot \gamma_1'(t) + \gamma_2''(t) \cdot \gamma_2'(t))\\
                              &= 2 \cdot \langle \gamma''(t), \gamma'(t) \rangle
              \end{align*}
    \end{enumerate}
\end{beweis}

\begin{definition}%In Vorlesung: Definition 16.2
    Sei $\gamma: I \rightarrow \mdr^2$ eine durch Bogenlänge
    parametrisierte Kurve.

    \begin{defenum}
        \item Für $t \in I$ sei $n(t)$ \textbf{Normalenvektor}\xindex{Normalenvektor}
              an $\gamma$ in $t$ wenn gilt:
              \[\langle n(t), \gamma'(t) \rangle = 0 \text{, } \|n(t)\|=1 \text{ und } \det((\gamma'(t), n(t))) = +1\]
        \item Seit $\kappa: I \rightarrow \mdr$ so, dass gilt:
              \[\gamma''(t) = \kappa(t) \cdot n(t)\]
              Dann heißt $\kappa(t)$ \textbf{Krümmung}\xindex{Krümmung}
              von $\gamma$ in $t$.
    \end{defenum}
\end{definition}

Da $n(t)$ und $\gamma''(t)$ nach \cref{bem:16.1d} linear
              abhängig sind, existiert $\kappa(t)$.

\begin{beispiel}%In Vorlesung: Beispiel 16.3
    Gegeben sei ein Kreis mit Radius $r$, d.~h. mit Umfang $2\pi r$.
    Es gilt:

    \[\gamma(t) = \left (r \cdot \cos \frac{t}{r}, r \cdot \sin \frac{t}{r} \right ) \text{ für } t \in [0, 2\pi r]\]
    ist parametrisiert durch Bogenlänge, da gilt:

    \begin{align*}
        \gamma'(t)  &= \left ((r \cdot \frac{1}{r}) (- \sin \frac{t}{r}), r \frac{1}{r} \cos \frac{t}{r} \right )\\
                    &= \left (- \sin \frac{t}{r}, \cos \frac{t}{r} \right )
    \end{align*}

    Der Normalenvektor von $\gamma$ in $t$ ist
    \[n(t) = \left (- \cos \frac{t}{r}, - \sin \frac{t}{r} \right )\]
    da gilt:

    \begin{align*}
        \langle n(t), \gamma'(t) \rangle &=
        \left \langle
            \begin{pmatrix}- \cos \frac{t}{r}\\ - \sin \frac{t}{r}\end{pmatrix},
            \begin{pmatrix}- \sin \frac{t}{r}\\ \cos \frac{t}{r}\end{pmatrix}
        \right \rangle\\
        &= (- \cos \frac{t}{r}) \cdot (- \sin \frac{t}{r}) + (- \sin \frac{t}{r}) \cdot (\cos \frac{t}{r})\\
        &= 0\\
        \|n(t)\| &= \left \| (- \cos \frac{t}{r}, - \sin \frac{t}{r}) \right \|\\
        &=(- \cos \frac{t}{r})^2 + (- \sin \frac{t}{r})^2\\
        &= 1\\
        \det(\gamma_1'(t), n(t)) &= \left \|
            \begin{pmatrix}
                - \sin \frac{t}{r} & - \cos \frac{t}{r}\\
                  \cos \frac{t}{r} & - \sin \frac{t}{r}
            \end{pmatrix}
        \right \|\\
        &= (- \sin \frac{t}{r})^2 - (- \cos \frac{t}{r}) \cdot \cos \frac{t}{r}\\
        &= 1
    \end{align*}

    Die Krümmung ist für jedes $t$ konstant $\frac{1}{r}$, da gilt:
    \begin{align*}
        \gamma''(t) &= \left (- \frac{1}{r} \cos \frac{t}{r}, - \frac{1}{r} \sin \frac{t}{r} \right )\\
                    &= \frac{1}{r} \cdot \left (- \cos \frac{t}{r}, - \sin \frac{t}{r} \right )\\
        \Rightarrow \kappa(t) &= \frac{1}{r}
    \end{align*}
\end{beispiel}

\begin{definition}%In Vorlesung: Def.+Bem. 16.4
    Sei $\gamma: I \rightarrow \mdr^3$ eine durch Bogenlänge parametrisierte
    Kurve.

    \begin{defenum}
        \item Für $t \in I$ heißt $\kappa(t) := \|\gamma''(t)\|$ die
              \textbf{Krümmung}\xindex{Krümmung} von $\gamma$ in $t$.
        \item Ist für $t \in I$ die Ableitung $\gamma''(t) \neq 0$,
              so heißt $\frac{\gamma''(t)}{\|\gamma''(t)\|}$ \textbf{Normalenvektor}\xindex{Normalenvektor}
              an $\gamma$ in $t$.
        \item \label{def:16.4c} $b(t)$ sei ein Vektor, der $\gamma'(t), n(t)$
              zu einer orientierten Orthonormalbasis von $\mdr^3$ ergänzt.
              Also gilt:
              \[\det(\gamma'(t), n(t), b(t)) = 1\]
              $b(t)$ heißt \textbf{Binormalenvektor}\xindex{Binormalenvektor},
              die Orthonormalbasis
              \[\Set{\gamma'(t), n(t), b(t)}\]
              heißt \textbf{begleitendes Dreibein}\xindex{Dreibein!begleitendes}.
    \end{defenum}
\end{definition}

\begin{bemerkung}[Eigenschaften von Kurven II]%In Vorlesung: Def.+Bem 16.4
    Sei $\gamma: I \rightarrow \mdr^3$ durch Bogenlänge parametrisierte
    Kurve.

    \begin{bemenum}
        \item $n(t)$ ist orthogonal zu $\gamma'(t)$.
        \item $b(t)$ aus \cref{def:16.4c} ist eindeutig.
    \end{bemenum}
\end{bemerkung}

\section{Tangentialebene}\index{Tangentialebene|(}
Erinnerung Sie sich an \cref{def:8.5} \enquote{reguläre Fläche}.

Äquivalent dazu ist: $S$ ist lokal von der Form
\[V(f) = \Set{x \in \mdr^3 | f(x) = 0 }\]
für eine $C^\infty$-Funktion $f: \mdr^3 \rightarrow \mdr$.

\begin{definition}\label{def:Tangentialebene}%In Vorlesung: 17.1
    Sei $S \subseteq \mdr^3$ eine reguläre Fläche, $s \in S$,
    $F: U \rightarrow V \cap S$ eine lokale Parametrisierung um $s \in V$:
    \[(u,v) \mapsto (x(u,v), y(u,v), z(u,v))\]
    Für $p=F^{-1}(s) \in U$ sei
    \[        J_F(p) = \begin{pmatrix}
            \frac{\partial x}{\partial u} (p) & \frac{\partial x}{\partial v} (p)\\
            \frac{\partial y}{\partial u} (p) & \frac{\partial y}{\partial v} (p)\\
            \frac{\partial z}{\partial u} (p) & \frac{\partial z}{\partial v} (p)
        \end{pmatrix}\]
    und $D_p F: \mdr^2 \rightarrow \mdr^3$ die durch $J_F (p)$
    definierte lineare Abbildung.

    Dann heißt $T_s S := \Bild(D_p F)$ die \textbf{Tangentialebene}\xindex{Tangentialebene}
    an $s \in S$.
\end{definition}

\begin{bemerkung}[Eigenschaften der Tangentialebene]%
    \begin{bemenum}
        \item $T_s S$ ist $2$-dimensionaler Untervektorraum von $\mdr^3$.%In Vorlesung: 17.2
        \item $T_s S = \langle \tilde{u}, \tilde{v} \rangle$, wobei $\tilde{u}, \tilde{v}$
              die Spaltenvektoren der Jacobi-Matrix $J_F(p)$ sind.
        \item $T_s S$ hängt nicht von der gewählten Parametrisierung ab.%In Vorlesung: 17.3
        \item Sei $S=V(f)$ eine reguläre Fläche in $\mdr^3$, also %In Vorlesung: Bemerkung 17.4
                $f:V \rightarrow \mdr$ eine $C^\infty$-Funktion, $V \subseteq \mdr^3$
                offen, $\grad(f)(x) \neq 0$ für alle $x \in S$.

                Dann ist $T_s S = (\grad(f)(s))^\perp$ für jedes $s \in S$.
    \end{bemenum}
\end{bemerkung}

\begin{beweis}\leavevmode
    \begin{enumerate}[label=\alph*)]
        \item \label{bew:tangentialebene.a} $J_F$ ist eine $3 \times 2$-Matrix, die mit einem $2 \times 1$-Vektor
              multipliziert wird. Das ist eine lineare Abbildung und aus der
              linearen Algebra ist bekannt, das das Bild ein Vektorraum ist.
              Da $\rang(J_F) = 2$, ist auch $\dim (T_s S) = 2$.
        \item Hier kann man wie in \cref{bew:tangentialebene.a} argumentieren
        \item $T_s S = \{x \in \mdr^3 | \exists \text{parametrisierte Kurve }
          \gamma:[- \varepsilon, + \varepsilon] \rightarrow S
          \text{ für ein } \varepsilon > 0
          \text{ mit } \gamma(0) = s \text{ und } \gamma'(0) = x
          \}$\\
          Wenn jemand diesen Beweis führt, bitte an info@martin-thoma.de
          schicken.%TODO
        \item Sei $x \in T_s S, \gamma:[-\varepsilon, +\varepsilon] \rightarrow S$
    eine parametrisierte Kurve mit $\varepsilon > 0$ und $\gamma'(0) = s$,
    sodass $\gamma'(0) = x$ gilt. Da $\gamma(t) \in S$ für alle
    $t \in [-\varepsilon, \varepsilon]$, ist $f \circ \gamma = 0$\\
    $\Rightarrow 0 = (f \circ \gamma)'(0) = \langle \grad(f)(\gamma(0)), \gamma'(0) \rangle$\\
    $\Rightarrow T_s S \subseteq \grad (f)(s)^\perp$\\
    $\xRightarrow{\dim = 2} T_s S = (\grad(f)(s))^\perp$
    \end{enumerate}
\end{beweis}

%%%%%%%%%%%%%%%%%%%%%%%%%%%%%%%%%%%%%%%%%%%%%%%%%%%%%%%%%%%%%%%%%%%%%
% Mitschrieb vom 04.02.2014                                         %
%%%%%%%%%%%%%%%%%%%%%%%%%%%%%%%%%%%%%%%%%%%%%%%%%%%%%%%%%%%%%%%%%%%%%
\begin{definition}%In Vorlesung: Def.+Bem 17.5
    \begin{defenum}
        \item Ein \textbf{Normalenfeld}\xindex{Normalenfeld} auf der regulären
              Fläche $S \subseteq \mdr^3$ ist eine Abbildung $n: S \rightarrow S^2 \subseteq \mdr^3$
              mit $n(s) \in T_s S^\perp$ für jedes $s \in S$.
        \item $S$ heißt \textbf{orientierbar}\xindex{Fläche!orientierbare},
              wenn es ein stetiges Normalenfeld auf $S$ gibt.
    \end{defenum}
\end{definition}

Manchmal wird zwischen einem \textit{Normalenfeld} und einem
\textit{Einheitsnormalenfeld}\xindex{Einheitsnormalenfeld} unterschieden.
Im Folgenden werden diese Begriffe jedoch synonym benutzt.

\begin{bemerkung}[Eigenschaften von Normalenfeldern]%In Vorlesung: Def.+Bem 17.5
    \begin{bemenum}
        \item Ein Normalenfeld auf $S$ ist genau dann stetig, wenn es
              glatt ist (also $C^\infty$).
        \item Zu jedem $s \in S$ gibt es eine Umgebung $V \subseteq \mdr^3$
              von $s$ und eine lokale Parametrisierung $F: U \rightarrow V$
              von $S$ um $s$, sodass auf $F(U) = V \cap S$
              ein stetiges Normalenfeld existiert.
        \item $S$ ist genau dann orientierbar, wenn es einen
              differenzierbaren Atlas von $S$ aus lokalen Parametrisierungen
              $F_i: U_i \rightarrow V_i,\;i \in I$ gibt, sodass
              für alle $i, j \in F$ und alle $s \in V_i \cap V_j \cap S$
              gilt:
              \[\det(\underbrace{D_s \overbrace{F_j \circ F_i^{-1}}^{V_i \rightarrow V_j}}_{\in \mdr^{3 \times 3}}) > 0\]
    \end{bemenum}
\end{bemerkung}

\begin{beweis}
    Wird hier nicht geführt.%TODO: Übung? Übungsblatt?
\end{beweis}

\begin{beispiel}[Normalenfelder]
    \begin{bspenum}
        \item $S = S^2$, $n_1 = \id_{S^2}$ ist ein stetiges Normalenfeld.\\
              Auch $n_2 = - \id_{S^2}$ ist ein stetiges Normalenfeld.
        \item $S = \text{Möbiusband}$ (vgl. \cref{fig:moebius-strip})
              ist nicht orientierbar. Es existiert ein Normalenfeld,
              aber kein stetiges Normalenfeld.
    \end{bspenum}
\end{beispiel}

\begin{figure}[htp]\xindex{Möbiusband}
    \centering
    \includegraphics[width=0.5\linewidth, keepaspectratio]{figures/moebius-strip.pdf}
    \caption{Möbiusband}
    \label{fig:moebius-strip}
\end{figure}
\index{Tangentialebene|)}
\section{Gauß-Krümmung}\index{Gauß-Krümmung|(}
\begin{bemerkung}\label{bem:18.1}%In Vorlesung: Bemerkung 18.1
    Sei $S$ eine reguläre Fläche, $s \in S$, $n(s)$ ist ein Normalenvektor
    in $s$, $x \in T_s S$, $\|x\| = 1$.

    Sei $E$ der von $x$ und $n(s)$ aufgespannte 2-dimensionale
    Untervektorraum von $\mdr^3$.

    Dann gibt es eine Umgebung $V \subseteq \mdr^3$ von $s$, sodass
    \[C := (s + E) \cap S \cap V\]
    das Bild einer durch Bogenlänge parametrisierten Kurve
    $\gamma:[-\varepsilon, \varepsilon] \rightarrow S$ enthält mit
    $\gamma(0) = s$ und $\gamma'(0) = x$.
\end{bemerkung}

\begin{beweis}
    \enquote{Satz über implizite Funktionen}\footnote{Siehe z.~B.
    \url{https://github.com/MartinThoma/LaTeX-examples/tree/master/documents/Analysis\%20II}}
\end{beweis}

\begin{definition}\xindex{Normalkrümmung}%In Vorlesung: Definition 18.2
    In der Situation aus \cref{bem:18.1} heißt die Krümmung $\kappa_\gamma(0)$
    der Kurve $\gamma$ in der Ebene $(s+ E)$ im Punkt $s$ die
    \textbf{Normalkrümmung} von $S$ in $s$ in Richtung
    $x = \gamma'(0)$.

    Man schreibt: $\kappanor(s, x) := \kappa_\gamma(0)$
\end{definition}

\underline{Hinweis}: Die Krümmung ist nur bis auf das Vorzeichen bestimmt.

\begin{beispiel}[Gauß-Krümmung]%In Vorlesung: Beispiel 18.3
    \begin{bspenum}
        \item $S = S^2 = V(X^2 + Y^2 + Z^2 - 1)$ ist die Kugel um den Ursprung mit Radius~1,
              $n = \id$, $s=(0,0,1)$, $x=(1,0,0)$\\
              $\Rightarrow E = \mdr \cdot x + \mdr \cdot n(s)$ ($x,z\text{-Ebene}$)

              $C = E \cap S$ ist Kreislinie\\
              $\kappanor(s, x) = \frac{1}{r} = 1$
        \item $S = V(X^2 + Z^2 - 1) \subseteq \mdr^3$ ist ein Zylinder (siehe \cref{fig:regular-zylinder}).
              $s = (1,0,0)$\\
              $x_1 = (0,1,0) \Rightarrow E_1 = \mdr \cdot e_1 + \mdr \cdot e_2$ ($x,y\text{-Ebene}$)\\
              $S \cap E_1 = V(X^2 + Y^2 - 1) \cap E$, Kreislinie in $E$\\
              $\Rightarrow \kappanor(s, x_1) = \pm 1$\\
              $x_2 = (0, 0, 1), E_2 = \mdr \cdot e_1 + \mdr \cdot e_3$ ($x,z\text{-Ebene}$)\\
              $V \cap E_2 \cap S = \Set{(1, 0, z) \in \mdr^3 | z \in \mdr}$ ist eine Gerade\\
              $\Rightarrow \kappanor(s, x_2) = 0$
        \item $S = V(X^2 - Y^2 - Z)$, $s = (0,0,0)$ (Hyperbolisches Paraboloid\xindex{Paraboloid!hyperbolisches}, siehe \cref{fig:hyperbolic-paraboloid})\\
              $x_1 = (1,0,0)$, $n(s) = (0,0,1)$\\
              $x_2 = (0, 1, 0)$\\
              $\kappanor(s, x_1) = \hphantom{-}2$\\
              $\kappanor(s, x_2) = -2$
    \end{bspenum}
\end{beispiel}

\begin{figure}[ht]
    \centering
    \subfloat[$S = V(X^2 + Z^2 - 1)$]{
        \resizebox{0.4\linewidth}{!}{\input{figures/cylinder.tex}}
        \label{fig:regular-zylinder}
    }%
    \subfloat[$S = V(X^2 - Y^2 - Z)$]{
        \resizebox{0.4\linewidth}{!}{\input{figures/hyperbolic-paraboloid.tex}}
        \label{fig:hyperbolic-paraboloid}
    }%
    \label{fig:regular-surfaces}
    \caption{Beispiele für reguläre Flächen}
\end{figure}

%%%%%%%%%%%%%%%%%%%%%%%%%%%%%%%%%%%%%%%%%%%%%%%%%%%%%%%%%%%%%%%%%%%%%
% Mitschrieb vom 06.02.2014                                         %
%%%%%%%%%%%%%%%%%%%%%%%%%%%%%%%%%%%%%%%%%%%%%%%%%%%%%%%%%%%%%%%%%%%%%
\begin{definition}\label{def:18.4}\xindex{Normalkrümmung}%In Vorlesung: Def. 18.4
    Sei $S \subseteq \mdr^3$ eine reguläre Fläche, $s \in S$ und $n$ ein
    stetiges Normalenfeld auf $S$.

    $\gamma:[-\varepsilon, \varepsilon] \rightarrow S$ eine nach
    Bogenlänge parametrisierte Kurve ($\varepsilon > 0$) mit
    $\gamma(0) = s$ und $\gamma''(0) \neq 0$.

    Sei $n(0) := \frac{\gamma''(0)}{\|\gamma''(0)\|}$. Zerlege
    \[n(0) = n(0)^t + n(0)^\perp \text{ mit } n(0)^t \in T_s S \text{ und } n(0)^\perp \in (T_s S)^\perp\]

    Dann ist $n(0)^\perp = \langle n(0), n(s) \rangle \cdot n(s)$\\
    $\kappanor(s, \gamma) := \langle \gamma''(0), n(s) \rangle$
    die \textbf{Normalkrümmung}.
\end{definition}

\begin{bemerkung}
    Sei $\overline{\gamma}(t) = \gamma(-t)$, $t \in [- \varepsilon, \varepsilon]$.
    Dann ist $\kappanor(s, \overline{\gamma}) = \kappanor(s, \gamma)$.
\end{bemerkung}

\begin{beweis}
    $\overline{\gamma}''(0) = \gamma''(0)$, da $\overline{\gamma}'(0) = - \gamma'(0)$.

    Es gilt: $\kappanor(s,\gamma)$ hängt nur von $|\gamma'(0)|$ ab
    und ist gleich $\kappanor(s, \gamma'(0))$.
\end{beweis}

\begin{bemerkung}%In Vorlesung: Bem.+Def. 18.6
    Sei $S$ eine reguläre Fläche und $n=n(s)$ ein Normalenvektor an
    $S$ in $s$.

    Sei $T_{s}^{1} S = \Set{x \in T_s S | \|x\| = 1} \cong S^1$.
    Dann ist
    \[ \kappanor^n(s): T^1_s S \rightarrow \mdr, \;\;\; x \mapsto \kappanor(s,x)\]
    eine glatte Funktion und
    $\Bild \kappanor^n(s)$ ist ein abgeschlossenes Intervall.
\end{bemerkung}

\begin{definition}\xindex{Hauptkrümmung}\xindex{Gauß-Krümmung}%In Vorlesung: Bem.+Def. 18.6
    Sei $S$ eine reguläre Fläche und $n=n(s)$ ein Normalenvektor an
    $S$ in $s$.

    \begin{defenum}
        \item $\begin{aligned}[t]
                \kappa^n_1(s) :&= \min \Set{\kappanor^n(s,x) | x \in T_s^1 S} \text{ und }\\
                \kappa^n_2(s) :&= \max \Set{\kappanor^n(s,x) | x \in T_s^1 S}
              \end{aligned}$
              heißen \textbf{Hauptkrümmungen} von $S$ in $s$.
        \item $K(s) := \kappa_1^n(s) \cdot \kappa_2^n(s)$ heißt
              \textbf{Gauß-Krümmung} von $S$ in $s$.
    \end{defenum}
\end{definition}

\begin{bemerkung}%In Vorlesung: Bem.+Def. 18.6
    Ersetzt man $n$ durch $-n$, so gilt:

    \begin{align*}
                \kappanor^{-n}(s, x) &= - \kappanor^n(x)\; \forall x \in T_s^1 S\\
        \Rightarrow \kappa_1^{-n}(s) &= - \kappa_2^n(s)\\
                    \kappa_2^{-n}(s) &= - \kappa_1^n (s)\\
              \text{ und } K^{-n}(s) &= K^n(s) =: K(s)
    \end{align*}
\end{bemerkung}

\begin{beispiel}
    \begin{bspenum}
        \item $S = S^2$. Dann ist $\kappa_1(s) = \kappa_2(s) = \pm 1\;\forall s \in S^2$\\
              $\Rightarrow K(s) = 1$
        \item Zylinder:\\
              $\kappa_1(s) = 0, \kappa_2(s) = 1 \Rightarrow K(s) = 0$
        \item Sattelpunkt auf hyperbolischem Paraboloid:\\
              $\kappa_1(s) < 0, \kappa_2(s) = 0 \rightarrow K(s) < 0$
        \item $S = \text{Torus}$. Siehe \cref{fig:torus-gauss-kruemmung}\\
            \begin{figure}[htp]\xindex{Torus}
                \centering
                \includegraphics[width=0.95\linewidth, keepaspectratio]{figures/torus-gauss-kruemmung.pdf}
                \caption{$K(s_1) > 0$, $K(s_2) = 0$, $K(s_3) < 0$}
                \label{fig:torus-gauss-kruemmung}
            \end{figure}
    \end{bspenum}
\end{beispiel}

\begin{bemerkung}%In Vorlesung: Bem. 18.7
    Sei $S$ eine reguläre Fläche, $s \in S$ ein Punkt.
    \begin{bemenum}
        \item Ist $K(s) > 0$, so liegt $S$ in einer Umgebung von $s$
              ganz auf einer Seite von $T_s S + s$.
        \item Ist $K(s) < 0$, so schneidet jede Umgebung von $s$ in $S$
              beide Seiten von $T_s S + s$.
    \end{bemenum}
\end{bemerkung}
\index{Gauß-Krümmung|)}
%%%%%%%%%%%%%%%%%%%%%%%%%%%%%%%%%%%%%%%%%%%%%%%%%%%%%%%%%%%%%%%%%%%%%
% Mitschrieb vom 11.02.2014                                         %
%%%%%%%%%%%%%%%%%%%%%%%%%%%%%%%%%%%%%%%%%%%%%%%%%%%%%%%%%%%%%%%%%%%%%
\section{Erste und zweite Fundamentalform}%In Vorlesung: §19
Sei $S \subseteq \mdr^3$ eine reguläre Fläche, $s \in S$, $T_s S$ die Tangentialebene
an $S$ in $s$ und $F: U \rightarrow V$ eine lokale Parametrisierung von $S$ um
$s$. Weiter sei $p := F^{-1}(s)$.

\begin{definition}\xindex{Fundamentalform!erste}%In Vorlesung: Bem.+Def. 19.1
  Sei $I_S \in \mdr^{2 \times 2}$ definiert als
      \begin{align*}
        I_S :&= \begin{pmatrix}
                  g_{1,1}(s) & g_{1,2}(s)\\
                  g_{1,2}(s) & g_{2,2}(s)
               \end{pmatrix} =
               \begin{pmatrix}
                  E(s) & F(s) \\
                  F(s) & G(s)
               \end{pmatrix}\\
\text{mit } g_{i,j} &= g_s(D_p F(e_i), D_p F(e_j))\\
              &= \langle \frac{\partial F}{\partial u_i} (p), \frac{\partial F}{\partial u_j} (p) \rangle \;\;\; i,j \in \Set{1,2}
      \end{align*}
      Die Matrix $I_S$ heißt \textbf{erste Fundamentalform}
      von $S$ bzgl. der Parametrisierung $F$.
\end{definition}

\begin{bemerkung}%In Vorlesung: Bem.+Def. 19.1
    \begin{bemenum}
        \item \label{bem:19.1a} Die Einschränkung des Standardskalarproduktes des $\mdr^3$ auf
              $T_s S$ macht $T_s S$ zu einem euklidischen Vektorraum.
        \item $\Set{D_p F(e_1), D_p F(e_2)}$ ist eine Basis von $T_s S$.
        \item Bzgl. der Basis $\Set{D_p F(e_1), D_p F(e_2)}$ hat das
              Standardskalarprodukt aus \cref{bem:19.1a} die Darstellungsmatrix
              $I_S$.
        \item $g_{i,j}(s)$ ist eine differenzierbare Funktion von $s$.
    \end{bemenum}
\end{bemerkung}

\begin{bemerkung}
    \[\det(I_S) = \left \| \frac{\partial F}{\partial u_1}(p) \times \frac{\partial F}{\partial u_2}(p) \right \|^2\]
\end{bemerkung}

\begin{beweis}\leavevmode
    Sei $\frac{\partial F}{\partial u_1}(p) = \begin{pmatrix}
        x_1\\ x_2 \\ x_3
    \end{pmatrix}, \;\;\; \frac{\partial F}{\partial u_2}(p) = \begin{pmatrix}
        y_1\\ y_2 \\ y3
    \end{pmatrix}$

    Dann ist $\frac{\partial F}{\partial u_1}(p) \times \frac{\partial F}{\partial u_2}(p) = \begin{pmatrix}
        z_1 \\ z_2 \\ z_3
    \end{pmatrix}$ mit
    \begin{align*}
        z_1 &= x_2 y_3 - x_3 y_2\\
        z_2 &= x_3 y_1 - x_1 y_3\\
        z_3 &= x_1 y_2 - x_2 y_1\\
    \Rightarrow \|\frac{\partial F}{\partial u_1} (p) \times \frac{\partial F}{\partial u_2} (p)\| &= z_1^2 + z_2^2 + z_3^2\\
    \end{align*}
    \begin{align*}
        \det(I_S) &= g_{1,1} g_{2,2} - g_{1,2}^2\\
        &= \left \langle \begin{pmatrix} x_1 \\ x_2 \\ x_3 \end{pmatrix}, \begin{pmatrix} x_1 \\ x_2 \\ x_3 \end{pmatrix} \right \rangle \left \langle \begin{pmatrix} y_1 \\ y_2 \\ y_3 \end{pmatrix}, \begin{pmatrix} y_1 \\ y_2 \\ y_3 \end{pmatrix} \right \rangle - \left \langle \begin{pmatrix} x_1 \\ x_2 \\ x_3 \end{pmatrix}, \begin{pmatrix} y_1 \\ y_2 \\ y_3 \end{pmatrix} \right \rangle^2\\
        &= (x_1^2 + x_2^2 + x_3^2) (y_1^2 + y_2^2 + y_3^2) - (x_1 y_1 + x_2 y_2 + x_3 y_3)^2
    \end{align*}
\end{beweis}

\begin{definition}\xindex{Flächenelement}%In Vorlesung: Def.+Bem. 19.3 / Erinnerung
    \begin{defenum}
        \item Das Differential $\mathrm{d} A = \sqrt{\det (I)} \mathrm{d} u_1 \mathrm{d} u_2$
              heißt \textbf{Flächenelement} von $S$ bzgl. der Parametrisierung $F$.
        \item \label{def:berechenbares-integral}Für eine Funktion $f: V \rightarrow \mdr$ heißt
              \[\int_V f \mathrm{d} A := \int_U f(\underbrace{F(u_1, u_2)}_{=: s}) \sqrt{\det I(s)} \mathrm{d} u_1 \mathrm{d} u_2\]
              der \textbf{Wert des Integrals} von $f$ über $V$, falls das Integral rechts
              existiert.
    \end{defenum}

\end{definition}

\begin{bemerkung}
    \begin{bemenum}
        \item $\int_V f \mathrm{d} A$ ist unabhängig von der gewählten Parametrisierung.
        \item Sei $f: S \rightarrow \mdr$ eine Funktion, die im Sinne von
              \cref{def:berechenbares-integral} lokal integrierbar ist.

              Dann ist $\int_S f \mathrm{d} A$ wohldefiniert, falls (z.~B.) $S$
              kompakt ist.

              Etwa:
              \begin{align*}
                \int_S f \mathrm{d} A &= \sum_{i=1}^n \int_{\mathrlap{V_i}} f \mathrm{d} A \\
                &- \sum_{i \neq j} \int_{\mathrlap{V_i \cap V_j}} f \mathrm{d} A \\
                &+ \sum_{i,j,k} \int_{\mathrlap{V_i \cap V_j \cap V_k}} f \mathrm{d} A\\
                &- \dots
              \end{align*}
    \end{bemenum}
\end{bemerkung}

\begin{beweis}\leavevmode
    \begin{enumerate}[label=\alph*)]
        \item Mit Transformationsformel.%TODO
        \item Ist dem Leser überlassen.%TODO
    \end{enumerate}
\end{beweis}

\begin{proposition}\xindex{Weingarten-Abbildung}\label{prop:5.1}%
    Sei $S \subseteq \mdr^3$ eine reguläre, orientierbare Fläche mit glatten
    Normalenfeld $n: S \rightarrow S^2$. Dann gilt:

    \begin{propenum}
        \item \label{prop:5.1a} $n$ induziert für jedes $s \in S$ eine lineare Abbildung $d_s n: T_s S \rightarrow T_{n(s)} S^2$
              durch
              \[d_s n(x) = \frac{\mathrm{d}}{\mathrm{d} t} n (\underbrace{s \text{\enquote{+}} tx}_{\mathclap{\text{Soll auf Fläche $S$ bleiben}}}) \Bigr |_{t=0}\]
              Die Abbildung $d_s n$ heißt \textbf{Weingarten-Abbildung}
        \item $T_{n(s)} S^2 = T_s S$.
        \item $d_s n$ ist ein Endomorphismus von $T_s S$.
        \item $d_s n$ ist selbstadjungiert bzgl. des Skalarproduktes $I_S$.
    \end{propenum}
\end{proposition}

\underline{Hinweis:} Die Weingarten-Abbildung wird auch \textit{Formoperator}\index{Formoperator|see{Weingarten-Abbildung}} genannt.
\clearpage
\begin{beweis}\leavevmode
    \begin{enumerate}[label=\alph*)]
        \item Wenn jemand diesen Beweis führt, bitte an info@martin-thoma.de
              schicken.
        \item $T_{n(S)} S^2 = \langle n(s) \rangle^\perp = T_s S$
        \item Wegen \cref{prop:5.1a} ist $d_s n$ ein Homomorphismus.%\\
              %TODO: Warum sollte das ein Endomorphismus sein?
        \item Zu zeigen: $\forall x,y \in I_s S: \langle x, d_s n (y) \rangle = \langle d_s n(x), y \rangle$

        Aufgrund der Bilinearität des Skalarproduktes genügt es diese Eigenschaft
        für die Basisvektoren zu zeigen.

        Sei $x_i = D_p F(e_i) = \frac{\partial F}{\partial u_i} (p)\;\;\; i = 1,2$

        \underline{Beh.:}
          $\langle x_i, d_s n(x_j) \rangle = \langle \frac{\partial^2 F}{\partial u_i \partial u_j} (p), d_s n (x_i) \rangle$

        $\Rightarrow \langle \frac{\partial^2 F}{\partial u_i \partial u_j} (p), d_s n (x_i) \rangle = \langle x_j, d_s n (x_i) \rangle$

        \underline{Bew.:} $
        \begin{aligned}[t]
            0 &= \hphantom{\frac{\mathrm{d}}{\mathrm{d}t} \left (\right.} \langle \frac{\partial F}{\partial u} (p + t e_j), n(p + t e_j) \rangle\\
\Rightarrow 0 &= \frac{\mathrm{d}}{\mathrm{d}t} \left (\langle \frac{\partial F}{\partial u} (p + t e_j), n(p + t e_j) \rangle \right) \Bigr |_{t=0}\\
              &= \langle \underbrace{\frac{\mathrm{d}}{\mathrm{d}t} \frac{\partial F}{\partial u_i} (p + t e_j)}_{\frac{\partial^2 F}{\partial u_j \partial u_i} (p)} \Bigr |_{t=0}, n(s) \rangle + \langle x_i, d_s n \underbrace{D_p F (e_j)}_{x_j}\rangle
        \end{aligned}$
    \end{enumerate}
\end{beweis}

%%%%%%%%%%%%%%%%%%%%%%%%%%%%%%%%%%%%%%%%%%%%%%%%%%%%%%%%%%%%%%%%%%%%%
% Mitschrieb vom 13.02.2014                                         %
%%%%%%%%%%%%%%%%%%%%%%%%%%%%%%%%%%%%%%%%%%%%%%%%%%%%%%%%%%%%%%%%%%%%%
\begin{definition}\xindex{Fundamentalform!zweite}%In Vorlesung: Def. + Bem. 19.5 a)
    Die durch $-d_s n$ definierte symmetrische Bilinearform auf $T_s S$ heißt
    \textbf{zweite Fundamentalform} von $S$ in $s$ bzgl. $F$.

    Man schreibt: $II_s(x,y) =  \langle - d_s n(x), y \rangle = I_s (-d_s n(x), y)$
\end{definition}

\begin{bemerkung}%%In Vorlesung: Def. + Bem. 19.5 b)
    Bezüglich der Basis $\Set{x_1, x_2}$ von $T_s S$ hat $II_s$ die Darstellungsmatrix
    \[(h^{(s)}_{i,j})_{i,j=1,2} \text{ mit } h_{i,j}(s) = \langle \frac{\partial^2 F}{\partial u_i \partial u_j} (p), n(s) \rangle \]
\end{bemerkung}

\begin{proposition}\label{prop:19.6}%In Vorlesung: Proposition 19.6
    Sei $\gamma:[- \varepsilon, \varepsilon] \rightarrow S$ eine nach Bogenlänge
    parametrisierte Kurve mit $\gamma(0) = s$. Dann gilt:
    \[\kappanor(s, \gamma) = II_s(\gamma'(0), \gamma'(0))\]
\end{proposition}

\begin{beweis}
    Nach \cref{def:18.4} ist $\kappanor(s, \gamma) = \langle \gamma''(0), n(s) \rangle$.
    Nach Voraussetzung gilt
    \[n(\gamma(t)) \perp \gamma'(t) \Leftrightarrow \langle \gamma''(0), n(s) \rangle = 0\]
    Die Ableitung nach $t$ ergibt
    \begin{align*}
        0 &= \frac{\mathrm{d}}{\mathrm{d}t}(\langle n (\gamma(t)), \gamma'(t))\\
        &= \left \langle \frac{\mathrm{d}}{\mathrm{d}t} n(\gamma(t)) \Bigr |_{t=0}, \gamma'(0) \right \rangle + \langle n(s), \gamma''(0) \rangle\\
        &= \langle d_s n (\gamma'(0)), \gamma'(0) \rangle + \kappanor(s,\gamma)\\
        &= - II_s(\gamma'(0), \gamma'(0)) + \kappanor(s, \gamma)
    \end{align*}
\end{beweis}

\begin{folgerung}\xindex{Normalkrümmung}%In Vorlesung: Folgerung 19.7
    Die beiden Definitionen von Normalkrümmung in \cref{sec:Kurvenkrümmung} stimmen
    überein:
    \[\kappanor(s, \gamma) = \kappanor(s, \gamma'(0))\]
\end{folgerung}

\begin{satz}%In Vorlesung: Satz 19.8
    Sei $S \subseteq \mdr^3$ eine reguläre, orientierbare Fläche und $s \in S$.
    \begin{satzenum}
        \item Die Hauptkrümmungen $\kappa_1(s), \kappa_2(s)$ sind die Eigenwerte
              von $II_s$.
        \item Für die Gauß-Krümmung gilt: $K(s) = \det(II_s)$
    \end{satzenum}
\end{satz}

\begin{beweis}\leavevmode
    \begin{enumerate}[label=\alph*)]
        \item $II_s$ ist symmetrisch, $I_s S$ hat also eine Orthonormalbasis aus
              Eigenvektoren $y_1, y_2$ von $II_s$. Ist $x \in T_s S$, $\|x\| = 1$,
              so gibt es $\varphi \in [0,2\pi)$ mit $x = \cos \varphi \cdot y_1 + \sin \varphi \cdot y_2$.

              Seien $\lambda_1, \lambda_2$ die Eigenwerte von $II_s$, also
              $II_s(y_i, y_i) = \lambda_i$. Dann gilt:
              \begin{align*}
                  II_s (x,x) &= \cos^2 \varphi \lambda_1 + \sin^2 \varphi \lambda_2\\
                  &= (1- \sin^2 \varphi) \lambda_1 + \sin^2 \varphi \lambda_2\\
                  &= \lambda_1 + \sin^2 \varphi (\lambda_2 - \lambda_1) \geq \lambda_1\\
                  &= \cos^2 \varphi + (1 - \cos^2 \varphi) \lambda_2\\
                  &= \lambda_2 - \cos^2 \varphi (\lambda_2 - \lambda_1) \leq \lambda_2\\
                  \xRightarrow{\crefabbr{prop:19.6}} \lambda_1 &= \min \Set{\kappanor (s,x) | x \in T^1_s S}\\
                  \lambda_2 &= \max \Set{\kappanor (s,x) | x \in T^1_s S}
              \end{align*}
    \end{enumerate}
\end{beweis}

\begin{satz}[Satz von Gauß-Bonnet]\xindex{Satz von!Gauß-Bonnet}%
    Sei $S \subseteq \mdr^3$ eine kompakte orientierbare reguläre Fläche. Dann gilt:
    \[\int_S K(s) \mathrm{d}A = 2 \pi \chi(S)\]
    Dabei ist $\chi(S)$ die Euler-Charakteristik von $S$.
\end{satz}

\begin{beweis}
    Der Beweis wird hier nicht geführt. Er kann in \enquote{Elementare Differentialgeometrie}
    von Christian Bär (2. Auflage), ISBN 978-3-11-022458-0, ab Seite 281 nachgelesen werden.
\end{beweis}