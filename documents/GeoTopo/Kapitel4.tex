%%%%%%%%%%%%%%%%%%%%%%%%%%%%%%%%%%%%%%%%%%%%%%%%%%%%%%%%%%%%%%%%%%%%%
% Mitschrieb vom 09.01.2014                                         %
%%%%%%%%%%%%%%%%%%%%%%%%%%%%%%%%%%%%%%%%%%%%%%%%%%%%%%%%%%%%%%%%%%%%%
\chapter{Euklidische und nichteuklidische Geometrie}

\begin{definition}%
    Das Tripel $(X, d, G)$ heißt genau dann eine \textbf{Geometrie}\xindex{Geometrie},
    wenn $(X, d)$ ein metrischer Raum und $\emptyset \neq G \subseteq \powerset{X}$
    gilt. Dann heißt $G$ die Menge aller \textbf{Geraden}\xindex{Gerade}.
\end{definition}

\section{Axiome für die euklidische Ebene}
Axiome\xindex{Axiom} bilden die Grundbausteine jeder mathematischen Theorie. Eine
Sammlung aus Axiomen nennt man Axiomensystem\xindex{Axiomensystem}.
Da der Begriff des Axiomensystems so grundlegend ist, hat man auch
ein paar sehr grundlegende Forderungen an ihn: Axiomensysteme sollen
\textbf{widerspruchsfrei} sein, die Axiome sollen möglichst
\textbf{unabhängig} sein und \textbf{Vollständigkeit} wäre auch toll.
Mit Unabhängigkeit ist gemeint, dass kein Axiom sich aus einem anderem
herleiten lässt. Dies scheint auf den ersten Blick eine einfache
Eigenschaft zu sein. Auf den zweiten Blick muss man jedoch einsehen,
dass das Parallelenproblem, also die Frage ob das Parallelenaxiom
unabhängig von den restlichen Axiomen ist, über 2000 Jahre nicht
gelöst wurde. Ein ganz anderes Kaliber ist die Frage nach der
Vollständigkeit. Ein Axiomensystem gilt als Vollständig, wenn
jede Aussage innerhalb des Systems verifizierbar oder falsifizierbar
ist. Interessant ist hierbei der Gödelsche Unvollständigkeitssatz,
der z.~B. für die Arithmetik beweist, dass nicht alle Aussagen
formal bewiesen oder widerlegt werden können.

Kehren wir nun jedoch zurück zur Geometrie. Euklid hat in seiner
Abhandlung \enquote{Die Elemente} ein Axiomensystem für die Geometrie
aufgestellt.

\textbf{Euklids Axiome}
\begin{itemize}
    \item \textbf{Strecke} zwischen je zwei Punkten
    \item Jede Strecke bestimmt genau eine \textbf{Gerade}
    \item \textbf{Kreis} (um jeden Punkt mit jedem Radius)
    \item Je zwei rechte Winkel sind gleich (Isometrie, Bewegung)
    \item Parallelenaxiom von Euklid:\xindex{Parallelenaxiom}\\
        Wird eine Gerade so von zwei Geraden geschnitten, dass die
        Summe der Innenwinkel kleiner als zwei Rechte ist, dann schneiden sich
        diese Geraden auf der Seite dieser Winkel.\\
        \\
        Man mache sich klar, dass das nur dann nicht der Fall ist,
        wenn beide Geraden parallel sind und senkrecht auf die erste stehen.
\end{itemize}

\begin{definition}\xindex{Ebene!euklidische}%In Vorlesung: Definition 14.2
    Eine \textbf{euklidische Ebene} ist eine Geometrie $(X,d, G)$, die
    Axiome~\ref{axiom:1}~-~\ref{axiom:5} erfüllt:
    \begin{enumerate}[label=§\arabic*),ref=§\arabic*]
        \item \textbf{Inzidenzaxiome}\xindex{Inzidenzaxiome}:\label{axiom:1}
            \begin{enumerate}[label=(\roman*),ref=\theenumi{} (\roman*)]
                \item \label{axiom:1.1} Zu $P \neq Q \in X$ gibt es genau ein $g \in G$ mit
                      $\Set{P, Q} \subseteq g$.
                \item \label{axiom:1.2} $|g| \geq 2 \;\;\; \forall g \in G$
                \item \label{axiom:1.3} $X \notin G$
            \end{enumerate}
        \item \textbf{Abstandsaxiom}\xindex{Abstandsaxiom}: Zu $P, Q, R \in X$ gibt es \label{axiom:2}
              genau dann ein $g \in G$ mit $\Set{P, Q, R} \subseteq g$,
              wenn gilt:
              \begin{itemize}[]
                \item $d(P, R) = d(P, Q) + d(Q, R)$ oder
                \item $d(P, Q) = d(P, R) + d(R, Q)$ oder
                \item $d(Q, R) = d(Q, P) + d(P, R)$
              \end{itemize}
    \end{enumerate}
\end{definition}

\begin{definition}
    Sei $(X, d, G)$ eine Geometrie und seien $P, Q, R \in X$.
    \begin{defenum}
        \item $P, Q, R$ liegen \textbf{kollinear}\xindex{kollinear},
              wenn es $g \in G$ gibt mit $\Set{P, Q, R} \subseteq g$.
        \item $Q$ \textbf{liegt zwischen}\xindex{liegt zwischen} $P$
              und $R$, wenn $d(P, R) = d(P, Q) + d(Q, R)$
        \item \textbf{Strecke}\xindex{Strecke} $\overline{PR} := \Set{Q \in X | Q \text{ liegt zwischen } P \text{ und } R}$
        \item \textbf{Halbgeraden}\xindex{Halbgerade}:\\
              $\begin{aligned}[t]
                  PR^+ &:= \{Q \in X | Q \text{ liegt zwischen } P \text{ und } R \text{ oder } \\
             &\hphantom{:= \{Q \in X |\;} R \text{ liegt zwischen } P \text{ und } Q\}\\
                  PR^- &:= \Set{Q \in X | P \text{ liegt zwischen } Q \text{ und } R}
              \end{aligned}$
    \end{defenum}
\end{definition}

\begin{figure}[htp]
    \centering
    \begin{tikzpicture}
    \tikzstyle{point}=[circle,thick,draw=black,fill=black,inner sep=0pt,minimum width=4pt,minimum height=4pt]
    \node (Pleft) at (0,0) {};
    \node (P)[point,label=90:$P$] at (2,0) {};
    \node (R)[point,label=90:$R$] at (4,0) {};
    \node (Rright) at (6,0) {};
    \draw[dashed,very thick] (Pleft) -- (P);
    \draw[dotted,very thick] (P) -- (R) -- (Rright);
    \draw [thick,decoration={brace,mirror,raise=0.2cm},decorate] (Pleft) -- (P) node [pos=0.5,anchor=north,yshift=-0.25cm] {$PR^-$};
    \draw [thick,decoration={brace,mirror,raise=0.2cm},decorate] (P) -- (R) node [pos=0.5,anchor=north,yshift=-0.25cm] {$\overline{PR}$};
    \draw [thick,decoration={brace,mirror,raise=0.8cm},decorate] (P) -- (Rright) node [pos=0.5,anchor=north,yshift=-0.85cm] {$PR^+$};
\end{tikzpicture}

    \caption{Halbgeraden}
    \label{fig:halbgeraden}
\end{figure}

\begin{bemerkung}
    \begin{bemenum}
        \item $PR^+ \cup PR^- = PR$
        \item $PR^+ \cap PR^- = \Set{P}$
    \end{bemenum}
\end{bemerkung}

\begin{beweis}\leavevmode
    \begin{enumerate}[label=\alph*)]
        \item \enquote{$\subseteq$} folgt direkt aus der Definition von $PR^+$ und $PR^-$\\
              \enquote{$\supseteq$}: Sei $Q \in PR \Rightarrow P, Q, R$
              sind kollinear.\\
              $\overset{\ref{axiom:2}}{\Rightarrow}
              \begin{cases}
                Q \text{ liegt zwischen } P \text{ und } R \Rightarrow Q \in PR\\
                R \text{ liegt zwischen } P \text{ und } Q \Rightarrow Q \in PR\\
                P \text{ liegt zwischen } Q \text{ und } R \Rightarrow Q \in PR
              \end{cases}$
        \item \enquote{$\supseteq$} ist offensichtlich\\
              \enquote{$\subseteq$}: Sei $PR^+ \cap PR^-$. Dann ist
              $d(Q,R) = d(P,Q) + d(P,R)$ weil $Q \in PR^-$ und
              \begin{align*}
                &\left \{ \begin{array}{l}
                        d(P,R) = d(P,Q) + d(Q,R) \text{ oder }\\
                        d(P,Q) = d(P,R) + d(R,Q)
                       \end{array} \right \}\\
                &\Rightarrow d(Q,R) = 2d(P,Q) + d(Q,R)\\
                &\Rightarrow d(P,Q) = 0\\
                &\Rightarrow P=Q\\
                &d(P,Q) = 2d(P,R) + d(P,Q)\\
                &\Rightarrow P=R\\
                &\Rightarrow \text{Widerspruch}
              \end{align*}
    \end{enumerate}
\end{beweis}

\begin{definition}%
    \begin{enumerate}[label=§\arabic*),ref=§\arabic*,start=3]
        \item \label{axiom:3}\textbf{Anordnungsaxiome}\xindex{Anordnungsaxiome}
            \begin{enumerate}[label=(\roman*),ref=\theenumi{} (\roman*)]
                \item \label{axiom:3.1} Zu jeder
                      Halbgerade $H$ mit Anfangspunkt $P \in X$ und jedem
                      $r \in \mdr_{\geq 0}$ gibt es genau ein
                      $Q \in H$ mit $d(P,Q) = r$.
                \item \label{axiom:3.2} Jede Gerade zerlegt
                      $X \setminus g = H_1 \dcup H_2$ in zwei
                      nichtleere Teilmengen $H_1, H_2$,
                      sodass für alle $A \in H_i$, $B \in H_j$ mit
                      $i,j \in \Set{1,2}$ gilt:
                      $\overline{AB} \cap g \neq \emptyset \Leftrightarrow i \neq j$.\\
                      Diese Teilmengen $H_i$ heißen
                      \textbf{Halbebenen}\xindex{Halbebene} bzgl.
                      $g$.
            \end{enumerate}
        \item \label{axiom:4}\textbf{Bewegungsaxiom}\xindex{Bewegungsaxiom}:
            Zu $P, Q, P', Q' \in X$
            mit $d(P,Q) = d(P', Q')$ gibt es mindestens 2 Isometrien $\varphi_1, \varphi_2$
            mit $\varphi_i (P) = P'$ und $\varphi_i(Q) = Q'$ mit $i=1,2$.\footnote{Die \enquote{Verschiebung} von $P'Q'$ nach $PQ$ und die Isometrie, die zusätzlich an der Gerade durch $P$ und $Q$ spiegelt.}
        \item \label{axiom:5}\textbf{Parallelenaxiom}\xindex{Parallele}:
            Zu jeder Geraden $g \in G$ und jedem Punkt
            $P \in X \setminus g$ gibt es höchstens ein $h \in G$ mit $P \in h$ und
            $h \cap g = \emptyset$. $h$ heißt \textbf{Parallele zu $g$ durch $P$}.
    \end{enumerate}
\end{definition}

%%%%%%%%%%%%%%%%%%%%%%%%%%%%%%%%%%%%%%%%%%%%%%%%%%%%%%%%%%%%%%%%%%%%%
% Mitschrieb vom 14.01.2014                                         %
%%%%%%%%%%%%%%%%%%%%%%%%%%%%%%%%%%%%%%%%%%%%%%%%%%%%%%%%%%%%%%%%%%%%%
\begin{satz}[Satz von Pasch]\label{satz:pasch} %In Vorlesung: Bemerkung 14.5
    Seien $P$, $Q$, $R$ nicht kollinear, $g \in G$ mit $g \cap \Set{P, Q, R} = \emptyset$
    und $g \cap \overline{PQ} \neq \emptyset$.

    Dann ist entweder $g \cap \overline{PR} \neq \emptyset$ oder
                      $g \cap \overline{QR} \neq \emptyset$.
\end{satz}

Dieser Satz besagt, dass Geraden, die eine Seite eines Dreiecks
(also nicht nur eine Ecke) schneiden, auch eine weitere Seite
schneiden.

\begin{beweis}
    $g \cap \overline{PQ} \neq \emptyset$\\
    $\overset{\mathclap{\ref{axiom:3.2}}}{\Rightarrow} P$ und $Q$ liegen in verschiedenen Halbebenen bzgl. $g$\\
    $\Rightarrow$ \obda $R$ und $P$ liegen in verschieden
    Halbebenen bzgl. $g$\\
    $\Rightarrow g \cap \overline{RP} \neq \emptyset$
\end{beweis}

\begin{bemerkung}\label{kor:beh3}
    Sei $P, Q \in X$ mit $P \neq Q$ sowie $A, B \in X \setminus PQ$
    mit $A \neq B$.
    Außerdem seien $A$ und $B$ in der selben Halbebene bzgl. $PQ$ sowie
    $Q$ und $B$ in der selben Halbebene bzgl. $PA$.

    Dann gilt: $PB^+ \cap \overline{AQ} \neq \emptyset$
\end{bemerkung}

\begin{figure}[htp]
    \centering
    \input{figures/geometry-5.tex}
    \caption{Situation aus \cref{kor:beh3}}
    \label{fig:geometry-5}
\end{figure}

Auch \cref{kor:beh3} lässt sich umgangssprachlich sehr viel
einfacher ausdrücken: Die Diagonalen eines konvexen Vierecks
schneiden sich.

\begin{beweis}%In Vorlesung: Behauptung 3
    Sei $P' \in PQ^-, P' \neq P$
    $\xRightarrow{\cref{satz:pasch}} PB$ schneidet
    $\overline{AP'} \cup \overline{AQ}$

    Sei $C$ der Schnittpunkt. Dann gilt:
    \begin{enumerate}[label=(\roman*)]
        \item $C \in PB^+$, denn $A$ und $B$ liegen in derselben
              Halbebene bzgl. $PQ = P'Q$, also auch
              $\overline{AP'}$ und $\overline{AQ}$.
        \item $C$ liegt in derselben Halbebene bzgl. $PA$ wie
              $B$, weil das für $Q$ gilt.

              $\overline{AP'}$ liegt in der anderen Halbebene
              bzgl. $PA \Rightarrow C \notin \overline{P'A} \Rightarrow C \in \overline{AQ}$
    \end{enumerate}
    Da $C \in PB^+$ und $C \in \overline{AQ}$ folgt nun direkt:
    $\emptyset \neq \Set{C} \subseteq PB^+ \cap \overline{AQ} \qed$
\end{beweis}

\begin{bemerkung}\label{kor:14.6}%In Vorlesung: Bemerkung 14.6
    Seien $P, Q \in X$ mit $P \neq Q$ und $A, B \in X \setminus PQ$
    in der selben Halbebene bzgl. $PQ$. Außerdem sei $d(A,P)=d(B,P)$
    und $d(A, Q) = d(B, Q)$.

    Dann ist $A = B$.
\end{bemerkung}

\begin{figure}[htp]
    \centering
    \input{figures/geometry-2.tex}
    \caption{\cref{kor:14.6}: Die beiden roten und die beiden blauen Linien sind gleich lang. Intuitiv weiß man, dass daraus folgt, dass $A = B$ gilt.}
    \label{fig:geometriy-2}
\end{figure}

\begin{beweis} durch Widerspruch\\
    \underline{Annahme}: $A \neq B$

    Dann ist $B \notin (PA \cup QA)$ wegen \ref{axiom:2}.

    \begin{figure}[ht]
        \centering
        \subfloat[1. Fall]{
            \input{figures/geometry-3.tex}
            \label{fig:geometry-3}
        }%
        \subfloat[2. Fall]{
            \input{figures/geometry-4.tex}
            \label{fig:geometry-4}
        }%
        \label{fig:bem:14.6}
        \caption{Fallunterscheidung aus \cref{kor:14.6}}
    \end{figure}

    \underline{1. Fall}: $Q$ und $B$ liegen in derselben Halbebene bzgl. $PA$

    $\xRightarrow{\crefabbr{kor:beh3}} PB^+ \cap \overline{AQ} \neq \emptyset$.

    Sei $C$ der Schnittpunkt vom $PB$ und $AQ$.

    Dann gilt:
    \begin{enumerate}[label=(\roman*)]
        \item $d(A, C) + d(C, Q) = d(A, Q) \overset{\text{Vor.}}{=} d(B, Q) < d(B, C) + d(C, Q) \Rightarrow d(A, C) < d(B, C)$ \label{enum:komischer-beweis-i}
        \item \begin{enumerate}[label=\alph*)]
                \item $B$ liegt zwischen $P$ und $C$.

                      $d(P,A) + d(A, C) > d(P,C) = d(P,B) + d(B,C) = d(P,A) + d(B,C)$
                      $\Rightarrow d(A,C) > d(B,C) \Rightarrow$ Widerspruch zu \cref{enum:komischer-beweis-i}
                \item $C$ liegt zwischen $P$ und $B$

                      $d(P,C) + d(C,A) > d(P,A) = d(P,B) = d(P,C) + d(C, B)$\\
                      $\Rightarrow d(C, A) > d(C, B)$\\
                      $\Rightarrow$ Widerspruch zu \cref{enum:komischer-beweis-i}
            \end{enumerate}
    \end{enumerate}

    \underline{2. Fall}: $Q$ und $B$ liegen auf verschieden Halbebenen bzgl. $PA$.

    Dann liegen $A$ und $Q$ in derselben Halbebene bzgl. $PB$.

    Tausche $A$ und $B \Rightarrow$  Fall 1 $\qed$
\end{beweis}

\begin{bemerkung}\label{kor:beh2'}
    Sei $(X, d, G)$ eine Geometrie, die \ref{axiom:1}~-~\ref{axiom:3}
    erfüllt, $P, Q \in X$ mit $P \neq Q$ und $\varphi$ eine Isometrie mit
    $\varphi(P) = P$ und $\varphi(Q) = Q$.

    Dann gilt $\varphi(S) = S\;\;\;\forall S \in PQ$.
\end{bemerkung}

\begin{beweis}
    \begin{align*}
        \text{\Obda sei } S \in \overline{PQ} &\overset{\mathclap{\ref{axiom:2}}}{\Leftrightarrow} d(P,Q) = d(P,S) + d(S,Q)\\
        &\overset{\mathclap{\varphi \in \Iso(X)}}{\Rightarrow}\hspace{4 mm} d(\varphi(P),\varphi(Q)) = d(\varphi(P),\varphi(S)) + d(\varphi(S),\varphi(Q))\\
        &\overset{\mathclap{P, Q \in \Fix(\varphi)}}{\Rightarrow}\hspace{4 mm} d(P, Q) = d(P,\varphi(S)) + d(\varphi(S), Q)\\
        &\Rightarrow \varphi(S) \text{ liegt zwischen } P \text{ und } Q\\
        &\Rightarrow d(P,S) = d(\varphi(P), \varphi(S)) = d(P, \varphi(S))\\
        &\overset{\mathclap{\ref{axiom:3.1}}}{\Rightarrow} \varphi(S) = S
    \end{align*}

    $\qed$
\end{beweis}

\begin{proposition}\label{satz:14.4}%In Vorlesung: Satz 14.4
    In einer Geometrie, die \ref{axiom:1}~-~\ref{axiom:3} erfüllt,
    gibt es zu $P, P', Q, Q'$ mit $d(P, Q) = d(P', Q')$ höchstens
    zwei Isometrien mit $\varphi(P) = P'$ und $\varphi(Q) = Q'$

    Aus den Axiomen  folgt, dass es in
    der Situation von \ref{axiom:4} höchstens zwei Isometrien mit
    $\varphi_i(P) = P'$ und $\varphi_i(Q) = Q'$ gibt.
\end{proposition}

\begin{beweis}
    Seien $\varphi_1, \varphi_2, \varphi_3$ Isometrien mit
    $\varphi_i(P) = P'$, $\varphi_i(Q) = Q'$ mit $i=1,2,3$.

    Der Beweis von \cref{satz:14.4} erfolgt über zwei Teilaussagen:

    \begin{enumerate}[label=(Teil \roman*),ref=(Teil \roman*)]
        \item \label{bew:teil1} $\exists R \in X \setminus PQ$ mit $\varphi_{1} (R) = \varphi_{2} (R)$.
        \item \label{bew:teil2} Hat $\varphi$ 3 Fixpunkte, die nicht kollinear sind, so ist $\varphi = \id_X$.
    \end{enumerate}

    Aus \ref{bew:teil1} und \ref{bew:teil2} folgt, dass $\varphi_2^{-1} \circ \varphi_1 = \id_X$,
    also $\varphi_2 = \varphi_1$, da $P$, $Q$ und $R$ in diesem Fall
    Fixpunkte sind.

    Nun zu den Beweisen der Teilaussagen:
    \begin{enumerate}[label=(Teil \roman*),ref=(Teil \roman*)]
        \item Sei $R \in X \setminus PQ$. Von den drei Punkten
            $\varphi_1(R), \varphi_2(R), \varphi_3(R)$ liegen zwei
            in der selben Halbebene bzgl. $P'Q' = \varphi_i(PQ)$.

            \Obda seien $\varphi_1(R)$ und $\varphi_2(R)$ in der
            selben Halbebene.

            Es gilt: $\begin{aligned}[t]
                d(P', \varphi_1(R)) &= d(\varphi_1(P), \varphi_1(R))\\
                    &= d(P, R)\\
                    &= d(\varphi_2(P), \varphi_2(R))\\
                    &= d(P', \varphi_2(R))\\
            \end{aligned}$\\
            und analog $d(Q', \varphi_1(R)) = d(Q', \varphi_2(R))$
        \item Seien $P$, $Q$ und $R$ Fixpunkte von $\varphi$, $R \notin PQ$
        und $A \notin \overline{PQ} \cup \overline{PR} \cup \overline{QR}$.
        Sei $B \in \overline{PQ} \setminus \Set{P, Q}$. Dann ist
        $\varphi(B) = B$ wegen \cref{kor:beh2'}.

        Ist $R \in AB$, so enthält $AB$ 2 Fixpunkte von $\varphi$
        $\xRightarrow{\crefabbr{kor:beh2'}} \varphi(A) = A$.

        \begin{figure}[htp]
            \centering
            \input{figures/geometry-1.tex}
            \caption{$P, Q, R$ sind Fixpunkte, $B \in \overline{PQ} \setminus \Set{P,Q}$, $A \notin PQ \cup PR \cup QR$}
            \label{fig:geometry-1}
        \end{figure}

        Ist $R \notin AB$, so ist $AB \cap \overline{PR} \neq \emptyset$
        oder $AB \in \overline{RQ} \neq \emptyset$ nach \cref{satz:pasch}.
        Der Schnittpunkt $C$ ist dann Fixpunkt von $\varphi'$
        nach \cref{kor:beh2'} $\Rightarrow \varphi(A) = A$.
    \end{enumerate}
\end{beweis}

\begin{bemerkung}[SWS-Kongruenzsatz]\xindex{Kongruenzsatz!SWS}%
    Sei $(X, d, G)$ eine Geometrie, die \ref{axiom:1}~-~\ref{axiom:4} erfüllt.
    Seien außerdem $\triangle ABC$ und $\triangle A'B'C'$ Dreiecke, für die gilt:
    \begin{enumerate}[label=(\roman*)]
        \item \label{bem:sws.i} $d(A, B) = d(A', B')$
        \item \label{bem:sws.ii} $\angle CAB \cong \angle C'A'B'$
        \item \label{bem:sws.iii} $d(A, C) = d(A', C')$
    \end{enumerate}

    Dann ist $\triangle ABC$ kongruent zu $\triangle A'B'C'$ .
\end{bemerkung}

\begin{beweis}
    Sei $\varphi$ die Isometrie mit $\varphi(A') = A$, $\varphi(A'C'^+) = AC^+$
    und $\varphi(A'B'^+) = AB^+$. Diese Isometrie existiert wegen \cref{axiom:4}.

    $\Rightarrow C \in \varphi(A'C'^+)$ und $B \in \varphi(A'B'^+)$.

    $d(A',C')= d(\varphi(A'), \varphi(C')) = d(A, \varphi(C')) \xRightarrow{\ref{axiom:3.1}} \varphi(C') = C$

    $d(A',B')= d(\varphi(A'), \varphi(B')) = d(A, \varphi(B')) \xRightarrow{\ref{axiom:3.1}} \varphi(B') = B$

    Also gilt insbesondere $\varphi(\triangle A'B'C') = \triangle ABC$. $\qed$
\end{beweis}

\begin{bemerkung}[WSW-Kongruenzsatz]\xindex{Kongruenzsatz!WSW}%
    Sei $(X, d, G)$ eine Geometrie, die \ref{axiom:1}~-~\ref{axiom:4} erfüllt.
    Seien außerdem $\triangle ABC$ und $\triangle A'B'C'$ Dreiecke, für die gilt:
    \begin{enumerate}[label=(\roman*)]
        \item \label{bem:wsw.i} $d(A, B) = d(A', B')$
        \item \label{bem:wsw.ii} $\angle CAB \cong \angle C'A'B'$
        \item \label{bem:wsw.iii} $\angle ABC \cong \angle A'B'C'$
    \end{enumerate}

    Dann ist $\triangle ABC$ kongruent zu $\triangle A'B'C'$ .
\end{bemerkung}

\begin{beweis}
    Sei $\varphi$ die Isometrie mit $\varphi(A') = A$, $\varphi(B') = B$
    und $\varphi(C')$ liegt in der selben Halbebene bzgl. $AB$ wie $C$.
    Diese Isometrie existiert wegen \ref{axiom:4}.

    Aus $\angle CAB = \angle C'A'B' = \angle \varphi(C')\varphi(A')\varphi(B') = \angle \varphi(C')AB$ folgt, dass $\varphi(C')\in AC^+$.\\
    Analog folgt aus $\angle ABC = \angle A'B'C' = \angle \varphi(A')\varphi(B')\varphi(C') = \angle AB\varphi(C')$, dass $\varphi(C') \in BC^+$.

    Dann gilt $\varphi(C') \in AC \cap BC = \Set{C} \Rightarrow \varphi(C')=C$.

    Es gilt also $\varphi(\triangle A'B'C') = \triangle ABC$. $\qed$
\end{beweis}

%%%%%%%%%%%%%%%%%%%%%%%%%%%%%%%%%%%%%%%%%%%%%%%%%%%%%%%%%%%%%%%%%%%%%
% Mitschrieb vom 16.01.2014                                         %
%%%%%%%%%%%%%%%%%%%%%%%%%%%%%%%%%%%%%%%%%%%%%%%%%%%%%%%%%%%%%%%%%%%%%

\begin{definition}\label{def:14.8}%In Vorlesung: 14.8
    \begin{defenum}
        \item \label{def:14.8a} Ein \textbf{Winkel}\xindex{Winkel} ist ein Punkt $P \in X$
              zusammen mit $2$ Halbgeraden mit Anfangspunkt $P$.\\
              Man schreibt: $\angle R_1 P R_2$ bzw. $\angle R_2 P R_1$\footnote{Für dieses Skript gilt: $\angle R_1 P R_2 = \angle R_2 P R_1$. Also sind insbesondere alle Winkel $ \leq 180^\circ$.}
        \item Zwei Winkel sind \textbf{gleich}, wenn es eine Isometrie gibt,
              die den einen Winkel auf den anderen abbildet.
        \item \label{def:14.8c} $\angle R_1' P' R_2'$ heißt \textbf{kleiner} als
              $\angle R_1 P R_2$, wenn es eine Isometrie $\varphi$
              gibt, mit $\varphi(P') = P$, $\varphi(P'R'^{+}_{1}) = PR_{1}^{+}$
              und $\varphi(R_2')$ liegt in der gleichen Halbebene
              bzgl. $PR_1$ wie $R_2$ und in der gleichen Halbebene
              bzgl. $PR_2$ wie $R_1$
        \item \label{def:14.8d} Im Dreieck $\triangle PQR$ gibt es \textbf{Innenwinkel}\xindex{Innenwinkel} und
              \textbf{Außenwinkel}\xindex{Außenwinkel}.
    \end{defenum}
\end{definition}

\begin{figure}[ht]
    \centering
    \subfloat[$\angle R_1' P' R_2'$ ist kleiner als $\angle R_1 P R_2$, vgl. \cref{def:14.8c}]{
        \input{figures/smaller-angle.tex}
        \label{fig:def.14.8.1}
    }%
    \subfloat[{\color{green} Innenwinkel} und {\color{blue} Außenwinkel} in $\triangle PQR$, vgl. \cref{def:14.8d}]{
        \input{figures/interiour-exteriour-angles-triangle.tex}
        \label{fig:def.14.8.2}
    }
    \label{fig:def.14.8.0}
    \caption{Situation aus \cref{def:14.8}}
\end{figure}

\begin{bemerkung}\label{bem:14.9}%In Vorlesung: Bemerkung 14.9
    In einem Dreieck ist jeder Innenwinkel kleiner als jeder nicht
    anliegende Außenwinkel.
\end{bemerkung}

\begin{beweis}
    Zeige $\angle PRQ < \angle RQP'$.

    Sei $M$ der Mittelpunkt der Strecke $\overline{QR}$ und $P' \in PQ^+ \setminus \overline{PQ}$.
    Sei $A \in MP^-$ mit $d(P,M) = d(M,A)$.


    \begin{figure}[ht]
        \centering
        \subfloat[Parallelogramm AQPR]{
            \input{figures/geometry-9.tex}
            \label{fig:bem:14.9}
        }%
        \subfloat[Innen- und Außenwinkel von $\triangle PQR$]{
            \input{figures/geometry-7.tex}
            \label{fig:geometry-7}
        }%

        \label{fig:winkel-und-parallelogramm}
        \caption{Situation aus \cref{bem:14.9}}
    \end{figure}

    Es gilt: $d(Q,M) = d(M,R)$ und $d(P,M) = d(M,A)$ sowie
    $\angle PMR = \angle AMQ \Rightarrow \triangle MRQ$ ist
    kongruent zu $\triangle AMQ$, denn eine der beiden Isometrien, die
    $\angle PMR$ auf $\angle AMQ$ abbildet, bildet $R$ auf $Q$ und
    $P$ auf $A$ ab.

    $\Rightarrow \angle MQA = \angle MRP = \angle QRP = \angle PRQ$.

    Noch zu zeigen: $\angle MQA < \angle RQP'$, denn $A$ liegt in der
    selben Halbebene bzgl. $PQ$ wie $M$.
\end{beweis}

\begin{proposition}[Existenz der Parallelen]\label{prop:14.7}%In Vorlesung: Proposition 14.7
    Sei $(X, d, G)$ eine Geometrie mit den Axiomen \ref{axiom:1}~-~\ref{axiom:4}.

    Dann gibt es zu jeder Geraden $g \in G$ und jedem Punkt $P \in X \setminus g$
    mindestens eine Parallele $h \in G$ mit $P \in h$ und $g \cap h = \emptyset$.
\end{proposition}

\begin{figure}[htp]
    \centering
    \input{figures/geometry-6.tex}
    \caption{Situation aus \cref{prop:14.7}}
    \label{fig:geometry-6}
\end{figure}

\begin{beweis}
    Seien $P, Q \in f \in G$ und $\varphi$ die Isometrie, die $Q$ auf $P$ und $P$ auf $P' \in f$
    mit $d(P,P') = d(P, Q)$ abbildet und die Halbebenen bzgl. $f$ erhält.

    \underline{Annahme:} $\varphi(g) \cap g \neq \emptyset$\\
    $\Rightarrow$ Es gibt einen Schnittpunkt $\Set{R} = \varphi(g) \cap g$.\\
    Dann ist $\angle RQP = \angle RQP' < \angle RPP'$ nach
    \cref{bem:14.9} und $\angle RQP = \angle RPP'$, weil
    $\varphi(\angle RQP) = \angle RPP'$.\\
    $\Rightarrow$ Widerspruch\\
    $\Rightarrow \varphi(g) \cap g = \emptyset \qed$
\end{beweis}

\begin{folgerung}\label{folgerung:14.10}%In Vorlesung: Folgerung 14.10
    Die Summe zweier Innenwinkel in einem Dreieck ist kleiner als $\pi$.
\end{folgerung}

D.~h. es gibt eine Isometrie $\varphi$ mit $\varphi(Q) = P$
und $\varphi(QP^+) = PR^+$, sodass $\varphi(R)$ in der gleichen
Halbebene bzgl. $PQ$ liegt wie $R$.

\begin{beweis}
    Die Summe eines Innenwinkels mit den anliegenden Außenwinkeln ist
    $\pi$, d.~h. die beiden Halbgeraden bilden eine Gerade.
\end{beweis}

\begin{figure}[htp]
    \centering
    \includegraphics[width=0.4\linewidth, keepaspectratio]{figures/Spherical_triangle_3d_opti.png}
    \caption{In der sphärischen Geometrie gibt es, im Gegensatz zur euklidischen Geometrie, Dreiecke mit drei $90^\circ$-Winkeln.}
    \label{fig:spherical-triangle}
\end{figure}

\begin{proposition}\label{prop:14.11}%In Vorlesung: Proposition 14.11
    In einer Geometrie mit den Axiomen \ref{axiom:1}~-~\ref{axiom:4}
    ist in jedem Dreieck die Summe der Innenwinkel $\leq \pi$.
\end{proposition}

Sei im Folgenden \enquote{$\IWS$} die \enquote{Innenwinkelsumme}.

\begin{beweis}
    Sei $\triangle$ ein Dreieck mit $\IWS(\triangle) = \pi + \varepsilon$

    \begin{figure}[ht]
        \centering
        \subfloat[Summe der Winkel $\alpha$, $\beta$ und $\gamma$]{
            \resizebox{0.4\linewidth}{!}{\input{figures/three-angles.tex}}
            \label{fig:prop14.11.1}
        }%
        \subfloat[Situation aus \cref{prop:14.11}]{
            \resizebox{0.4\linewidth}{!}{\input{figures/geometry-8.tex}}
            \label{fig:prop14.11.2}
        }
        \label{fig:prop14.11.0}
        \caption{Situation aus \cref{prop:14.11}}
    \end{figure}

    Sei $\alpha$ ein Innenwinkel von $\triangle$.

    \begin{behauptung}
        Es gibt ein Dreieck $\triangle'$ mit
        $\IWS(\triangle') = \IWS(\triangle)$ und einem Innenwinkel
        $\alpha' \leq \frac{\alpha}{2}$.

        Dann gibt es für jedes $n$ ein $\triangle_n$ mit $\IWS(\triangle_n) = \IWS(\triangle)$
        und Innenwinkel $\alpha' \leq \frac{\alpha}{2^n}$. Für $\frac{\alpha}{2^n} < \varepsilon$
        ist dann die Summe der beiden Innenwinkel
        um $\triangle_n$ größer als $\pi \Rightarrow$ Widerspruch zu
        \cref{folgerung:14.10}.
    \end{behauptung}

    \begin{beweis}
        Es seien $A, B, C \in X$ und $\triangle $ das Dreieck mit den
        Eckpunkten $A, B, C$ und $\alpha$ sei der Innenwinkel bei $A$,
        $\beta$ der Innenwinkel bei $B$ und $\gamma$ der Innenwinkel bei $C$.

        Sei $M$ der Mittelpunkt der Strecke $\overline{BC}$. Sei außerdem
        $\alpha_1 = \angle CAM$ und $\alpha_2 = \angle BAM$.

        Sei weiter $A' \in MA^-$ mit $d(A', M) = d(A, M)$.

        Die Situation ist in \cref{fig:prop14.11.2} skizziert.

        $ \Rightarrow \triangle(MA'C)$ und
        $\triangle(MAB)$ sind kongruent.
        $\Rightarrow \angle ABM = \angle A'CM$ und $\angle MA'C = \angle MAB$.
        $\Rightarrow \alpha + \beta + \gamma =\IWS(\triangle ABC) = \IWS(\triangle AA'C)$
        und $\alpha_1 + \alpha_2 = \alpha$, also \obda $\alpha_1 \leq \frac{\alpha}{2}$
    \end{beweis}
\end{beweis}
%%%%%%%%%%%%%%%%%%%%%%%%%%%%%%%%%%%%%%%%%%%%%%%%%%%%%%%%%%%%%%%%%%%%%
% Mitschrieb vom 21.01.2014                                         %
%%%%%%%%%%%%%%%%%%%%%%%%%%%%%%%%%%%%%%%%%%%%%%%%%%%%%%%%%%%%%%%%%%%%%
\begin{bemerkung}\label{bem:14.12}%In Vorlesung: Bemerkung 14.12
    In einer euklidischen Ebene ist in jedem Dreieck die Innenwinkelsumme
    gleich $\pi$.
\end{bemerkung}

\begin{figure}[htp]
    \centering
    \input{figures/triangle-2.tex}
    \caption{Situation aus \cref{bem:14.12}}
    \label{fig:14.12}
\end{figure}

\begin{beweis}
    Sei $g$ eine Parallele von $AB$ durch $C$.

    \begin{itemize}
        \item Es gilt $\alpha' = \alpha$ wegen \cref{prop:14.7}.
        \item Es gilt $\beta' = \beta$ wegen \cref{prop:14.7}.
        \item Es gilt $\alpha'' = \alpha'$ wegen \cref{ub11:aufg1}.
    \end{itemize}
    $\Rightarrow \IWS(\triangle ABC) = \gamma + \alpha'' + \beta' = \pi$
\end{beweis}

Aus der Eigenschaft, dass die Innenwinkelsumme von Dreiecken in der euklidischen Ebene
gleich $\pi$ ist, folgen direkt die Kongruenzsätze SWW und WWS über den Kongruenzsatz
WSW.\xindex{Kongruenzsatz!SWW}

\section{Weitere Eigenschaften einer euklidischen Ebene}
\begin{satz}[Strahlensatz]
    In ähnlichen Dreiecken sind Verhältnisse entsprechender Seiten gleich.
\end{satz}

\begin{figure}[htp]
    \centering
    \documentclass[varwidth=true, border=10pt]{standalone}
\usepackage{tkz-euclide}
\usepackage{tkz-fct}
\newcommand{\iu}{{i\mkern1mu}} % imaginary unit

\begin{document}
\usetkzobj{all}
\begin{tikzpicture}
    \tkzSetUpPoint[shape=circle,size=3,color=black,fill=black]
    \tkzSetUpLine[line width=0.5]
    \tkzInit[xmax=4.5,ymax=3,xmin=-1,ymin=0]
    \tkzDefPoints{0/0/O, 3/3/lz, 2/2/z, 3/0/lzx, 2/0/zx}
    \tkzDrawLines(O,lz zx,z)
    \tkzDrawLine[add=0 and 0.2](lzx,lz)
    \tkzAxeXY[ticks=false]

    %\tkzDrawArc[R,line width=1pt,color=red](m1,1.5 cm)(0,180)
    %\tkzDrawArc[R,line width=1pt](m2,2.5 cm)(0,180)
    \tkzDrawPoints(z,lz)
    \tkzLabelPoint[left](z) {$z$}
    \tkzLabelPoint[above right](zx) {$x$}
    \tkzLabelPoint[right](lz) {$\lambda^2 z$}
    \tkzLabelPoint[above right](lzx) {$\lambda^2 x$}
    %\node[red] at ($(m1)+(1.5,-0.2)$)  {$m+1$};
\end{tikzpicture}
\end{document}

    \caption{Strahlensatz}
    \label{fig:hyperbolische-geometrie-2}
\end{figure}

Der Beweis wird hier nicht geführt. Für Beweisvorschläge wäre ich
dankbar.

\begin{figure}[htp]
    \centering
    \documentclass[varwidth=true, border=2pt]{standalone}
\usepackage{tikz}
\usepackage{tkz-euclide}

\begin{document}
\usetkzobj{all}
\begin{tikzpicture}
    \tkzSetUpPoint[shape=circle,size=10,color=black,fill=black]
    \tkzSetUpLine[line width=1]
    \tkzDefPoints{0/0/A, 3/0/B', 2/2/C, 4/4/C'}
    \tkzDefLine[parallel=through C](B',C') \tkzGetPoint{Phelper}
    \tkzInterLL(A,B')(C,Phelper) \tkzGetPoint{B}
    \tkzDrawLine[add=0 and 0.2](A,B')
    \tkzDrawLine[add=0 and 0.2](A,C')
    \tkzDrawSegment(B',C')

    \node at ($(A)+(-0.1,-0.2)$)  {$A$};
    \node at ($(B')+(0.2,-0.2)$)  {$B'$};
    \node at ($(C')+(0,0.4)$)  {$C'$};
    \node at ($(B)+(0.2,-0.2)$)  {$B$};
    \node at ($(C)+(0.28,0.5)$)  {$C$};
    \tkzDrawPolygon[ultra thick,color=blue,fill=blue!20](A,B',C')
    \tkzDrawPolygon[line width=0.3pt,color=red,fill=red!20](A,B,C)
    \tkzDrawPoints(A,B',C',B,C)
    \tkzLabelSegment[below,red](A,B){$c$}
    \tkzLabelSegment[left,red](A,C){$b$}
    \tkzLabelSegment[right,red](B,C){$a$}
    \tkzLabelSegment[below,blue,pos=0.8](A,B'){$c'$}
    \tkzLabelSegment[left,blue,pos=0.8](A,C'){$b'$}
    \tkzLabelSegment[right,blue](B',C'){$a'$}
\end{tikzpicture}
\end{document}

    \caption{Die Dreiecke $\triangle ABC$ und $\triangle AB'C'$ sind ähnlich.}
    \label{fig:triangle-similar}
\end{figure}

\subsection{Flächeninhalt}
\begin{definition}\xindex{Simplizialkomplexe!flächengleiche}%
    \enquote{Simplizialkomplexe} in euklidischer Ebene $(X,d)$ heißen
    \textbf{flächengleich},
    wenn sie sich in kongruente Dreiecke zerlegen lassen.
\end{definition}

\begin{figure}[ht]
    \centering
    \subfloat[Zwei kongruente Dreiecke]{
        \input{figures/rectangle-2.1.tex}
        \label{fig:rectangle-2.1}
    }%
    \subfloat[Zwei weitere kongruente Dreiecke]{
        \input{figures/rectangle-2.2.tex}
        \label{fig:rectangle-2.2}
    }%
    \label{fig:flaechengleichheit}
    \caption{Flächengleichheit}
\end{figure}

Der Flächeninhalt eines Dreiecks ist $\nicefrac{1}{2} \cdot \text{Grundseite} \cdot \text{Höhe}$.

\begin{figure}[htp]
    \centering
    \subfloat[$\nicefrac{1}{2} \cdot |\overline{AB}| \cdot |h_c|$]{
        \resizebox{0.45\linewidth}{!}{\input{figures/triangle-5.tex}}
        \label{fig:triangle-5}
    }%
    \subfloat[$\nicefrac{1}{2} \cdot |\overline{BC}| \cdot |h_a|$]{
        \resizebox{0.45\linewidth}{!}{\input{figures/triangle-4.tex}}
        \label{fig:triangle-4}
    }%
    \caption{Flächenberechnung im Dreieck}
    \label{fig:flaechenberechnung-dreieck}
\end{figure}

\underline{Zu zeigen:} Unabhängigkeit von der gewählten Grundseite.

\begin{figure}[htp]
    \centering
    \input{figures/triangle-3.tex}
    \caption{$\triangle ABL_a$ und $\triangle C{L_C}B$ sind ähnlich, weil $\IWS = \pi$}
    \label{fig:flaechenberechnung-dreieck-2}
\end{figure}

$\xRightarrow{\text{Strahlensatz}} \frac{a}{h_c} = \frac{c}{h_a} \rightarrow a \cdot h_a = c \cdot h_c$

\begin{satz}[Satz des Pythagoras]
    Im rechtwinkligen Dreieck gilt $a^2 + b^2 = c^2$, wobei $c$ die
    Hypotenuse und $a, b$ die beiden Katheten sind.
\end{satz}

\begin{figure}[ht]
    \centering
    \subfloat[$a,b$ sind Katheten und $c$ ist die Hypotenuse]{
        \input{figures/pythagoras.tex}
        \label{fig:pythagoras-bezeichnungen}
    }%
    \subfloat[Beweisskizze]{
        \input{figures/pythagoras-2.tex}
        \label{fig:pythagoras-2}
    }%
    \label{fig:pythagoras}
    \caption{Satz des Pythagoras}
\end{figure}

\begin{beweis}
    $(a+b) \cdot (a+b) = a^2 + 2ab + b^2 = c^2 +4 \cdot (\frac{1}{2} \cdot a \cdot b)$
\end{beweis}

\begin{satz}\label{satz:14.13} %In Vorlesung: Satz 14.13
    Bis auf Isometrie gibt es genau eine euklidische Ebene $(X, d, G)$, nämlich
    $X=\mdr^2$, $d = \text{euklidischer Abstand}$, $G = \text{Menge der üblichen Geraden}$.
\end{satz}
\goodbreak
\begin{beweis}\leavevmode
    \begin{enumerate}[label=(\roman*)]
        \item $(\mdr^2, d_\text{Euklid})$ ist offensichtlich eine euklidische Ebene.
        \item Sei $(X,d)$ eine euklidische Ebene und $g_1, g_2$ Geraden
              in $X$, die sich in einem Punkt $0$ im rechten Winkel
              schneiden.

              Sei $P \in X \setminus (g_1 \cup g_2)$ ein Punkt und $P_X$ der
              Fußpunkt des Lots von $P$ auf $g_1$ (vgl. \cref{ub11:aufg3.c})
              und $P_Y$ der Fußpunkt des Lots von $P$ auf $g_2$.

              Sei $x_P := d(P_X, 0)$ und $y_P := d(P_Y, 0)$.

              In \cref{fig:14.13.0.1} wurde die Situation skizziert.

            \begin{figure}[htp]
                \centering
                \subfloat[Schritt 1]{
                    \resizebox{0.45\linewidth}{!}{\input{figures/coordinate-system-1.tex}}
                    \label{fig:14.13.1}
                }%
                \subfloat[Schritt 2]{
                    \resizebox{0.45\linewidth}{!}{\begin{tikzpicture}
    \tkzSetUpPoint[shape=circle,size=10,color=black,fill=black]
    \tkzSetUpLine[line width=1]
    \tkzDefPoints{0/0/O, 1/0/X, 0/1/Y, 2/1/P}

    \tkzMarkAngle[fill=green!20,size=0.3cm,opacity=.5](X,O,Y)
    \tkzLabelAngle[pos=0.15](X,O,Y){$\cdot$}

    \tkzDrawLine[add=3 and 2](O,X)
    \tkzLabelLine[below,pos=3](O,X){$g_1$}
    \tkzLabelLine[right,pos=3](O,Y){$g_2$}
    \tkzDrawLine[add=3 and 2](O,Y)

    \tkzDefLine[orthogonal=through P,/tikz/overlay](O,X) \tkzGetPoint{helper}
    \tkzInterLL(O,X)(P,helper) \tkzGetPoint{xp}
    \draw [decorate,decoration={brace,amplitude=4pt,mirror}]
        (O) -- (xp) node [black,midway,xshift=0cm, yshift=-0.3cm]
        {\footnotesize $x_P$};

    \tkzDefLine[orthogonal=through P,/tikz/overlay](O,Y) \tkzGetPoint{helper}
    \tkzInterLL(O,Y)(P,helper) \tkzGetPoint{yp}
    \draw [decorate,decoration={brace,amplitude=4pt}]
        (O) -- (yp) node [black,midway,xshift=-0.4cm]
        {\footnotesize $y_P$};

    \tkzDrawPolygon(O,xp,P,yp)

    \tkzLabelPoint[above right](P){$P$}
    \tkzLabelPoint[below left](O){$0$}
    \tkzLabelPoint[below](xp){$P_X$}
    \tkzLabelPoint[left](Y){$P_Y$}
    \node at ($(-2,2)$){$X$};
    \tkzDrawPoints(P,Y,xp)
\end{tikzpicture}}
                    \label{fig:14.13.2}
                }%
                \caption{Beweis zu \cref{satz:14.13}}
                \label{fig:14.13.0.1}
            \end{figure}


              Sei $h:X \rightarrow \mdr^2$ eine Abbildung mit
              $h(P) := (x_P, y_P)$
              Dadurch wird $h$ auf dem Quadranten
              definiert, in dem $P$ liegt, d.~h.
              \[\forall Q \in X \text{ mit } \overline{PQ} \cap g_1 = \emptyset = \overline{PQ} \cap g_2\]

              Fortsetzung auf ganz $X$ durch konsistente Vorzeichenwahl.

			Im Folgenden werden zwei Aussagen gezeigt:
			\begin{enumerate}[label=(\roman*)]
				\item \label{bew:euklid-1} $h$ ist surjektiv
				\item \label{bew:euklid-2} $h$ ist eine Isometrie
			\end{enumerate}

			Da jede Isometrie injektiv ist, folgt aus \ref{bew:euklid-1}
			und \ref{bew:euklid-2}, dass $h$ bijektiv ist.

			Nun zu den Beweisen der Teilaussagen:

		 	\begin{enumerate}[label=(\roman*)]
				\item Sei $(x, y) \in \mdr^2$, z.~B. $x \geq 0, y \geq 0$.
                Sei $P' \in g_1$ mit $d(0, P') = x$ und
                $P'$ auf der gleichen Seite von $g_2$ wie $P$.
				\item \begin{figure}[htp]
                    \centering
                    \begin{tikzpicture}
    \tkzSetUpPoint[shape=circle,size=10,color=black,fill=black]
    \tkzSetUpLine[line width=1]
    \tkzDefPoints{0/0/O, 1/0/X, 0/1/Y, 2/1/P, 3/3/Q}
    \tkzDrawLine[add=3 and 2.2](O,X)
    \tkzLabelLine[below,pos=3](O,X){$g_1$}
    \tkzLabelLine[left,pos=3](O,Y){$g_2$}
    \tkzDrawLine[add=3 and 2.2](O,Y)

    \tkzDefLine[orthogonal=through P,/tikz/overlay](O,X) \tkzGetPoint{helper}
    \tkzInterLL(O,X)(P,helper) \tkzGetPoint{xp}
    \draw [decorate,decoration={brace,amplitude=4pt,mirror}]
        (O) -- (xp) node [black,midway,xshift=0cm, yshift=-0.3cm]
        {\footnotesize $x_P$};

    \tkzDefLine[orthogonal=through P,/tikz/overlay](O,Y) \tkzGetPoint{helper}
    \tkzInterLL(O,Y)(P,helper) \tkzGetPoint{yp}
    \draw [decorate,decoration={brace,amplitude=4pt}]
        (O) -- (yp) node [black,midway,xshift=-0.4cm]
        {\footnotesize $y_P$};

    \tkzDrawPolygon(O,xp,P,yp)

    \tkzDefLine[orthogonal=through Q,/tikz/overlay](O,X) \tkzGetPoint{helper}
    \tkzInterLL(O,X)(Q,helper) \tkzGetPoint{xq}
    \tkzDefLine[orthogonal=through Q,/tikz/overlay](O,Y) \tkzGetPoint{helper}
    \tkzInterLL(O,Y)(Q,helper) \tkzGetPoint{yq}

    \tkzInterLL(yp,P)(Q,xq) \tkzGetPoint{qxp}
    \tkzInterLL(xp,P)(Q,yq) \tkzGetPoint{R}

    \tkzDrawPolygon(O,xq,Q,yq)

    \tkzDrawSegments[green](xp,xq R,Q)
    \tkzDrawSegments[very thick,orange](yp,yq P,R)

    \tkzLabelPoint[above right](P){$P$}
    \tkzLabelPoint[above right](Q){$Q$}
    \tkzLabelPoint[below left](O){$0$}
    \tkzLabelPoint[above](R){$R$}
    \node at ($(-2,2)$){$X$};
    \tkzDrawPoints(P,Q,R)
\end{tikzpicture}

                    \caption{Beweis zu \cref{satz:14.13}}
                    \label{fig:14.13.0.1}
                \end{figure}
                Zu Zeigen: $d(P, Q) = d(h(P), h(Q))$

                $d(P, Q)^2 \overset{\text{Pythagoras}}{=} d(P, R)^2 + d(R, Q)^2 = (y_Q - y_P)^2 + (x_Q - x_P)^2$.

                $h(Q) = (x_Q, y_Q)$
			\end{enumerate}
    \end{enumerate}
\end{beweis}
%%%%%%%%%%%%%%%%%%%%%%%%%%%%%%%%%%%%%%%%%%%%%%%%%%%%%%%%%%%%%%%%%%%%%
% Mitschrieb vom 23.01.2014                                         %
%%%%%%%%%%%%%%%%%%%%%%%%%%%%%%%%%%%%%%%%%%%%%%%%%%%%%%%%%%%%%%%%%%%%%
\section{Hyperbolische Geometrie}
\begin{definition}\xindex{Gerade!hyperbolische}%
    Sei
        \[\mdh:= \Set{z \in \mdc | \Im(z) > 0} = \Set{(x,y) \in \mdr^2 | y > 0}\]
    die obere Halbebene bzw. Poincaré-Halbebene und $G = G_1 \cup G_2$
    mit
        \begin{align*}
            G_1 &= \Set{g_1 \subseteq \mdh | \exists m \in \mdr, r \in \mdr_{>0}: g_1 = \Set{z \in \mdh : |z-m|=r}}\\
            G_2 &= \Set{g_2 \subseteq \mdh | \exists x \in \mdr: g_2 = \Set{z \in \mdh: \Re(z) = x}}
        \end{align*}

    Die Elemente aus $G$ heißen \textbf{hyperbolische Geraden}.
\end{definition}

\begin{bemerkung}[Eigenschaften der hyperbolischen Geraden]
    Die hyperbolischen Geraden erfüllen\dots
    \begin{bemenum}
        \item \dots die Inzidenzaxiome \ref{axiom:1}
        \item \dots das Anordnungsaxiom \ref{axiom:3.2}
        \item \dots nicht das Parallelenaxiom \ref{axiom:5}
    \end{bemenum}
\end{bemerkung}

\begin{beweis}\leavevmode
    \begin{enumerate}[label=\alph*), ref=\theproposition (\alph*)]
        \item Offensichtlich sind \ref{axiom:1.3} und \ref{axiom:1.2}
              erfüllt. Für \ref{axiom:1.1} gilt:\\
              Gegeben $z_1, z_2 \in \mdh$\\
              \textbf{Existenz:}
            \begin{enumerate}
                \item[Fall 1] $\Re(z_1) = \Re(z_2)$\\
                    $\Rightarrow z_1$ und $z_2$ liegen auf
                    \[g = \Set{z \in \mdc | \Re(z) = \Re(z_1) \land \mdh}\]
                    Siehe \cref{fig:hyperbolische-geometrie-axiom-1-1}.
                \item[Fall 2] $\Re(z_1) \neq \Re(z_2)$\\
                    Betrachte nun $z_1$ und $z_2$ als Punkte in der
                    euklidischen Ebene. Die Mittelsenkrechte zu diesen
                    Punkten schneidet die $x$-Achse. Alle Punkte auf
                    der Mittelsenkrechten zu $z_1$ und $z_2$ sind gleich
                    weit von $z_1$ und $z_2$ entfernt. Daher ist
                    der Schnittpunkt mit der $x$-Achse der Mittelpunkt
                    eines Kreises durch $z_1$ und $z_2$ (vgl. \cref{fig:hyperbolische-geometrie-axiom-1-2})
            \end{enumerate}

            \begin{figure}[ht]
                \centering
                \subfloat[Fall 1]{
                    \resizebox{0.45\linewidth}{!}{\input{figures/hyperbolische-geometrie-axiom-1-1.tex}}
                    \label{fig:hyperbolische-geometrie-axiom-1-1}
                }%
                \subfloat[Fall 2]{
                    \resizebox{0.45\linewidth}{!}{\input{figures/hyperbolische-geometrie-axiom-1-2.tex}}
                    \label{fig:hyperbolische-geometrie-axiom-1-2}
                }%
                \label{fig:hyperbolische-geometrie-axiom-1-0}
                \caption{Zwei Punkte liegen in der hyperbolischen Geometrie immer auf genau einer Geraden}
            \end{figure}
        \item Sei $g \in G_1 \dcup G_2$ eine hyperbolische Gerade.\\

              Es existieren disjunkte Zerlegungen von $\mdh \setminus g$:

              \underline{Fall 1:} $g = \Set{z \in \mdh | |z-m| = r} \in G_1$\\
              Dann gilt:
              \[\mdh = \underbrace{\Set{z \in \mdh | |z-m| < r}}_{=:H_1 \text{ (Kreisinneres)}} \dcup \underbrace{\Set{z \in \mdh | |z-m| > r}}_{=:H_2 \text{ (Kreisäußeres)}}\]
              Da $r > 0$ ist $H_1$ nicht leer, da $r \in \mdr$ ist $H_2$ nicht leer.

              \underline{Fall 2:} $g = \Set{z \in \mdh | \Re{z} = x} \in G_2$\\
              Die disjunkte Zerlegung ist:
              \[\mdh = \underbrace{\Set{z \in \mdh | \Re(z) < x}}_{=: H_1 \text{ (Links)}} \dcup \underbrace{\Set{z \in \mdh | \Re(z) > x}}_{=: H_2 \text{ (Rechts)}}\]

              \underline{Zu zeigen:}
              $\forall A \in H_i$, $B \in H_j$ mit
                      $i,j \in \Set{1,2}$ gilt:
                      $\overline{AB} \cap g \neq \emptyset \Leftrightarrow i \neq j$\\
              \enquote{$\Leftarrow$}: $A \in H_1, B \in H_2: \overline{AB} \cap g \neq \emptyset$

              Da $d_\mdh$ stetig ist, folgt diese Richtung
              direkt. Alle Punkte in $H_1$ haben einen Abstand von $m$ der kleiner
              ist als $r$ und alle Punkte in $H_2$ haben einen Abstand von $m$ der
              größer ist als $r$. Da man jede Strecke von $A$ nach $B$ insbesondere
              auch als stetige Abbildung $f: \mdr \rightarrow \mdr_{>0}$ auffassen
              kann, greift der Zwischenwertsatz $\Rightarrow$ $\overline{AB} \cap g \neq \emptyset$

              \enquote{$\Rightarrow$}: $A \in H_i, B \in H_j \text{ mit } i,j \in \Set{1,2}: \overline{AB} \cap g \neq \emptyset \Rightarrow i \neq j$

              Sei $h$ die Gerade, die durch $A$ und $B$ geht.

              Da $A,B \notin g$, aber $A, B \in h$ gilt, haben $g$ und $h$
              insbesondere
              mindestens einen unterschiedlichen Punkt. Aus \ref{axiom:1.1} folgt, dass sich
              $g$ und $h$ in höchstens einen Punkt schneiden. Sei $C$ dieser
              Punkt.

              Aus $A,B \notin g$ folgt: $C \neq A$ und $C \neq B$. Also liegt
              $C$ zwischen $A$ und $B$. Daraus folgt, dass $A$ und $B$ bzgl.
              $g$ in verschiedenen Halbebenen liegen.

        \item Siehe \cref{fig:hyperbolische-halbebene-axiom-5}.
            \begin{figure}[hp]
                \centering
                \input{figures/hyperbolic-geometry-not-parallel.tex}
                \caption{Hyperbolische Geraden erfüllen \ref{axiom:5} nicht.}
                \label{fig:hyperbolische-halbebene-axiom-5}
            \end{figure}
    \end{enumerate}
\end{beweis}

\begin{definition}\xindex{Möbiustransformation}%
    Es seien $a,b,c,d \in \mdr$ mit $ad - bc \neq 0$ und
    $\sigma: \mdc \rightarrow \mdc$ eine Abbildung definiert durch
    \[\sigma(z) := \frac{az + b}{cz+d}\]

    $\sigma$ heißt \textbf{Möbiustransformation}.
\end{definition}

\begin{proposition}%In Vorlesung: Proposition 15.2
    \begin{propenum}
        \item Die Gruppe $\SL_2(\mdr)$ operiert auf $\mdh$ durch die Möbiustransformation
              \[\sigma(z):= \begin{pmatrix}a & b\\c & d\end{pmatrix} \circ z := \frac{az + b}{cz + d}\]
        \item Die Gruppe $\PSL_2(\mdr) = \SL_2(\mdr) /_{(\pm I)}$ operiert durch $\sigma$ auf $\mdh$.
        \item \label{prop:15.2c} $\PSL_2(\mdr)$ operiert auf $\mdr \cup \Set{\infty}$.
              Diese Gruppenoperation ist 3-fach transitiv, d.~h. zu
              $x_0 < x_1 < x_\infty \in \mdr$ gibt es genau ein
              $\sigma \in \PSL_2(\mdr)$ mit $\sigma(x_0) = 0$,
              $\sigma(x_1) = 1$, $\sigma(x_\infty) = \infty$.
        \item \label{prop:15.2d} $\SL_2(\mdr)$ wird von den Matrizen
              \[\underbrace{\begin{pmatrix}\lambda & 0\\ 0 & \lambda^{-1}\end{pmatrix}}_{=: A_{\lambda}},
                \underbrace{\begin{pmatrix}1 & t\\ 0 & 1\end{pmatrix}}_{=: B_{t}} \text{ und }
                \underbrace{\begin{pmatrix}0 & 1\\-1 & 0\end{pmatrix}}_{=: C} \text{ mit } t, \lambda \in \mdr^\times\]
              erzeugt.
        \item \label{prop:15.2e} $\PSL_2(\mdr)$ operiert auf $G$.
    \end{propenum}
\end{proposition}

\begin{beweis}\leavevmode
    \begin{enumerate}[label=\alph*)]
        \item Sei $z = x + \iu y \in \mdh$, d.~h. $y>0$ und
              $\sigma=\begin{pmatrix}a&b\\c&d\end{pmatrix} \in \SL_2(\mdr)$
              \begin{align*}
                \Rightarrow \sigma(z) &= \frac{a(x + \iu y) + b}{c(x + \iu y) +d}\\
                &= \frac{(ax + b) + \iu ay}{(cx + d) + \iu cy} \cdot \frac{(cx+d)-\iu cy}{(cx+d)-\iu cy}\\
                &=   \frac{(ax+b)(cx+d) + aycy}{(cx+d)^2 + (cy)^2} + \iu \frac{ay(cx + d) - (ax+b)cy}{(cx+d)^2 + (cy)^2}\\
                &= \frac{axcx+axd+bcx+bd+aycy}{(cx+d)^2 + (cy)^2} + \iu \frac{(ad-bc)y}{(cx+d)^2 + (cy)^2}\\
                &\overset{\mathclap{\SL_2(\mdr)}}{=}\hspace{5 mm} \frac{ac(x^2+y^2)+adx+bcx+bd}{(cx+d)^2 + (cy)^2} + \iu \frac{y}{(cx+d)^2 + (cy)^2}
              \end{align*}
              $\Rightarrow \Im(\sigma(z)) = \frac{y}{(cx+d)^2 + (cy)^2} > 0$

              Die Abbildung bildet also nach $\mdh$ ab. Außerdem gilt:
              \[\begin{pmatrix}1&0\\0&1\end{pmatrix} \circ z = \frac{x+\iu y}{1} = x + \iu y = z\]
              und
              \begin{align*}
                \begin{pmatrix}a&b\\c&d\end{pmatrix} \circ  \left ( \begin{pmatrix}a'&b'\\c'&d'\end{pmatrix}   \circ z \right )&=
                            \begin{pmatrix}a&b\\c&d\end{pmatrix} \circ \frac{a'z + b'}{c'z + d'}\\
                    &= \frac{a \frac{a'z + b'}{c'z + d'} + b}{c \frac{a'z + b'}{c'z + d'} + d}\\
                    &= \frac{\frac{a(a'z+b') + b(c'z+d')}{c'z+d'}}{\frac{c(a'z+b')+d(c'z+d')}{c'z+d'}}\\
                    &= \frac{a(a'z+b')+b(c'z+d')}{c(a'z+b') + d(c'z+d')}\\
                    &= \frac{(aa'+bc')z + ab' + bd'}{(ca'+db')z + cb' + dd'}\\
                    &= \begin{pmatrix}aa'+bc'&ab'+bd'\\ca'+db'&cb'+dd'\end{pmatrix} \circ z\\
                    &= \left ( \begin{pmatrix}a&b\\c&d\end{pmatrix} \cdot \begin{pmatrix}a'&b'\\c'&d'\end{pmatrix} \right ) \circ z
              \end{align*}
        \item Es gilt $\sigma(z) = (-\sigma)(z)$ für alle $\sigma \in \SL_2(\mdr)$
              und $z \in \mdh$.
        \item Ansatz: $\sigma = \begin{pmatrix}a & b\\c & d\end{pmatrix}$
              $\sigma(x_0) = \frac{ax_0 + b}{c x_0 + d} \overset{!}{=} 0$
              $\Rightarrow a x_0 + b = 0 \Rightarrow b = -a x_0$\\
              $\sigma(x_\infty) = \infty \Rightarrow c x_\infty + d = 0 \Rightarrow d = - c x_\infty$\\
              $\sigma(x_1) = 1 \Rightarrow a x_1 + b = c x_1 + d$\\
              $a (x_1 - x_0) = c (x_1 - x_\infty) \Rightarrow c = a \frac{x_1 - x_0}{x_1 - x_\infty}$\\
              $\Rightarrow - a^2 \cdot x_\infty \frac{x_1 - x_0}{x_1 - x_\infty} + a^2 x_0 \frac{x_1 - x_0}{x_1 - x_\infty} = 1$\\
              $\Rightarrow a^2 \frac{x_1 - x_0}{x_0 - x_\infty} (x_0 - x_\infty) = 1$
              $\Rightarrow a^2 = \frac{x_1 - x_\infty}{(x_1 - x_\infty) (x_1 - x_0)}$
        \item Es gilt:
              \begin{align*}
                A_{\lambda}^{-1} &= A_{\frac{1}{\lambda}}\\
                B_t^{-1}         &= B_{-t}\\
                C^{-1}           &= C^3
              \end{align*}

              Daher genügt es zu zeigen, dass man mit $A_{\lambda}$, $B_t$ und $C$ alle Matrizen
              aus $\SL_2(\mdr)$ erzeugen kann, genügt es also von einer beliebigen
              Matrix durch Multiplikation mit Matrizen der Form $A_{\lambda}$,
              $B_t$ und $C$ die Einheitsmatrix zu generieren.

              Sei also
              \[M = \begin{pmatrix} a & b\\ c & d\end{pmatrix} \in \SL_2(\mathbb{R})\]
              beliebig.

              \underline{Fall 1:} $a = 0$\\
              Da $M \in \SL_2(\mdr)$ ist, gilt $\det{M} = 1 = ad - bc = -bc$.
              Daher ist insbesondere $c \neq 0$. Es folgt:

              \[\begin{pmatrix} 0 & 1\\ -1 & 0\end{pmatrix} \cdot \begin{pmatrix} a & b\\ c & d\end{pmatrix} = \begin{pmatrix} c & d\\ -a & -b\end{pmatrix}\]

              Gehe zu Fall 2.

              \underline{Fall 2:} $a \neq 0$\\
              Nun wird in $M$ durch $M \cdot A_{\frac{1}{a}}$ an der Stelle von
              $a$ eine $1$ erzeugt:

              \[\begin{pmatrix} a & b\\ c & d\end{pmatrix} \cdot \begin{pmatrix} \frac{1}{a} & 0\\ 0 & a\end{pmatrix} = \begin{pmatrix} 1 & ab\\ \frac{c}{a} & ad\end{pmatrix}\]

              Gehe zu Fall 3.

              \underline{Fall 3:} $a = 1$\\
              \[\begin{pmatrix} 1 & b\\ c & d\end{pmatrix} \cdot \begin{pmatrix} 1 & -b\\ 0 & 1\end{pmatrix} = \begin{pmatrix} 1 & 0\\ c & d-bc\end{pmatrix}\]
              Da wir $\det M = 1 = ad - bc = d - bc$ wissen, gilt sogar
              $M_{2,2} = 1$.

              Gehe zu Fall 4.

              \underline{Fall 4:} $a = 1$, $b=0$, $d=1$\\
              \[A_{-1} C B_c C \begin{pmatrix}1 & 0 \\ c & 1\end{pmatrix} = \begin{pmatrix}1 & 0 \\ 0 & 1\end{pmatrix}\]
              Daher erzeugen Matrizen der Form $A_{\lambda}$, $B_t$ und $C$
              die Gruppe $\SL_2{\mdr}$. $\qed$
        \item Es genügt die Aussage für Matrizen aus \cref{prop:15.2d}
              zu zeigen.
            \begin{itemize}
                \item $\sigma = \begin{pmatrix}\lambda & 0\\ 0 & \lambda^{-1}\end{pmatrix}$, also $\sigma(z) = \lambda^2 z$.
                      Daraus ergeben sich die Situationen, die in \cref{fig:prop15.2.e.fall1.1} und
                      \cref{fig:prop15.2.e.fall1.2} dargestellt sind.
                    \begin{figure}[ht]
                        \centering
                        \subfloat[Fall 1]{
                            \resizebox{0.45\linewidth}{!}{\documentclass[varwidth=true, border=10pt]{standalone}
\usepackage{tkz-euclide}
\usepackage{tkz-fct}
\newcommand{\iu}{{i\mkern1mu}} % imaginary unit

\begin{document}
\usetkzobj{all}
\begin{tikzpicture}
    \tkzSetUpPoint[shape=circle,size=3,color=black,fill=black]
    \tkzSetUpLine[line width=1]
    \tkzInit[xmax=7,ymax=3,xmin=-1,ymin=0]
    \tkzDefPoints{2/0/m1,4/0/m2,1/1.1/a1,1/0/a1x, 2.5/2.0/a2,2.5/0/a2x}
    \tkzDrawSegments(a1x,a1 a2x,a2)
    \tkzAxeXY[ticks=false]

    \tkzDrawArc[R,line width=1pt,color=red](m1,1.5 cm)(0,180)
    \tkzDrawArc[R,line width=1pt](m2,2.5 cm)(0,180)
    \tkzDrawPoints(m1,m2,a1,a2)
    \tkzLabelPoint[above](m1) {$m$}
    \tkzLabelPoint[above](m2) {$\lambda^2 m$}
    \tkzLabelPoint[above](a1) {$m+\iu r$}
    \tkzLabelPoint[above](a2) {$\lambda^2 m+\iu \lambda^2 r$}
    \node[red] at ($(m1)+(1.5,-0.2)$)  {$m+1$};
\end{tikzpicture}
\end{document}
}
                            \label{fig:prop15.2.e.fall1.1}
                        }%
                        \subfloat[Fall 2 (Strahlensatz)]{
                            \resizebox{0.45\linewidth}{!}{\documentclass[varwidth=true, border=10pt]{standalone}
\usepackage{tkz-euclide}
\usepackage{tkz-fct}
\newcommand{\iu}{{i\mkern1mu}} % imaginary unit

\begin{document}
\usetkzobj{all}
\begin{tikzpicture}
    \tkzSetUpPoint[shape=circle,size=3,color=black,fill=black]
    \tkzSetUpLine[line width=0.5]
    \tkzInit[xmax=4.5,ymax=3,xmin=-1,ymin=0]
    \tkzDefPoints{0/0/O, 3/3/lz, 2/2/z, 3/0/lzx, 2/0/zx}
    \tkzDrawLines(O,lz zx,z)
    \tkzDrawLine[add=0 and 0.2](lzx,lz)
    \tkzAxeXY[ticks=false]

    %\tkzDrawArc[R,line width=1pt,color=red](m1,1.5 cm)(0,180)
    %\tkzDrawArc[R,line width=1pt](m2,2.5 cm)(0,180)
    \tkzDrawPoints(z,lz)
    \tkzLabelPoint[left](z) {$z$}
    \tkzLabelPoint[above right](zx) {$x$}
    \tkzLabelPoint[right](lz) {$\lambda^2 z$}
    \tkzLabelPoint[above right](lzx) {$\lambda^2 x$}
    %\node[red] at ($(m1)+(1.5,-0.2)$)  {$m+1$};
\end{tikzpicture}
\end{document}
}
                            \label{fig:prop15.2.e.fall1.2}
                        }%
                        \label{fig:prop15.2.e.fall1.0}
                        \caption{Beweis von \cref{prop:15.2e} für eine Diagonalmatrix}
                    \end{figure}
                \item Offensichtlich gilt die Aussage für $\sigma = \begin{pmatrix}1 & a\\0 & 1\end{pmatrix}$
                \item Sei nun $\sigma = \begin{pmatrix}0 & 1\\-1 & 0\end{pmatrix}$, also $\sigma(z) = - \frac{1}{z}$
                    \begin{figure}[htp]
                        \centering
                        \input{figures/inversion-am-kreis.tex}
                        \caption{Inversion am Kreis}
                        \label{fig:inversion-am-kreis}
                    \end{figure}
            \end{itemize}
    \end{enumerate}
\end{beweis}

%%%%%%%%%%%%%%%%%%%%%%%%%%%%%%%%%%%%%%%%%%%%%%%%%%%%%%%%%%%%%%%%%%%%%
% Mitschrieb vom 28.01.2014                                         %
%%%%%%%%%%%%%%%%%%%%%%%%%%%%%%%%%%%%%%%%%%%%%%%%%%%%%%%%%%%%%%%%%%%%%
\begin{bemerkung}%In Vorlesung: Bemerkung 15.3
    Zu hyperbolischen Geraden $g_1, g_2$ gibt es $\sigma \in \PSL_2(\mdr)$
    mit $\sigma(g_1) = g_2$.
\end{bemerkung}
\begin{beweis}
    Nach \cref{prop:15.2c} gibt es $\sigma$ mit $\sigma(a_1) = b_1$
    und $\sigma(a_2) = b_2$. Dann existiert $\sigma(g_1) := g_2$
    wegen dem Inzidenzaxiom \ref{axiom:1} und ist eindeutig bestimmt.
\end{beweis}

\begin{definition}\xindex{Doppelverhältnis}%In Vorlesung: Def+Prop 15.4
    Seien $z_1, z_2, z_3, z_4 \in \mdc$ paarweise verschieden.

    Dann heißt
    \[\DV(z_1, z_2, z_3, z_4) := \frac{\frac{z_1 - z_4}{z_1 - z_2}}{\frac{z_3 - z_4}{z_3 - z_2}} = \frac{(z_1 - z_4) \cdot (z_3 - z_2)}{(z_1 - z_2) \cdot (z_3 - z_4)}\]
    \textbf{Doppelverhältnis} von
    $z_1, \dots, z_4$.
\end{definition}

\begin{bemerkung}[Eigenschaften des Doppelverhältnisses]
    \begin{bemenum}
        \item $\DV(z_1, \dots, z_4) \in \mdc \setminus \Set{0,1}$
        \item \label{bem:15.4b.ii} $\DV(z_1, z_4, z_3, z_2) = \frac{1}{\DV(z_1, z_2, z_3, z_4)}$
        \item \label{bem:69.c} $\DV(z_3, z_2, z_1, z_4) = \frac{1}{\DV(z_1, z_2, z_3, z_4)}$
        \item $\DV$ ist auch wohldefiniert, wenn eines der $z_i = \infty$
              oder wenn zwei der $z_i$ gleich sind.
        \item $\DV(0, 1, \infty, z_4) = z_4$ (Der Fall $z_4 \in \Set{0, 1, \infty}$ ist zugelassen).
        \item \label{bem:15.4d} Für $\sigma \in \PSL_2(\mdc)$ und $z_1, \dots, z_4 \in \mdc \cup \Set{\infty}$
              ist
              \[\DV(\sigma(z_1), \sigma(z_2), \sigma(z_3), \sigma(z_4)) = \DV(z_1, z_2, z_3, z_4)\]
              und für $\sigma(z) = \frac{1}{\overline{z}}$ gilt
              \[\DV(\sigma(z_1), \sigma(z_2), \sigma(z_3), \sigma(z_4)) = \overline{\DV(z_1, z_2, z_3, z_4)}\]
        \item \label{bem:15.4e} $\DV(z_1, z_2, z_3, z_4) \in \mdr \cup \Set{\infty} \Leftrightarrow z_1, \dots, z_4$
              liegen auf einer hyperbolischen Geraden.
    \end{bemenum}
\end{bemerkung}

\begin{beweis}\leavevmode
    \begin{enumerate}[label=\alph*)]
        \item $\DV(z_1, \dots, z_4) \neq 0$, da $z_i$ paarweise verschieden\\
              $\DV(z_1, \dots, z_4) \neq 1$, da:

            \begin{adjustwidth}{2.5em}{0pt}
                \underline{Annahme:} $\DV(z_1, \dots, z_4) = 1$
                \begin{align*}
                    \Leftrightarrow (z_1 - z_2) (z_3 - z_4) &= (z_1 - z_4) (z_3 - z_2)\\
                    \Leftrightarrow z_1 z_3 - z_2 z_3 - z_1 z_4 + z_2 z_4 &= z_1 z_3 - z_3 z_4 - z_1 z_2 + z_2 z_4\\
                    \Leftrightarrow z_2 z_3 + z_1 z_4 &= z_3 z_4 + z_1 z_2\\
                    \Leftrightarrow z_2 z_3 - z_3 z_4 &= z_1 z_2 - z_1 z_4\\
                    \Leftrightarrow z_3 (z_2 - z_4) &= z_1 (z_2 - z_4)\\
                    \Leftrightarrow z_3 &= z_1 \text{ oder } z_2 = z_4
                \end{align*}
                Alle $z_i$ sind paarweise verschieden $\Rightarrow$ Widerspruch $\qed$
            \end{adjustwidth}
        \item $\DV(z_1, z_4, z_3, z_2) = \frac{(z_1 - z_2) \cdot (z_3 - z_4)}{(z_1 - z_4) \cdot (z_3 - z_2)} = \frac{1}{\DV(z_1, z_2, z_3, z_4)}$
        \item $\DV(z_3, z_2, z_1, z_4) = \frac{(z_3 - z_4) \cdot (z_1 - z_2)}{(z_3 - z_2) \cdot (z_1 - z_4)} = \frac{1}{\DV(z_1, z_2, z_3, z_4)}$
        \item Zwei der $z_i$ dürfen gleich sein, da:
            \begin{itemize}
                \item[Fall 1] $z_1 = z_4$ oder $z_3 = z_2$\\
                    In diesem Fall ist $\DV(z_1, \dots, z_4) = 0$
                \item[Fall 2] $z_1 = z_2$ oder $z_3 = z_4$\\
                    Mit der Regel von L'Hospital folgt, dass in diesem
                    Fall $\DV(z_1, \dots, z_4) = \infty$ gilt.
                \item[Fall 3] $z_1 = z_3$ oder $z_2 = z_4$\\
                    Durch Einsetzen ergibt sich $\DV(z_1, \dots, z_4)=1$.
            \end{itemize}

            Im Fall, dass ein $z_i = \infty$ ist, ist
            entweder $\DV(0, 1, \infty, z_4) = 0$ oder $\DV(0, 1, \infty, z_4) \pm \infty$
        \item $\DV(0, 1, \infty, z_4) = \frac{(0- z_4) \cdot (\infty - 1)}{(0 -1) \cdot (\infty - z_4)} = \frac{z_4 \cdot (\infty - 1)}{\infty - z_4} = z_4$
        \item Wenn jemand diesen Beweis führt, bitte an info@martin-thoma.de schicken.%TODO
        \item  Sei $\sigma \in \PSL_2(\mdc)$ mit $\sigma(z_1) = 0$, $\sigma(z_2) = 1$,
            $\sigma(z_3) = \infty$. Ein solches $\sigma$ existiert, da man drei
            Parameter von $\sigma$ wählen darf.

            $\overset{\mathclap{\crefabbr{bem:15.4d}}}{\Rightarrow}\hspace{4mm} \DV(z_1, \dots, z_4) = \DV(0, 1, \infty, \sigma(z_4))$\\
            $\Rightarrow\hspace{4mm} \DV(z_1, \dots, z_4) \in \mdr \cup \Set{\infty}$\\
            $\Leftrightarrow \sigma(z_4) \in \mdr \cup \Set{\infty}$

            Behauptung folgt, weil $\sigma^{-1}(\mdr \cup \infty)$ ein Kreis oder
            eine Gerade in $\mdc$ ist.
    \end{enumerate}
\end{beweis}

\begin{definition}\xindex{Metrik!hyperbolische}%
    Für $z_1, z_2 \in \mdh$ sei $g_{z_1, z_2}$ die eindeutige hyperbolische
    Gerade durch $z_1$ und $z_2$ und $a_1, a_2$ die
    \enquote{Schnittpunkte} von $g_{z_1, z_2}$ mit $\mdr \cup \Set{\infty}$.

    Dann sei $d_{\mdh}(z_1, z_2) := \frac{1}{2} | \ln \DV(a_1, z_1, a_2, z_2) |$
    und heiße \textbf{hyperbolische Metrik}.
\end{definition}

\begin{behauptung}
    Für $z_1, z_2 \in \mdh$ sei $g_{z_1, z_2}$ die eindeutige hyperbolische
    Gerade durch $z_1$ und $z_2$ und $a_1, a_2$ die
    \enquote{Schnittpunkte} von $g_{z_1, z_2}$ mit $\mdr \cup \Set{\infty}$.

    Dann gilt:
    \[\frac{1}{2} | \ln \DV(a_1, z_1, a_2, z_2) | = \frac{1}{2} | \ln \DV(a_2, z_1, a_1, z_2) |\]
\end{behauptung}

\begin{beweis}
    Wegen \cref{bem:69.c} gilt:
    \[\DV(a_1, z_1, a_2, z_2) = \frac{1}{\DV(a_2, z_1, a_1, z_2)}\]
    Außerdem gilt:
    \[\ln \frac{1}{x} = \ln x^{-1} = (-1) \cdot \ln x = - \ln x\]
    Da der $\ln$ im Betrag steht, folgt direkt:
    \[\frac{1}{2} | \ln \DV(a_1, z_1, a_2, z_2) | = \frac{1}{2} | \ln \DV(a_2, z_1, a_1, z_2)|\]
    Es ist also egal in welcher Reihenfolge die \enquote{Schnittpunkte} mit
    der $x$-Achse im Doppelverhältnis genutzt werden. $\qed$
\end{beweis}

\begin{behauptung}
    Die hyperbolische Metrik ist eine Metrik auf $\mdh$.
\end{behauptung}

\begin{beweis}
    Wegen \cref{bem:15.4d} ist
        \[d(z_1, z_2) := d(\sigma(z_1), \sigma(z_2)) \text{ mit } \sigma(a_1) = 0,\; \sigma(a_2) = \infty\]
    d.~h. $\sigma(g_{z_1, z_2}) = \iu \mdr$ (imaginäre Achse).

    also gilt \obda $z_1 = \iu a$ und $z_2 = \iu b$ mit $a,b \in \mdr$ und $a < b$.
    \begin{align*}
        2d(\iu a, \iu b)&= \mid \ln \DV(0, \iu a, \infty, \iu b) \mid \\
                        &= \mid \ln \frac{(0 - \iu b) (\infty - \iu a)}{(0 - \iu a)(\infty - \iu b)} \mid \\
                        &= \mid \ln \frac{b}{a} \mid\\
                        &= \ln b - \ln a
    \end{align*}

    Also: $d(z_1, z_2) \geq 0$, $d(z_1, z_2) = 0 \gdw z_1 = z_2$

    \begin{align*}
        2 d(z_2, z_1) &= \mid \ln \DV(a_2, z_2, a_1, z_1) \mid\\
                      &= \mid \ln \DV(\infty, \iu b, 0, \iu a) \mid\\
            &\overset{\mathclap{\crefabbr{bem:15.4b.ii}}}{=}\hspace{5mm} \mid \ln \DV(0, \iu b, \infty, \iu a) \mid \\
                      &= 2 d(z_1, z_2)
    \end{align*}

    Liegen drei Punkte $z_1, z_2, z_3 \in \mdc$ auf einer hyperbolischen
    Geraden, so gilt $d(z_1, z_3) = d(z_1, z_2) + d(z_2, z_3)$
    (wenn $z_2$ zwischen $z_1$ und $z_3$ liegt).

    Dreiecksungleichung: Beweis ist umständlich und wird hier nicht geführt. Es sei auf die Vorlesung \enquote{Hyperbolische Geometrie}
    verwiesen.
\end{beweis}

\begin{satz}%In Vorlesung: Satz 15.6
    Die hyperbolische Ebene $\mdh$ mit der hyperbolischen Metrik $d$
    und den hyperbolischen Geraden bildet eine \enquote{nichteuklidische Geometrie},
    d.~h. die Axiome~\ref{axiom:1}~-~\ref{axiom:4} sind erfüllt,
    aber Axiom~\ref{axiom:5} ist verletzt.
\end{satz}

% Die Übungsaufgaben sollen ganz am Ende des Kapitels sein.
\clearpage
\section*{Übungsaufgaben}
\addcontentsline{toc}{section}{Übungsaufgaben}

\begin{aufgabe}\label{ub11:aufg1}
    Seien $(X, d)$ eine absolute Ebene und $P, Q, R \in X$ Punkte.
    Der \textit{Scheitelwinkel}\xindex{Scheitelwinkel} des Winkels $\angle PQR$ ist
    der Winkel, der aus den Halbgeraden $QP^-$ und $QR^-$ gebildet
    wird. Die \textit{Nebenwinkel}\xindex{Nebenwinkel} von $\angle PQR$
    sind die von $QP^+$ und $QR^-$ bzw. $QP^-$ und $QR^+$ gebildeten
    Winkel.

    Zeigen Sie:
    \begin{aufgabeenum}
        \item Die beiden Nebenwinkel von $\angle PQR$ sind gleich.
        \item Der Winkel $\angle PQR$ ist gleich seinem Scheitelwinkel.
    \end{aufgabeenum}
\end{aufgabe}

\begin{aufgabe}\label{ub11:aufg3}
    Sei $(X, d)$ eine absolute Ebene. Der \textit{Abstand}\xindex{Abstand} eines
    Punktes $P$ zu einer Menge $Y \subseteq X$ von Punkten ist
    definiert durch $d(P, Y) := \inf{d(P, y) | y \in Y}$.

    Zeigen Sie:
    \begin{aufgabeenum}
        \item \label{ub11:aufg3.a} Ist $\triangle ABC$ ein Dreieck, in dem die Seiten
              $\overline{AB}$ und $\overline{AC}$ kongruent sind, so
              sind die Winkel $\angle ABC$ und $\angle BCA$ gleich.
        \item \label{ub11:aufg3.b} Ist $\triangle ABC$ ein beliebiges Dreieck, so liegt
              der längeren Seite der größere Winkel gegenüber und
              umgekehrt.
        \item \label{ub11:aufg3.c} Sind $g$ eine Gerade und $P \notin g$ ein Punkt, so gibt
              es eine eindeutige Gerade $h$ mit $P \in h$ und die
              $g$ im rechten Winkel schneidet. Diese Grade heißt
              \textit{Lot}\xindex{Lot} von $P$ auf $g$ und der
              Schnittpunkt des Lots mit $g$ heißt \textit{Lotfußpunkt}\xindex{Lotfußpunkt}.
    \end{aufgabeenum}
\end{aufgabe}

\begin{aufgabe}\label{ub-tut-24:a1}
    Seien $f, g, h \in G$ und paarweise verschieden.

    Zeigen Sie: $f \parallel g \land g \parallel h \Rightarrow f \parallel h$
\end{aufgabe}

\begin{aufgabe}\label{ub-tut-24:a3}%
    Beweise den Kongruenzsatz $SSS$.
\end{aufgabe}

