\documentclass[usepdftitle=false,hyperref={pdfpagelabels=false}]{beamer}
\usepackage{../templates/myStyle}

\begin{document}
\title{\titleText}
\subtitle{JUnit, Vererbung, toString(), Interfaces}
\author{\tutor}
\date{\today}
\subject{Programmieren}

\frame{\titlepage}

\frame{
    \frametitle{Inhaltsverzeichnis}
    \setcounter{tocdepth}{1}
    \tableofcontents
    \setcounter{tocdepth}{2}
}

\section{Einleitung}
\subsection{Quiz}
\begin{frame}{Quiz: Vererbung}
    \begin{minipage}[b]{0.45\linewidth}
        \inputminted[linenos=false, numbersep=5pt, tabsize=4, fontsize=\tiny, label=Animal.java, frame=lines]{java}{Animal.java}
        \inputminted[linenos=false, numbersep=5pt, tabsize=4, fontsize=\tiny, label=Jungle.java, frame=lines]{java}{Jungle.java}
    \end{minipage}
    \hspace{0.5cm}
    \begin{minipage}[b]{0.45\linewidth}
        \inputminted[linenos=false, numbersep=5pt, tabsize=4, fontsize=\tiny, label=Tiger.java, frame=lines]{java}{Tiger.java}
        \inputminted[linenos=false, numbersep=5pt, tabsize=4, fontsize=\tiny, label=Cat.java, frame=lines]{java}{Cat.java}
        \begin{itemize}
            \item Gibt es einen Compiler-Fehler?
            \item Gibt es einen Laufzeit-Fehler?
            \item Gibt es eine Ausgabe? Welche?
        \end{itemize}
    \end{minipage}
\end{frame}

\begin{frame}{Quiz: Antwort}
    \begin{minipage}[b]{0.45\linewidth}
        \inputminted[linenos=false, numbersep=5pt, tabsize=4, fontsize=\tiny, label=Animal.java, frame=lines]{java}{Animal.java}
        \inputminted[linenos=false, numbersep=5pt, tabsize=4, fontsize=\tiny, label=Jungle.java, frame=lines]{java}{Jungle.java}
    \end{minipage}
    \hspace{0.5cm}
    \begin{minipage}[b]{0.45\linewidth}
        \inputminted[linenos=false, numbersep=5pt, tabsize=4, fontsize=\tiny, label=Tiger.java, frame=lines]{java}{Tiger.java}
        \inputminted[linenos=false, numbersep=5pt, tabsize=4, fontsize=\tiny, label=Cat.java, frame=lines]{java}{Cat.java}
        \begin{itemize}
            \item null
            \item Cat:null
            \item Cat:null
            \item null
            \item null
        \end{itemize}
    \end{minipage}
\end{frame}

\begin{frame}{Erklärung}
    \begin{itemize}
        \item Zeile 2 und 3: \myCode{sound} im Konstruktor von \myCode{Cat} ist eine
              lokale Variable, kein Attribut
        \item In Java werden nur Methoden vererbt
            \begin{itemize}
                \item Klassen: Signatur und Implementierung
                \item Interfaces: Nur Signatur
            \end{itemize}
    \end{itemize}

    Mehr dazu später
\end{frame}

\section{JUnit}
\subsection{Allgemeines}
\begin{frame}{JUnit: Allgemeines}
    JUnit \dots
    \begin{itemize}[<+->]
        \item ist ein Java-Paket
        \item ist ein Framework zum Testen von  Java-Programmen
        \item ist SEHR verbreitet
        \item dient der Erstellung von Unit-Tests
        \item wurde von Erich Gamma und Kent Beck erstellt
    \end{itemize}
\end{frame}

\subsection{Beispiel}
\begin{frame}{JUnit: Beispiel}
    \inputminted[linenos=true, numbersep=5pt, tabsize=4, fontsize=\tiny, label=LevenshteinCompilationTest.java, frame=lines]{java}{LevenshteinCompilationTest.java}
\end{frame}

\framedgraphic{Eclipse: JUnit}{Eclipse-JUnit-new-test.png}
\framedgraphic{Eclipse: JUnit}{Eclipse-JUnit-project-explorer.png}
\framedgraphic{Eclipse: JUnit}{Eclipse-JUnit-execution-button.png}
\framedgraphic{Eclipse: JUnit}{Eclipse-JUnit-execution-result.png}
\framedgraphic{Eclipse: JUnit}{Eclipse-JUnit-detailed-results.png}
\framedgraphic{Eclipse: JUnit}{Eclipse-JUnit-new-filter-trace.png}

\subsection{Fehler}
\begin{frame}{JUnit: Fehler}
    \begin{alertblock}{Fehler}
        The import org.junit cannot be resolved
    \end{alertblock}
    \begin{block}{Lösung}
        \begin{itemize}
            \item \href{https://github.com/KentBeck/junit/downloads}{Hier} junit-4.11.jar mit Hamcrest herunterladen
            \item \menu{Project > Properties > Java Build Path > Libraries > Add External JARs...}
            \item \texttt{junit-4.11.jar} auswählen
            \item Auf OK klicken
        \end{itemize}
    \end{block}
\end{frame}

\section{Vererbung}
\subsection{Allgemeines}
\begin{frame}{Allgemeines}
    Vererbung \dots
    \begin{itemize}[<+->]
        \item ist ein Schlüsselelement der OOP
        \item ist in Java eingeschränkt: Eine Klasse erbt in Java
              von genau einer anderen Klasse
            \begin{itemize}
              \item alle Klassen erben von \href{http://docs.oracle.com/javase/7/docs/api/java/lang/Object.html}{Object}
            \end{itemize}
        \item dient der Spezialisierung
    \end{itemize}
\end{frame}

\begin{frame}{Beispiel}
    Wo kann Vererbung nützlich sein?
    \begin{itemize}[<+->]
        \item Oberklasse Liste, Unterklassen SinglyLinkedList und
              DoubleLinkedList
          \begin{itemize}
            \item \myCode{contains()} ist gleich
            \item \myCode{append()} ist unterschiedlich
            \item \myCode{remove()} ist unterschiedlich
          \end{itemize}
        \item Oberklasse Animal, Unterklassen Säugetier, Tiger, Schlange, Bär, \dots
        \item Brettspiele:
            \begin{itemize}
                \item Klasse Spielbrett; Unterklassen: Schachbrett, Dame-Brett, Mensch-ärgere-dich-nicht
                \item Klasse Spielfigur; Unterklassen: Bauer, Dame, Springer, Turm
                \item Klasse Spiellogik; Unterklassen: DameLogik, SchachLogik
            \end{itemize}
    \end{itemize}
\end{frame}

\framedgraphic{Beispiel}{Klassendiagramm.pdf}

\begin{frame}{Vererbung: Beispiel in Java}
    \begin{minipage}[b]{0.45\linewidth}
        \inputminted[linenos=false, numbersep=5pt, tabsize=4, fontsize=\tiny, label=Animal.java, frame=lines]{java}{Animal.java}
        \inputminted[linenos=false, numbersep=5pt, tabsize=4, fontsize=\tiny, label=Jungle.java, frame=lines]{java}{Jungle.java}
    \end{minipage}
    \hspace{0.5cm}
    \begin{minipage}[b]{0.45\linewidth}
        \inputminted[linenos=false, numbersep=5pt, tabsize=4, fontsize=\tiny, label=Tiger.java, frame=lines]{java}{Tiger.java}
        \inputminted[linenos=false, numbersep=5pt, tabsize=4, fontsize=\tiny, label=Cat.java, frame=lines]{java}{Cat.java}
    \end{minipage}
\end{frame}

\begin{frame}{Allgemeines}
    \begin{block}{\href{http://docs.oracle.com/javase/specs/jls/se7/html/jls-8.html\#jls-8.4.8}{JLS 8.4.8}}
    A class C inherits from its direct superclass and direct
    superinterfaces all abstract and non-abstract methods of the
    superclass and superinterfaces that are public, protected, or
    declared with default access in the same package as C, and are
    neither overridden (§8.4.8.1) nor hidden (§8.4.8.2) by a
    declaration in the class.
    \end{block}
\end{frame}

\section{toString()}
\subsection{Allgemeines}
\begin{frame}{toString()}
    \begin{itemize}[<+->]
        \item Jedes Objekt hat eine Methode \href{http://docs.oracle.com/javase/7/docs/api/java/lang/Object.html\#toString()}{toString()}
        \item Diese wird von \myCode{Object} vererbt
        \item und \st{kann} sollte überschrieben werden
    \end{itemize}
\end{frame}

\begin{frame}{Information aus den Javadoc}
    Wie sollte toString() aussehen?
    \begin{itemize}
        \item Eine kurze textuelle Repräsentation des Objekts
        \item Soll von Menschen gelesen werden
        \item Per Standard: \myCode{getClass().getName() + '@' + Integer.toHexString(hashCode())}
    \end{itemize}
\end{frame}

\begin{frame}{Beispiel}
    \inputminted[linenos=true, numbersep=5pt, tabsize=4, label=Node.java, frame=lines]{java}{Node.java}
\end{frame}

\section{Interfaces}
\subsection{Allgemeines}
\begin{frame}{Interfaces: Allgemeines}
    \begin{itemize}[<+->]
        \item auf Deutsch: Schnittstelle
        \item es werden nur Methodensignaturen vererbt
        \item die Implementierung muss komplett selbst durchgeführt werden!
        \item wird wie Klassen in einer eigenen "`MeinInterface.java"' Datei gespeichert
    \end{itemize}
    \pause[\thebeamerpauses]
    \begin{block}{Namenskonvention}
        Der Name einer Schnittstelle endet oft mit -able.
    \end{block}
\end{frame}

\subsection{Beispiel}
\begin{frame}{Interfaces: Beispiel}
    \inputminted[linenos=true, numbersep=5pt, tabsize=4, fontsize=\small, label=Bicycle.java, frame=lines]{java}{Bicycle.java}
    \inputminted[linenos=true, numbersep=5pt, tabsize=4, fontsize=\small, label=ACMEBicycle.java, frame=lines]{java}{ACMEBicycle.java}
    \small{Quelle: \href{http://docs.oracle.com/javase/tutorial/java/concepts/interface.html}{docs.oracle.com}: What Is an Interface?}\\
    Weitere Informationen: \href{http://docs.oracle.com/javase/tutorial/java/IandI/createinterface.html}{docs.oracle.com}: Interfaces
\end{frame}

\subsection{Real World Examples}
\begin{frame}{Real World Examples}
    \begin{itemize}
        \item \href{http://docs.oracle.com/javase/7/docs/api/java/lang/Comparable.html}{Comperable}: Vergleichen mit \myCode{<}
        \item \href{http://docs.oracle.com/javase/7/docs/api/java/util/List.html}{List}: Viele Listenoperationen
        \item \href{http://docs.oracle.com/javase/7/docs/api/java/lang/Iterable.html}{Iterable}: foreach
        \item \href{http://docs.oracle.com/javase/7/docs/api/java/io/Serializable.html}{Serializable}: Speichern / verschicken übers Netzwerk
        \item \href{http://docs.oracle.com/javase/7/docs/api/java/lang/Runnable.html}{Runnable}: Multithreading
    \end{itemize}
\end{frame}

\subsection{Weitere Informationen}
\begin{frame}{Interfaces: Weitere Informationen}
    \begin{itemize}
        \item \href{http://docs.oracle.com/javase/specs/jls/se7/html/jls-9.html}{JLS 7}
        \item \href{http://openbook.galileodesign.de/javainsel5/javainsel06_010.htm}{Galileo openbook}
    \end{itemize}
\end{frame}

\section{Nachbesprechung ÜB 3}
\subsection{Allgemeines}
\begin{frame}{Allgemeines}
    Lösungen sind \href{https://github.com/MartinThoma/prog-ws1213/tree/master/Blatt-03}{hier} zu finden.
\end{frame}
\begin{frame}{Allgemeines}
    \begin{itemize}[<+->]
        \item \myCode{\href{http://docs.oracle.com/javase/7/docs/api/java/lang/String.html\#charAt(int)}{char charAt(int index)}}:
              Returns the char value at the specified index.
        \item \myCode{\href{http://docs.oracle.com/javase/7/docs/api/java/lang/String.html\#matches(java.lang.String)}{public boolean matches(String regex)}}
              Tells whether or not this string matches the given regular expression.
        \item \myCode{\href{http://docs.oracle.com/javase/7/docs/api/java/lang/String.html\#substring(int, int)}{String substring(int beginIndex,
               int endIndex)}} Returns a new string that is a substring of this string.
    \end{itemize}
    \pause[\thebeamerpauses]
    \begin{block}{Eclipse-Tipp}
        Wenn Eclipse euch im Projektordner einen Fehler anzeigt, aber
        keine Datei fehlerhaft ist, solltet ihr mal einen
        Blick in \menu{Window > Show View > Problem} werfen.
    \end{block}
\end{frame}

\section{Abspann}
\subsection{Kommende Tutorien}
\begin{frame}{Kommende Tutorien}
  \begin{itemize}
    \item[5.] 17.12.2012: Generics?, Video "`Library"' zeigen
    \item[-] 24.12.2012: Heiligabend - \href{http://www.fmc.uni-karlsruhe.de/faq/wann-sind-die-weihnachtsferien}{Kein Tutorium}
    \item[-] 31.12.2012: Silvester - Kein Tutorium
    \item[4.] 07.01.2013
    \item[3.] 14.01.2013
    \item[2.] 21.01.2013
    \item[1.] 28.01.2013: Abschlussprüfunsvorbereitung
    \item[0.] 04.02.2013: Abschlussprüfunsvorbereitung
    \item[-] 10.02.2013: Ende der Vorlesungszeit des WS 2012/2013 (\href{http://www.kit.edu/studieren/2873.php}{Quelle})
  \end{itemize}
\end{frame}

\framedgraphic{Beware of physicist fathers}{../images/Beware-of-physicist-fathers.png}

\end{document}
