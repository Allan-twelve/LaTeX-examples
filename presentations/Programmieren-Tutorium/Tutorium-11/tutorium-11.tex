\documentclass[usepdftitle=false,hyperref={pdfpagelabels=false}]{beamer}
\usepackage{../templates/myStyle}

\begin{document}
\title{\titleText}
\subtitle{Sortieren, equals(), hashCode(), abstrakte Klassen, finale Klassen}
\author{\tutor}
\date{\today}
\subject{Programmieren}

\frame{\titlepage}

\frame{
    \frametitle{Inhaltsverzeichnis}
    \setcounter{tocdepth}{1}
    \tableofcontents
    \setcounter{tocdepth}{2}
}

\section{Einleitung}
\subsection{Quiz}
\begin{frame}{Quiz}
    \inputminted[linenos=true, numbersep=5pt, tabsize=4, fontsize=\tiny]{java}{QuizMain.java}
    \begin{itemize}
        \item Gibt es einen Compiler-Fehler?
        \item Gibt es einen Laufzeit-Fehler?
        \item Gibt es eine Ausgabe? Welche?
    \end{itemize}
\end{frame}

\begin{frame}{Quiz: Antwort}
    \inputminted[linenos=true, numbersep=5pt, tabsize=4, fontsize=\tiny, firstnumber=7, firstline=7, lastline=19, ]{java}{QuizMain.java}
    \begin{block}{Compiler-Fehler}
        {\small
        Exception in thread "main" java.lang.Error: Unresolved compilation problem:\\
        Bound mismatch: The generic method \myCode{sort(List<T>)} of
        type Collections is not applicable for the arguments
        (\myCode{List<List<String$>>$}).\\
        The inferred type \myCode{List<String>} is not a valid substitute for
        the bounded parameter \myCode{<T extends Comparable<? super T$>>$}\\
        at Main.main(Main.java:18)}
    \end{block}
\end{frame}

\subsection{Altes Übungsblatt}
\begin{frame}{Altes Übungsblatt}
   \href{http://stackoverflow.com/q/14200941/562769}{Does it make sense to implement clone(), equals() or hashCode() for an abstract class?}

    \begin{block}{Answer}
        I wouldn't implement clone().

        But it makes sense to implement \myCode{equals()},
        \myCode{hashCode()}, and
        \myCode{toString()} to provide the default behavior for all subclasses.
        Children can choose to use it if they add no new class
        members or supplement as needed.
    \end{block}
\end{frame}

\subsection{Generics}
\begin{frame}{Generics}
   \begin{block}{Übungsleiter}
        generics werden wir für die Abschlussaufgaben vermeiden.
    \end{block}
\end{frame}

\section{equals()}
\subsection{instanceof vs. getClass()}
\begin{frame}{instanceof vs. getClass()}
    \begin{itemize}[<+->]
        \item instanceof akzeptiert auf Untertypen:
              \inputminted[linenos=false, numbersep=5pt, tabsize=4, fontsize=\tiny, firstline=8, lastline=15]{java}{singleLines.java}
        \item getClass nicht:
              \inputminted[linenos=false, numbersep=5pt, tabsize=4, fontsize=\tiny, firstline=17, lastline=18]{java}{singleLines.java}
        \item[$\Rightarrow$] Bei \myCode{equals()} eher \myCode{getClass} verwenden
        \item \href{http://docs.oracle.com/javase/specs/jls/se7/html/jls-15.html\#jls-15.20.2}{instanceof} funktioniert auch mit \myCode{null}
        \item \myCode{null.getClass()} gibt \myCode{NullPointerException} nicht
        \item[$\Rightarrow$] zuerst auf \myCode{null} überprüfen
    \end{itemize}
\end{frame}

\begin{frame}{instanceof vs. getClass()}
    Aber \dots
    \begin{itemize}
        \item \href{http://stackoverflow.com/q/596462/562769}{Sehr viele} ziehen \myCode{instanceof} in \myCode{equals()} der \myCode{getClass()}
              Variante vor
        \item Es gibt Argumente für beides
        \begin{itemize}
            \item pro-instanceof: Debug-Klassen
            \item pro-instanceof: \href{http://en.wikipedia.org/wiki/Liskov_substitution_principle}{Liskov substitution principle}
            \item pro-getClass(): Die Klassen stimmen wirklich überein
        \end{itemize}
        \item Achtung: Andere Semantik!
    \end{itemize}
\end{frame}

\section{Sortieren}
\subsection{Was kann man sortieren?}
\begin{frame}{Was kann man sortieren?}
    \begin{itemize}
        \item Zahlen
        \item Wörter
        \item Länder nach Anzahl der Einwohner
        \item Spielkarten
        \item \dots
    \end{itemize}
\end{frame}

\subsection{Was braucht man?}
\begin{frame}{Was braucht man?}
    Totale Ordnungsrelation $\preceq$ auf einer Menge $C$:
    \begin{itemize}
        \item Totalität: $\forall x, y \in C: x \preceq y \lor y \preceq x$
        \item Antisymmetrie: $\forall x,y \in C: x \preceq y \land y \preceq x \Rightarrow x = y$
        \item Transitivität: $\forall x,y,z \in C: x \preceq y \land y \preceq z \Rightarrow x \preceq z$
    \end{itemize}
\end{frame}

\begin{frame}{Wo ist das nicht gegeben?}
    \begin{itemize}
        \visible<1->{\item Totalität: $\forall x, y \in C: x \preceq y \lor y \preceq x$?}
        \visible<2->{\item[$\Rightarrow$] Menge $\mathbb{C}$, Relation $\leq$: $i$ und $1$ stehen nicht in Relation!}
        \visible<2->{\item[$\Rightarrow$] Menge $\mathcal{P}(\Set{1,2,3})$, Relation $\subseteq$: $\Set{1}$ und $\Set{2}$ stehen nicht in Relation!}
        \visible<3->{\item Antisymmetrie: $\forall x,y \in C: x \preceq y \land y \preceq x \Rightarrow x = y$?}
        \visible<4->{\item[$\Rightarrow$] Menge $\mathbb{R}$, Relation $\preceq: x \preceq y \Leftrightarrow x,y \in \mathbb{R}$ (vgl. \href{http://math.stackexchange.com/q/276907/6876}{SO})}
        \visible<5->{\item Transitivität: $\forall x,y,z \in C: x \preceq y \land y \preceq z \Rightarrow x \preceq z$?}
        \visible<6->{\item[$\Rightarrow$] ?}
    \end{itemize}
\end{frame}

\begin{frame}{Hilfe, ich komme mit Relationen nicht zurecht!}
    Don't Panic!

    \begin{itemize}
        \item Meist vergleicht man indirekt Zahlen
        \item[$\rightarrow$] Bei \myCode{double} und \myCode{float} den Epsilon-Vergleich machen!
        \item Sonst vergleicht man Strings
        \item[$\rightarrow$] \myCode{myString.compareTo(myOtherString)}
        \item Die JavaDoc von \href{http://docs.oracle.com/javase/7/docs/api/java/lang/Comparable.html\#compareTo\%28T\%29}{compareTo(other)}
              sind weniger mathematisch formuliert
    \end{itemize}
\end{frame}

\subsection{Wie sortiert man?}
\begin{frame}{Wie sortiert man?}
    Vergleichsbasierte Sortieralgorithmen:
    \begin{itemize}
        \item Selectionsort
        \item Bubblesort
        \item Quicksort
        \item \dots
    \end{itemize}

    Nicht vergleichsbasierte Algorithmen:
    \begin{itemize}
        \item Radixsort
        \item Countingsort
    \end{itemize}

    Implementierungen und Vergleiche dieser und weiterer Algorithmen sind
    \href{http://martin-thoma.com/ubersicht-uber-sortieralgorithmen/}{hier}
    zu finden.

    \begin{block}{Info am Rande}
        Collections.sort() verwendet Mergesort-Variante (vermutlich Timsort)
    \end{block}
\end{frame}

\subsection{Sortieren in Java}
\begin{frame}{Sortieren in Java: Arrays}
    \inputminted[linenos=true, numbersep=5pt, tabsize=4, fontsize=\small]{java}{Main-Arrays-sort.java}
\end{frame}

\begin{frame}{Sortieren in Java: Collections}
    \inputminted[linenos=true, numbersep=5pt, tabsize=4, fontsize=\small]{java}{Main-Collection-sort.java}
\end{frame}

\begin{frame}{Sortieren in Java: Comparator}
    \inputminted[linenos=true, numbersep=5pt, tabsize=4, fontsize=\small]{java}{PopulationDensityComperator.java}
\end{frame}

\begin{frame}{Sortieren in Java: Comparator benutzen}
    \inputminted[linenos=true, numbersep=5pt, tabsize=4, fontsize=\tiny]{java}{ComparatorMain.java}
\end{frame}

\section{hashCode()}
\subsection{Allgemeines}
\begin{frame}{Vier Gewinnt}
    \begin{block}{Frage}
        Wie viele Situationen gibt es auf einem $7 \times 6$-Feld
        bei "`4 Gewinnt"'?
    \end{block}

    \begin{itemize}[<+->]
        \item maximal: $3^{7 \cdot 6} = 3^{42} = 109418989131512359209 \approx 109 \cdot 10^{18}$
        \item minimal: schwer zu sagen
        \item Idee: Brute-Force
        \begin{itemize}
            \item Alle möglichen Spielentscheidungen durchgehen
            \item Kommt man auf eine bereits bekannte Situation, ist es keine neue
            \item Man muss also alte Situationen speichern (z.B. ein \myCode{char[42]} pro Spielsituation)
            \item Man muss eine alte Situationen finden können
            \item Vermutung: min. $20\,000\,000$ Spielsituationen
            \item[$\Rightarrow$] lineare Suche nach bekannten Situationen dauert zu lange
        \end{itemize}
    \end{itemize}
\end{frame}

\begin{frame}{Vier Gewinnt: Hashing}
    \begin{block}{Frage}
        Wie kann ich schnell eine Spielsituation speichern und wieder
        finden?
    \end{block}

    Antwort: Hash-Funktion mit linearer Sondierung!

    \pause

    \begin{itemize}[<+->]
        \item Ich will eine Funktion: $h: \text{Spielsituationen} \rightarrow \text{Array-Index}$
        \item Die Spiel-Situation kann ich als Zahl $x$ auffassen mit $0 \leq x \leq 110 \cdot 10^{18}$
        \item Für den Array-Index $i$ gilt: $0 \leq i \le 20\,000\,000$
        \item $h$ ist also nicht injektiv
        \item Sobald der Array voll ist, können wir aufhören
        \item Falls $h(x)$ ein Array-Index ist, der bereits belegt ist, aber der Array nicht voll ist, müssen wir
              die nächste freie Stelle suchen.\\
              Dazu gehen wir einfach auf $(h(x) + 1) \% 20\,000\,000$
        \item[$\Rightarrow$] wird "`lineares Sondieren genannt"'
    \end{itemize}
\end{frame}

\begin{frame}{Vier Gewinnt: hash-Funktion}
    \begin{block}{Frage}
        Wie sieht eine gute hash-Funktion aus?
    \end{block}

    \begin{itemize}[<+->]
        \item Sie sollte surjektiv sein
        \item Sie sollte gleichmäßig auf die Bildmenge abbilden
        \item Vorschlag: \inputminted[linenos=true, numbersep=5pt, tabsize=4, fontsize=\tiny]{c}{vierGewinnt.c}
        \item Beispiel-Code ist auf \href{https://github.com/MartinThoma/connect-four/blob/master/C/connectfour.c}{GitHub}
    \end{itemize}
\end{frame}

\begin{frame}{Hashing: Allgemein}
    \begin{block}{Frage}
        Was ist nun eine Hash-Funktion im Allgemeinen?
    \end{block}

    \begin{block}{Antwort}
        Eine Hash-Funktion $h$ bildet von einem sehr großem
        Definitionsbereich auf einen deutlich kleineren Wertebereich
        ab.
    \end{block}

    \pause

    \begin{itemize}[<+->]
        \item Meist ist der Wertebereich ein \myCode{int}, also $[-2147483648, 2147483647] = [-2^{31},2^{31}-1] \approx [-2 \cdot 10^9, 2 \cdot 10^9]$
        \item Der Definitionsbereich kann alles mögliche sein
        \item Normalerweise ist es nicht möglich, eine injektive Funktion zu finden
    \end{itemize}
\end{frame}

\subsection{hashCode() in Java}
\begin{frame}{hashCode() in Java}
    \begin{block}{Signatur}
        \myCode{public int hashCode()}
    \end{block}

    \pause

    \begin{block}{Bedingung 1}
        Whenever it is invoked on the same object more than once
        during an execution of a Java application, the hashCode
        method \textbf{must consistently return the same integer},
        provided no information used in equals comparisons on the
        object is modified. This integer need not remain consistent
        from one execution of an application to another execution of
        the same application.
    \end{block}
    {\tiny source: \href{http://docs.oracle.com/javase/7/docs/api/java/lang/Object.html\#hashCode()}{JavaDoc}}
\end{frame}

\begin{frame}{hashCode() in Java}
    \begin{block}{Bedingung 2}
        If two objects are equal according to the equals(Object)
        method, then calling the hashCode method on each of the two
        objects \textbf{must} produce the same integer result.
    \end{block}
    {\tiny source: \href{http://docs.oracle.com/javase/7/docs/api/java/lang/Object.html\#hashCode()}{JavaDoc}}

    \pause

    \begin{block}{Klarstellung 1}
        Es muss gelten:\\
        $A.equals(B) \Rightarrow A.hashCode() == B.hashCode()$\\
        Aber nicht:\\
        $A.hashCode() == B.hashCode() \Rightarrow A.equals(B)$\\
        Das ist meist auch nicht möglich. Beispiel:\\
        Eine Klasse mit einem \myCode{double} als Rückgabewert.
    \end{block}
\end{frame}

\begin{frame}{hashCode() in Java}
    \begin{block}{Klarstellung 2}
        It is \textbf{not required} that if two objects are unequal according
        to the equals(java.lang.Object) method, then calling the
        hashCode method on each of the two objects must produce
        distinct integer results.\\
        However, the programmer should be
        aware that producing distinct integer results for unequal
        objects may improve the performance of hash tables.
    \end{block}
    {\tiny source: \href{http://docs.oracle.com/javase/7/docs/api/java/lang/Object.html\#hashCode()}{JavaDoc}}
\end{frame}

\begin{frame}{hashCode(): Quiz}
    Ist das eine korrekte Hash-Funktion?
    \inputminted[linenos=true, numbersep=5pt, tabsize=4, fontsize=\tiny]{java}{Country-hashCode.java}
\end{frame}

\begin{frame}{hashCode(): Antwort}
    \begin{itemize}[<+->]
        \item Ja, ist nach \href{http://docs.oracle.com/javase/7/docs/api/java/lang/Object.html\#hashCode()}{JavaDoc} eine korrekte
              Hash-Funktion
        \item Aber: Es ist die wohl schlechteste korrekte Hash-Funktion
        \item Sogar praktisch nutzlos
        \item \alert<5>{NIEMALS} so machen!
    \end{itemize}
\end{frame}

\begin{frame}{hashCode(): Set}
    \href{http://docs.oracle.com/javase/7/docs/api/java/lang/Object.html\#equals(java.lang.Object)}{hashCode()} und \href{http://docs.oracle.com/javase/7/docs/api/java/lang/Object.html\#equals(java.lang.Object)}{equals()} sind für \href{http://docs.oracle.com/javase/7/docs/api/java/util/Set.html}{Set} wichtig.
\end{frame}

\section{Interface}
\subsection{Allgemeines}
\begin{frame}{Allgemeines}
    War Thema in Tutorium Nr. 8
\end{frame}

\section{abstract}
\subsection{Allgemeines}
\begin{frame}{Allgemeines}
    \begin{block}{docs.oracle.com Tutorial}
        Eine abstrakte Klasse ist eine Klasse mit dem \myCode{abstract}-Schlüsselwort.
        Sie kann abstrakte Methoden beinhalten.\\
        Abstrakte Klassen können nicht instanziiert werden, aber man kann
        Unterklassen bilden.
    \end{block}

    Abstrakte Klassen
    \begin{itemize}[<+->]
        \item \dots müssen keine abstrakten Methoden beinhalten\\
              {\tiny Quelle: \href{http://stackoverflow.com/q/4811678/562769}{Defining an abstract class without any abstract methods}}
        \item \dots sollten eine abstrakte Methode beinhalten\\
              {\tiny Quelle: \href{http://stackoverflow.com/q/2283399/562769}{Should an abstract class have at least one abstract method?}}
        \item \dots können Konstruktoren haben\\
              {\tiny Quelle: \href{http://stackoverflow.com/q/7644342/562769}{Abstract class is using it's own abstract method?}}
        \item \dots können konkret impementierte Methoden haben\\
              {\tiny Quelle: \href{http://answers.yahoo.com/question/index?qid=20111207141904AABXAvh}{Can an abstract class have concrete(non-abstract method) methods?}}
    \end{itemize}
\end{frame}

\begin{frame}{Beispiel}
    \begin{block}{What are practical examples of abstract classes in java?}
        \begin{itemize}
            \item \href{http://stackoverflow.com/a/1509826/562769}{StackOverflow}
            \item FileParser, CameraFileParser
        \end{itemize}
    \end{block}
\end{frame}

\subsection{Abstract Classes vs Interfaces}
\begin{frame}{Abstract Classes vs Interfaces}
    Abstrakte Klassen können \dots
    \begin{itemize}
        \item Attribute haben, die nicht \myCode{static} und \myCode{final} sind
        \item Implementierte Methoden haben
    \end{itemize}

    \begin{block}{Wenn nutze ich Interfaces?}
        Wenn ich nur abstrakte Methoden habe
    \end{block}
\end{frame}



\subsection{Literatur}
\begin{frame}{Literatur}
    \begin{itemize}
        \item docs.oracle.com Tutorial: \href{http://docs.oracle.com/javase/tutorial/java/IandI/abstract.html}{Abstract Methods and Classes}
        \item codestyle.org: \href{http://www.codestyle.org/java/faq-Abstract.shtml}{Java abstract classes FAQ}
        \item openbook.galileodesign.de: \href{http://openbook.galileodesign.de/javainsel5/javainsel06_009.htm\#Rxx747java06009040001EA1F04F100}{Abstrakte Klassen und abstrakte Methoden}
    \end{itemize}
\end{frame}

\section{final}
\subsection{Allgemeines}
\begin{frame}{Allgemeines}
    \begin{itemize}
        \item Finale Klassen können keine Unterklassen haben
        \item Beispiel: \myCode{String}, \myCode{StringBuffer}, \myCode{StringBuilder}, \myCode{Math}:
        \inputminted[linenos=true, numbersep=5pt, tabsize=4, fontsize=\tiny, firstnumber=1, firstline=1, lastline=6]{java}{singleLines.java}
    \end{itemize}
\end{frame}

\subsection{Literatur}
\begin{frame}{Literatur}
    \begin{itemize}
        \item docs.oracle.com Tutorial: \href{http://docs.oracle.com/javase/tutorial/java/IandI/final.html}{Writing Final Classes and Methods}
        \item openbook.galileodesign.de: \href{http://openbook.galileodesign.de/javainsel5/javainsel06_006.htm\#Rxx747java06006040001E71F019239}{Finale Klassen}
    \end{itemize}
\end{frame}

%\section{Tipp}
%\subsection{Java API-Code anschauen}
%\begin{frame}{Java API-Code anschauen}
%    \begin{itemize}
%        \item In Eclipse gehen
%        \item Eine Funktion der Java API-Klasse, die ihr nutzen wollt, hinschreiben
%        \item \menukeys{Strg} + Rechtsklick auf die Funktion
%    \end{itemize}
%\end{frame}


\section{Abspann}
\subsection{Klausuranmeldung}
\begin{frame}{Klausuranmeldung}
    \begin{itemize}
        \item Anmeldebeginn: 28.1.
        \item Anmeldeschluss / Abmeldeschluss: 28.2.
        \item Ausgabetermin für Teil 1: 28.1.
        \item Ausgabetermin für Teil 2: 11.2.
        \item Abgabefrist für Teil 1: 11.3.
        \item Abgabefrist für Teil 2: 25.3.
    \end{itemize}
\end{frame}

\subsection{Kommende Tutorien}
\begin{frame}{Kommende Tutorien}
  \begin{itemize}
    \item[3.] 14.01.2013
    \item[2.] 21.01.2013
    \item[1.] 28.01.2013: Abschlussprüfunsvorbereitung
    \item[-] 28.01.2013: Ausgabetermin für Teil 1
    \item[0.] 04.02.2013: Abschlussprüfunsvorbereitung
    \item[-] 10.02.2013: Ende der Vorlesungszeit des WS 2012/2013 (\href{http://www.kit.edu/studieren/2873.php}{Quelle})
  \end{itemize}
\end{frame}

\framedgraphic{Vielen Dank für eure Aufmerksamkeit!}{../images/geekandpoke-user-too-dumb.jpg}

\end{document}
