\documentclass[usepdftitle=false,hyperref={pdfpagelabels=false}]{beamer}

% use KIT-Theme
% see http://sdqweb.ipd.kit.edu/wiki/Dokumentvorlagen
%\usetheme{Frankfurt} % see http://deic.uab.es/~iblanes/beamer_gallery/index_by_theme.html as fallback
\InputIfFileExists{../templates/beamerthemekit.sty}{\usepackage{../templates/beamerthemekit}}{\usetheme{Frankfurt}}
\usefonttheme{professionalfonts}

\usepackage{hyperref}
\usepackage{lmodern}
\usepackage{listings}
\usepackage{wrapfig}        % see http://en.wikibooks.org/wiki/LaTeX/Floats,_Figures_and_Captions
\usepackage[utf8]{inputenc} % this is needed for german umlauts
\usepackage[ngerman]{babel} % this is needed for german umlauts
\usepackage[T1]{fontenc}    % this is needed for correct output of umlauts in pdf
\usepackage{verbatim}
\usepackage{relsize}
\usepackage{subfigure}
\usepackage{algorithm,algpseudocode}
\usepackage{minted}         % needed for the inclusion of source code
\usepackage{xcolor}
\usepackage{menukeys}
\usepackage{wasysym}
\usepackage{braket} % needed for \Set
\usepackage{tikz}
% w/o following space!
\newcommand{\quem}{\tikz[baseline=(wi.base)]{\node[fill=black,rotate=45,inner sep=1.5mm] {};%
\node[ font=\color{white}] (wi) {?};}}
\usepackage{../templates/myStyle}

\newcommand\tutor{Martin Thoma}
\newcommand\tutNR{10}
\newcommand\titleText{Programmieren-Tutorium Nr. \tutNR{} bei \tutor}
\institute{Fakultät für Informatik}

\hypersetup{pdftitle={\titleText}}
\beamertemplatenavigationsymbolsempty

\newcommand\InsertToC[1][]{
  \begin{frame}{Outline}
    \tableofcontents[subsectionstyle=show/show/show, subsubsectionstyle=show/show/show, #1]
  \end{frame}
}

\begin{document}
\title{\titleText}
\subtitle{Javadoc, Bytecode}
\author{\tutor}
\date{\today}
\subject{Programmieren}

\frame{\titlepage}

\frame{
    \frametitle{Inhaltsverzeichnis}
    \setcounter{tocdepth}{1}
    \tableofcontents
    \setcounter{tocdepth}{2}
}

\section{Einleitung}
\subsection{Quiz: Teil 1}
\begin{frame}{Quiz: Teil 1}
    \inputminted[linenos=true, numbersep=5pt, tabsize=4, fontsize=\small, label=Quiz.java, frame=lines]{java}{Quiz-float-1.java}
    \begin{itemize}
        \item Gibt es einen Compiler-Fehler?
        \item Gibt es einen Laufzeit-Fehler?
        \item Gibt es eine Ausgabe? Welche?
    \end{itemize}
\end{frame}

\subsection{Quiz: Teil 2}
\begin{frame}{Quiz: Teil 2}
    \inputminted[linenos=true, numbersep=5pt, tabsize=4, fontsize=\small, label=Quiz.java, frame=lines]{java}{Quiz-float-2.java}
    \begin{itemize}
        \item Gibt es einen Compiler-Fehler?
        \item Gibt es einen Laufzeit-Fehler?
        \item Gibt es eine Ausgabe? Welche?
    \end{itemize}
\end{frame}

\subsection{Quiz: Teil 3}
\begin{frame}{Quiz: Teil 3}
    \inputminted[linenos=true, numbersep=5pt, tabsize=4, fontsize=\small, label=Quiz.java, frame=lines]{java}{Quiz-float-3.java}
    \begin{itemize}
        \item Gibt es einen Compiler-Fehler?
        \item Gibt es einen Laufzeit-Fehler?
        \item Gibt es eine Ausgabe? Welche?
    \end{itemize}
\end{frame}

\section{Nachbesprechung: 1. ÜB}
\subsection{Gleitkomma vs. Ganzzahl}
\begin{frame}{Gleitkomma vs. Ganzzahl}
    Welche Vorteile bieten Ganzzahl-Variablen im Vergleich zu
    Gleitkomma-Variablen?
    \begin{itemize}[<+->]
        \item \textbf{Speicherplatz}? \onslide<4->{Nein, vgl. \myCode{long} und \myCode{float}}
        \item \textbf{Geschwindigkeit}? \onslide<5->{Kommt drauf an: Wenn keine weitere Umrechnung nötig ist und die Gleitkommazahl nicht kleiner ist, eher ja.}
        \item \textbf{Genauigkeit}? \onslide<6->{Ja.\\
        Beispiel: $(0,1)_{10} = (0,0\overline{0011})_2$\\
        vgl. Java-Puzzle}
    \end{itemize}
\end{frame}

\subsection{Variablennamen}
\begin{frame}{Variablennamen}
    Bitte nur alphanumerische Zeichen für Namen (Pakete, Methoden, Variablen, \dots).\\
    Soll heißen:
    \begin{itemize}[<+->]
        \item abcdefghijklmnopqrstuvwxyz sind erlaubt
        \item ABCDEFGHIJKLMNOPQRSTUVWXYZ sind erlaubt
        \item 0123456789 sind erlaubt
        \item Unterstrich - also "`\_"' ist nur in Kostanten erlaubt
        \item Der Rest ist verboten und gibt Punktabzug! Keine Umlaute!
    \end{itemize}
\end{frame}

\subsection{Kommentare}
\begin{frame}{Kommentare}
    \begin{block}{Regel}
        Kommentiert, was ihr macht. \\
        Nicht wie ihr es macht.
    \end{block}

    \begin{itemize}[<+->]
        \item Gut zu kommentieren ist schwer
        \item Viel (fremden) Code ansehen hilft
        \item Eigenen Code nach Jahren ansehen hilft
        \item JEDER Kommentar ist für Java-Entwickler gedacht
        \item[$\Rightarrow$] Kommentare à la "`Methode"' oder
              "`Methodensignatur"' sind nutzlos!\\
        \item Javadoc-Kommentare sind für Entwickler, die nichts von
              eurem Code sehen können, ihn aber dennoch nutzen wollen
    \end{itemize}
\end{frame}

\subsection{Kommentare: Beispiel 1}
\begin{frame}{Kommentare: Beispiel 1}
    \inputminted[linenos=true, numbersep=5pt, tabsize=4, fontsize=\tiny]{java}{Kommentare-1.java}

    \only<2->{
        \begin{itemize}
            \item An sich gute Kommentare
            \item Wäre besser als Javadoc direkt über der Methode
        \end{itemize}
    }
\end{frame}

\subsection{Kommentare: Beispiel 2}
\begin{frame}{Kommentare: Beispiel 2}
    \inputminted[linenos=true, numbersep=5pt, tabsize=4, fontsize=\small]{java}{Kommentare-2.java}
\end{frame}

\begin{frame}{Kommentare: Beispiel 2}
    \begin{itemize}
        \item Zeile 1 \& 2 bieten einem Entwickler nicht mehr Informationen
              $\Rightarrow$ nutzlos
        \item Sonst okay
        \item Bitte nicht \myCode{chainwheel0}, sondern \myCode{chainwheel}
              und später \myCode{this}-Operator nutzen\\
              Gibt in Zukunft -0,5 Punkte
        \item Gute Zeilenlänge \smiley
    \end{itemize}
\end{frame}

\subsection{Kommentare: Beispiel 3}
\begin{frame}{Kommentare: Beispiel 3}
    \inputminted[linenos=true, numbersep=5pt, tabsize=4, fontsize=\tiny]{java}{Kommentare-3.java}
    \only<2->{
        \begin{itemize}
            \item Javadoc ist okay
            \item "`Methodensignatur"'-Kommentar in Z. 5 ist nutzlos
            \item Kommentare in Z. 7 - 9 sind nutzlos
            \item Z. 8 ist arg lang $\rightarrow$ den Kommentar hätte
                  man einfach über die Zeile schreiben können.
        \end{itemize}
    }
\end{frame}

\subsection{Namenskonventionen für booleans}
\begin{frame}{Namenskonventionen für booleans}
    \begin{block}{Regel}
        Der Präfix "`is"' sollte für boolesche Variablen und Methoden
        mit dem Rückgabewert \myCode{boolean} genutzt werden.
    \end{block}

    \onslide<2->{
        \begin{exampleblock}{Beispiele}
            isSet, isVisible, isFinished, isFound, isOpen
        \end{exampleblock}
    }

    \onslide<3->{
        Auch okay sind "`has"', "`should"' oder ähnliche Präfixe.
    }

    \onslide<4->{
        \begin{exampleblock}{Beispiele}
            \inputminted[linenos=false, numbersep=5pt, tabsize=4, fontsize=\small, firstline=1, lastline=3]{java}{singleLines.java}
        \end{exampleblock}
    }
\end{frame}

\subsection{booleans: Positiv- und Negativbeispiel}
\begin{frame}{booleans: Positiv- und Negativbeispiel}
    \begin{exampleblock}{Negativbeispiel: So nicht!}
            \inputminted[linenos=false, numbersep=5pt, tabsize=4, fontsize=\small, firstline=5, lastline=6]{java}{singleLines.java}
    \end{exampleblock}

    \onslide<2->{
        \begin{exampleblock}{Positivbeispiel: Aber so}
                \inputminted[linenos=false, numbersep=5pt, tabsize=4, fontsize=\small, firstline=8, lastline=9]{java}{singleLines.java}
        \end{exampleblock}
    }

    \onslide<3->{
        In Zukunft: -0,5 Punkte
    }
\end{frame}

\subsection{Boolean: Was ist mit Gettern/Settern?}
\begin{frame}{Boolean: Was ist mit Gettern/Settern?}
    \inputminted[linenos=false, numbersep=5pt, tabsize=4, firstline=11, lastline=17]{java}{singleLines.java}
    \begin{block}{Hinweis}
    Es ist okay, wenn ein Attribut genauso heißt wie eine Methode
    \end{block}
\end{frame}

\subsection{Kombinatorik}
\begin{frame}{Kombinatorik}
    Wenn man 3 Gänge vorne und 7 hinten hat, wie viele Gänge gibt es?
    \vspace{1.0cm}
    \onslide<2->{
        \textbf{Antwort:} $3 \cdot 7 = 21$\\
    }
    \vspace{0.5cm}
    \onslide<3->{\textbf{Erklärung}:
        Sei $\Set{a, b, c}$ die Menge der vorderen Gänge und
        $\Set{1, 2, 3, 4, 5, 6, 7}$ die Menge der hinteren Gänge.\\
    }
    \vspace{0.2cm}
    \onslide<4->{
        Dann gibt es folgende Kombinationen:\\
        a1, a2, a3, a4, a5, a6, a7\\
        b1, b2, b3, b4, b5, b6, b7\\
        c1, c2, c3, c4, c5, c6, c7
    }

\end{frame}

\subsection{Formatierung}
\begin{frame}{Formatierung}
    Mit Eclipse:
    \begin{itemize}
        \item Alles markieren: \keys{\ctrl+A}
        \item Formatieren: \keys{\ctrl+\shift+F}
    \end{itemize}
    Falsche Formatierung gibt in Zukunft pro Fehler -0,5 Punkte.\\
    Auch Folgefehler geben Punktabzug!
\end{frame}

\subsection{Aussagekräftige Variablen!}
\begin{frame}{Aussagekräftige Variablen!}
    In Zukunft:
        -1 P. bis -5 P. für Variablennamen wie "`Kr"' für Kettenräder oder
        "`Pr"' für Preis!
\end{frame}

\subsection{Quiz}
\begin{frame}{Quiz}
    \begin{table}[ht]
    \begin{tabular}{p{0.4\linewidth}l}
    \begin{minipage}[b]{0.8\linewidth}\centering
    \inputminted[linenos=true, numbersep=5pt, tabsize=4, fontsize=\tiny, label=Baby.java, frame=lines]{java}{Static-Quiz-Baby.java}
    \end{minipage}
    &
    \begin{minipage}[b]{0.5\linewidth}
    \centering
    \inputminted[linenos=true, numbersep=5pt, tabsize=4, fontsize=\tiny, label=World.java, frame=lines]{java}{Static-Quiz-World.java}
    \end{minipage}
    \end{tabular}
    \end{table}
    \begin{itemize}
        \item Gibt es einen Compiler-Fehler?
        \item Gibt es einen Laufzeit-Fehler?
        \item Gibt es eine Ausgabe? Welche?
    \end{itemize}
\end{frame}

\subsection{Quiz: Antwort}
\begin{frame}{Quiz: Antwort}
    Ausgabe:
    \begin{itemize}
      \item Alice: 56
      \item Bob: 56
    \end{itemize}
    \vspace{0.5cm}
    Warum?
    \begin{itemize}[<+->]
      \item \myCode{static} macht ein Attribut zu einem "`\textbf{Klassenattribut}"'
      \item Das Attribut gehört dann nicht mehr den einzelnen Objekten
      \item Und sollte auch nicht über Objecte aufgerufen werden!
      \item Schlecht: \myCode{alice.size;}
      \item Auch schlecht: \myCode{alice.getSize();}
      \item Besser: \myCode{Baby.size;} oder \myCode{Baby.getSize();}
      \item In Zukunft: min. -2 P. für falsche \myCode{static}-Verwendung
    \end{itemize}
\end{frame}

\section{Exkurs}
\subsection{Exkurs: Bytecode}
\begin{frame}{Exkurs: Bytecode}
    \begin{alertblock}{Hinweis}
        Das folgende ist nicht Prüfungsrelevant!\\
        Also zurücklehnen und genießen :-)
    \end{alertblock}

    \vspace{1cm}
    Mit dem Befehl
    \inputminted[linenos=false, numbersep=5pt, tabsize=4, firstline=1, lastline=1]{bash}{bash.sh}
    könnt ihr euch den Java-Bytecode ansehen.
\end{frame}

\begin{frame}{Exkurs: SimpleLoop Java Code}
    \inputminted[linenos=true, numbersep=5pt, tabsize=4, label=SimpleLoop.java, frame=lines]{java}{SimpleLoop.java}
\end{frame}

\begin{frame}{Bytecode von SimpleLoop.java}
    \inputminted[linenos=false, numbersep=5pt, tabsize=4, fontsize=\tiny]{java}{Bytecode.txt}
\end{frame}
\begin{frame}{Bytecode von SimpleLoop.java: Anfang}
    \inputminted[linenos=false, numbersep=5pt, tabsize=4, firstline=1, lastline=7, fontsize=\tiny]{java}{Bytecode.txt}
    \begin{itemize}
        \item \textbf{aload\_0}: Lade eine Objektreferenz aus dem Array der lokalen Variablen auf den Operandenstapel. (\href{http://www.javaseiten.de/jvmisa.html\#aload\_0}{Quelle})
        \item \textbf{iload\_1}: Lade den int-Wert einer lokalen Variablen auf den Operandenstapel. (\href{http://www.javaseiten.de/jvmisi.html\#iload_1}{Quelle})
        \item \textbf{invokespecial [method-spec]}: invoke method belonging to a specific class (\href{https://www.vmth.ucdavis.edu/incoming/Jasmin/ref--33.html}{Quelle})
    \end{itemize}
\end{frame}

\begin{frame}{Bytecode von SimpleLoop.java: Ende}
    \inputminted[linenos=false, numbersep=5pt, tabsize=4, firstline=9, lastline=32, fontsize=\tiny]{java}{Bytecode.txt}
\end{frame}

\begin{frame}{Interessanter Teil des Bytecodes}
    \inputminted[linenos=false, numbersep=5pt, tabsize=4, fontsize=\tiny]{java}{Bytecode-interessant.txt}
\end{frame}

\subsection{Javadoc}
\framedgraphic{Offizielle \href{http://docs.oracle.com/javase/6/docs/api/}{Java 6 API} Javadoc}{java-6-javadoc.png}
\framedgraphic{Offizielle \href{http://docs.oracle.com/javase/7/docs/api/}{Java 7 API} Javadoc}{java-7-javadoc.png}

\subsection{Javadoc erzeugen}
\begin{frame}{Javadoc erzeugen}
    \begin{itemize}
        \item Order erstellen, in dem die Javadoc landen soll
        \item In den Ordner mit euren Quelldateien wechseln
        \item Befehl \myCode{javadoc -d ../pfad/zum/javadoc/ordner/ *}
    \end{itemize}
\end{frame}
\begin{frame}{Javadoc erzeugen}
    \inputminted[linenos=true, numbersep=5pt, tabsize=4, firstline=1, lastline=30, fontsize=\tiny]{console}{javadoc.sh-session}
\end{frame}
\begin{frame}{Javadoc erzeugen}
    \inputminted[linenos=true, numbersep=5pt, tabsize=4, firstline=31, firstnumber=31, lastline=59, fontsize=\tiny]{console}{javadoc.sh-session}
\end{frame}
\begin{frame}{Javadoc erzeugen}
    \inputminted[linenos=true, numbersep=5pt, tabsize=4, firstline=60, firstnumber=60, lastline=82, fontsize=\tiny]{console}{javadoc.sh-session}
\end{frame}
\begin{frame}{Javadoc: Codierung}
    \begin{itemize}[<+->]
        \item \quem{} ist ein sicheres Zeichen, dass was bei der
              Zeichenkodierung schief ging.
        \item Bitte verwendet \textbf{immer} UTF-8!
        \item Eclipse: \menu{Window > Preferences > General > Workspace > Text file encoding > UTF-8}
    \end{itemize}
\end{frame}

\section{Dies und Das}
\subsection{Konvention: Leerzeichen}
\begin{frame}{Konvention: Leerzeichen}
Keine Whitespaces nach
    \begin{itemize}
        \item ~ (Bitweises Komplement)
        \item ! (Logisches Komplement)
        \item ++ (Präfix-Inkrementierung, z.B. ++i;)
        \item -- (Präfix-Dekrementierung, z.B. --i;)
        \item . (Punkt)
        \item - (Unäres Minus, z.B. -5)
        \item + (Unäres Plus, z.B. +4)
    \end{itemize}
    Und
    \begin{itemize}
        \item Exakt eines vor und nach "`="'
        \item Um Operatoren herum:
    \end{itemize}

    \inputminted[linenos=false, numbersep=5pt, tabsize=4, fontsize=\small, firstline=20, lastline=24]{java}{singleLines.java}
\end{frame}

\subsection{Mehrdimensionale Arrays}
\begin{frame}{Mehrdimensionale Arrays}
    Vorstellung:
    \begin{itemize}
        \item 1D: Vektor, Liste
        \item 2D: Matrix, Tabelle
        \item 3D: Quader
        \item 4D: Hyperwürfel (falls quadratisch)
    \end{itemize}

    \inputminted[linenos=false, numbersep=5pt, tabsize=4, fontsize=\small, firstline=26, lastline=33]{java}{singleLines.java}
\end{frame}

\subsection{Wrap-Up: Was ihr können solltet}
\begin{frame}{Wrap-Up: Das Programmierer $1 \cdot 1$}
    \textbf{Was ihr können solltet}:
    \begin{itemize}
        \item \textbf{Einfache Probleme modellieren}:\\
              Welche Klassen / Methoden brauche ich?
        \item \textbf{Konstrollstrukturen}:
          \begin{itemize}
            \item \myCode{if (<Bedingung>) \{ \dots \}}
            \item \myCode{while (<Bedingung>) \{ \dots \}}
            \item \myCode{for (<Initialisierung>, <Bedingung>, <Update>) \{\dots\}}
            \item \myCode{switch (<Variable>) \{case <Wert>:\}}
          \end{itemize}
        \item \textbf{Arrays}: 1D- und 2D
        \item \textbf{Kommentare}
        \item \textbf{Koventionen}: Javadoc, Leerzeichen-Setzung
        \item \textbf{Debuggen}: Einfache Fehler in eurem Code finden
    \end{itemize}

    \textbf{Was ihr hier noch lernt}:
    \begin{itemize}
        \item Verwendung der Java Standardbibliothek
        \item Eingabe von Daten
    \end{itemize}
\end{frame}

\subsection{Quiz: For-Schleifen}
\begin{frame}{Quiz: For-Schleifen (1/2)}
    \inputminted[linenos=true, numbersep=5pt, tabsize=4, fontsize=\small, label=QuizFor.java, frame=lines]{java}{QuizFor.java}
    \begin{itemize}
        \item Gibt es einen Compiler-Fehler?
        \item Gibt es einen Laufzeit-Fehler?
        \item Gibt es eine Ausgabe? Welche?
    \end{itemize}
\end{frame}

\begin{frame}{Quiz: For-Schleifen (1/2) - Antwort}
    \inputminted[linenos=true, numbersep=5pt, tabsize=4, fontsize=\small, label=QuizFor.java, frame=lines]{java}{QuizFor.java}
    Ausgabe: \myCode{end}, da die Bedingung auch am Anfang überprüft wird
\end{frame}

\begin{frame}{Quiz: For-Schleifen (2/2)}
    \inputminted[linenos=true, numbersep=5pt, tabsize=4, fontsize=\small, label=QuizFor.java, frame=lines]{java}{QuizFor-2.java}
    \begin{itemize}
        \item Gibt es einen Compiler-Fehler?
        \item Gibt es einen Laufzeit-Fehler?
        \item Gibt es eine Ausgabe? Welche?
    \end{itemize}
\end{frame}

\begin{frame}{Quiz: For-Schleifen (2/2) - Antwort}
    \inputminted[linenos=true, numbersep=5pt, tabsize=4, fontsize=\small, label=QuizFor.java, frame=lines]{java}{QuizFor-2.java}
    Ausgabe: Endlosschleife\\
        \myCode{0 bottles of beer}\\
        \myCode{1 bottles of beer}\\
        \myCode{2 bottles of beer}\\
        \vdots
\end{frame}

\section{Hinweise zum ÜB 2}
\subsection{Allgemeine Hinweise}
\begin{frame}{Allgemeine Hinweise}
    \begin{alertblock}{Hinweise}
        \begin{itemize}[<+->]
            \item Auf \href{http://baldur.iti.kit.edu/programmieren/}{offizieller Lösung} aufbauen (Verpflichtend!)
            \item Auf Leerzeichen, gute Variablennamen und Konventionen achten
            \item Wird eine Bedinung von einem Setter-Parameter nicht eingehalten, schreibt ihr den Wert nicht
            \begin{itemize}
                \item Stattdessen: Fehlermeldung per \myCode{System.out.println("dies und das ist falsch");} ausgeben
                \item Das ist nur eine Hilfslösung, weil ich noch keine Exceptions hattet
                \item Später: (Fast) immer Exceptions!
            \end{itemize}
            \item Genauigkeit:
            \begin{itemize}
                \item Positiv bedeutet: $> 0$
                \item Negativ bedeutet: $< 0$
                \item nicht-negativ bedeutet: $\geq 0$
            \end{itemize}
        \end{itemize}
    \end{alertblock}
\end{frame}

\subsection{Wie werde ich besser?}
\begin{frame}{Wie werde ich besser?}
    \begin{itemize}
        \item \href{http://projecteuler.net/}{ProjectEuler.net}:
              401 Aufgaben jeder Schwierigkeitsstufe
        \item Anderen Code anschauen
        \item Probleme am PC lösen
    \end{itemize}
\end{frame}

\section{Praxis}
\subsection{ProjectEuler 28: Aufgabe}
\framedgraphic{ProjectEuler 28: Aufgabe}{euler-28.png}

\begin{frame}{ProjectEuler 28: Lösung mit Aufbau der Spirale}
    \inputminted[linenos=true, numbersep=5pt, tabsize=4, fontsize=\tiny, label=Euler28.java, frame=lines]{java}{Euler28.java}
\end{frame}

\begin{frame}{ProjectEuler 28: Lösung ohne Aufbau der Spirale}
    \inputminted[linenos=true, numbersep=5pt, tabsize=4, fontsize=\tiny, label=Euler28.java, frame=lines]{java}{Euler28-fast.java}
\end{frame}

\section{Abspann}
\subsection{Kommende Tutorien}
\begin{frame}{Kommende Tutorien}
  \begin{itemize}
    \item[10.] 12.11.2012
    \item[9.] 19.11.2012
    \item[8.] 26.11.2012
    \item[7.] 03.12.2012
    \item[6.] 10.12.2012
    \item[5.] 17.12.2012: Video "`Library"' zeigen
    \item[-] 24.12.2012: Heiligabend - \href{http://www.fmc.uni-karlsruhe.de/faq/wann-sind-die-weihnachtsferien}{Kein Tutorium}
    \item[-] 31.12.2012: Silvester - Kein Tutorium
    \item[4.] 07.01.2013
    \item[3.] 14.01.2013
    \item[2.] 21.01.2013
    \item[1.] 28.01.2013: Abschlussprüfunsvorbereitung
    \item[0.] 04.02.2013: Abschlussprüfunsvorbereitung
    \item[-] 10.02.2013: Ende der Vorlesungszeit des WS 2012/2013 (\href{http://www.kit.edu/studieren/2873.php}{Quelle})
  \end{itemize}
\end{frame}

\subsection{Comic}
\framedgraphic{Vielen Dank für eure Aufmerksamkeit!}{../images/xkcd-good_code.png}

\end{document}
