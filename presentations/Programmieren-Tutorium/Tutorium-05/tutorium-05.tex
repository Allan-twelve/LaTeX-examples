\documentclass[usepdftitle=false,hyperref={pdfpagelabels=false}]{beamer}

% use KIT-Theme
% see http://sdqweb.ipd.kit.edu/wiki/Dokumentvorlagen
%\usetheme{Frankfurt} % see http://deic.uab.es/~iblanes/beamer_gallery/index_by_theme.html as fallback
\InputIfFileExists{../templates/beamerthemekit.sty}{\usepackage{../templates/beamerthemekit}}{\usetheme{Frankfurt}}
\usefonttheme{professionalfonts}

\usepackage{hyperref}
\usepackage{lmodern}
\usepackage{listings}
\usepackage{wrapfig}        % see http://en.wikibooks.org/wiki/LaTeX/Floats,_Figures_and_Captions
\usepackage[utf8]{inputenc} % this is needed for german umlauts
\usepackage[ngerman]{babel} % this is needed for german umlauts
\usepackage[T1]{fontenc}    % this is needed for correct output of umlauts in pdf
\usepackage{verbatim}
\usepackage{relsize}
\usepackage{subfigure}
\usepackage{algorithm,algpseudocode}
\usepackage{minted}         % needed for the inclusion of source code
\usepackage{menukeys}
\usepackage{xcolor}
\usepackage{graphicx}
\usepackage{wrapfig}
\usepackage{../templates/myStyle}

\newcommand\tutor{Martin Thoma}
\newcommand\tutNR{10}
\newcommand\titleText{Programmieren-Tutorium Nr. \tutNR{} bei \tutor}
\institute{Fakultät für Informatik}

\hypersetup{pdftitle={\titleText}}
\beamertemplatenavigationsymbolsempty

\newcommand\InsertToC[1][]{
  \begin{frame}{Outline}
    \tableofcontents[subsectionstyle=show/show/show, subsubsectionstyle=show/show/show, #1]
  \end{frame}
}

\begin{document}
\title{\titleText}
\subtitle{Coding Style, Arrays}
\author{\tutor}
\date{\today}
\subject{Programmieren}

\frame{\titlepage}

\frame{
    \frametitle{Inhaltsverzeichnis}
    \setcounter{tocdepth}{1}
    \tableofcontents
    \setcounter{tocdepth}{2}
}

\section{Einleitung}
\subsection{Quiz}
\begin{frame}{Quiz}
    \inputminted[linenos=true, numbersep=5pt, tabsize=4, fontsize=\small, label=QuizArray.java, frame=lines]{java}{QuizArray.java}
    \begin{itemize}
        \item Gibt es einen Compiler-Fehler?
        \item Gibt es einen Laufzeit-Fehler?
        \item Gibt es eine Ausgabe? Welche?
    \end{itemize}
\end{frame}

\begin{frame}{Quiz: Antwort}
    \begin{block}{Quelltext}
        \inputminted[linenos=true, numbersep=5pt, tabsize=4, fontsize=\tiny]{java}{QuizArray.java}
    \end{block}

    \begin{block}{Ausgabe}
        myArray1:  geh  du alter sack\\
        myArray2:  geh  du alter sack
    \end{block}
\end{frame}

\begin{frame}{Shallow copy und deep copy}
    \includegraphics*[height=5cm,angle=-90]{so-obj1.pdf}

    \begin{minipage}[b]{0.45\linewidth}
    \begin{figure}[htb]
        \includegraphics*[height=5cm,angle=-90]{so-obj1-shallow-copy.pdf}
        \caption{shallow copy}
        \label{UploadForm}
    \end{figure}
    \end{minipage}
    \hspace{0.5cm}
    \begin{minipage}[b]{0.45\linewidth}
    \begin{figure}[htb]
        \includegraphics*[height=5cm,angle=-90]{so-obj1-deep-copy.pdf}
        \caption{deep copy}
        \label{UploadForm}
    \end{figure}
    \end{minipage}

    \tiny{Weitere infos auf \href{http://stackoverflow.com/a/9835954/562769}{stackoverflow.com}}
\end{frame}

\begin{frame}{Shallow copy und deep copy}
    \begin{itemize}
        \item Java macht standardmäßig eine "`shallow copy"'
        \item Wenn ihr eine "`deep copy"' wollt, müsst ihr das selbst machen
    \end{itemize}
    \begin{alertblock}{Lehre aus dem Quiz}
        Beim kopieren von Objekten (alles außer \myCode{boolean}, \myCode{byte},
        \myCode{int}, \myCode{long}, \myCode{float}, \myCode{double} -
        oder einfacher, alles was mit einem Großbuchstaben beginnt)
        ist Vorsicht geboten.
    \end{alertblock}
\end{frame}

\begin{frame}{Shallow copy und deep copy}
    Deep copy sieht im Beispiel so aus:
    \inputminted[linenos=true, numbersep=5pt, tabsize=4, fontsize=\tiny, label=QuizArray.java, frame=lines]{java}{QuizArray-solution.java}
\end{frame}

\section{Coding Style}
\subsection{Magic Numbers}
\begin{frame}{Magic Numbers}
    \begin{alertblock}{Definition: Magic numbers}
        Unique values with unexplained meaning or multiple occurrences
        which could (preferably) be replaced with named constants.
    \end{alertblock}

    Warum sind Konstanten besser als magic numbers?
    \begin{itemize}
        \item Code wird lesbarer
        \item Code wird leichter erweiterbar
        \item Kein (nennenswerter) Geschwindigkeits- oder Speicherplatznachteil
    \end{itemize}

    \begin{block}{Faustregel}
        Alle Zahlen, bis auf \myCode{0}, \myCode{1} und \myCode{-1}
        sollten mittels Konstanten verwendet werden.
    \end{block}
\end{frame}

\begin{frame}{Magic Numbers: Beispiel mit magic numbers}
    \inputminted[linenos=true, numbersep=5pt, tabsize=4, fontsize=\tiny]{java}{ConnectFourGame-bad.java}
\end{frame}

\begin{frame}{Magic Numbers: Beispiel mit magic Konstanten}
    \inputminted[linenos=true, numbersep=5pt, tabsize=4, fontsize=\tiny]{java}{ConnectFourGame.java}
\end{frame}

\section{Praxis}
\subsection{Durchschnitt}
\begin{frame}{Durchschnitt: Aufgabe}
    Schreiben Sie eine Methode \myCode{float getAverage(int[] numbers) \{ \dots \}},
    die den Durchschnitt der Zahlen im Array \myCode{numbers} berechnet.
\end{frame}

\begin{frame}{Durchschnitt: Lösung}
    \inputminted[linenos=true, numbersep=5pt, tabsize=4, fontsize=\small]{java}{AverageCalculation.java}
\end{frame}

\subsection{Flughafenkürzel}
\begin{frame}{Flughafenkürzel: Aufgabe}
    Wir haben ein Array mit IATA-Codes (Flughafenkürzel) gegeben und
    möchten den Inhalt auf der Konsole ausgeben.
\end{frame}

\begin{frame}{Flughafenkürzel: Lösung}
    \inputminted[linenos=true, numbersep=5pt, tabsize=4, fontsize=\small]{java}{IataCode.java}
\end{frame}

\begin{frame}{Flughafenkürzel: Ausgabe}
    Kein Compiler- oder Laufzeitfehler, obwohl \myCode{iataCodes[3]}
    nicht initialisiert ist!

    \begin{block}{Ausgabe}
    MUC\\
    BER\\
    AGB\\
    null
    \end{block}
\end{frame}

\begin{frame}{Flughafenkürzel: Umgekehrte Reihenfolge}
    Modifiziere das Programm so, dass die Kürzel in umgekehrter
    Reihenfolge ausgegeben werden.
    \begin{block}{Lösung von vorhin}
    \inputminted[linenos=true, numbersep=5pt, tabsize=4, fontsize=\tiny]{java}{IataCode.java}
    \end{block}
\end{frame}

\begin{frame}{Flughafenkürzel: Umgekehrte Reihenfolge - Lösung}
    \inputminted[linenos=true, numbersep=5pt, tabsize=4, fontsize=\small]{java}{IataCode-2.java}
\end{frame}

\begin{frame}{Flughafenkürzel: Jumping}
    Modifiziere das Programm so, dass nur jedes dritte Element
    ausgegeben wird.
    \begin{block}{Lösung von vorhin}
    \inputminted[linenos=true, numbersep=5pt, tabsize=4, fontsize=\tiny]{java}{IataCode.java}
    \end{block}
\end{frame}

\begin{frame}{Flughafenkürzel: Jumping - Lösung}
    \inputminted[linenos=true, numbersep=5pt, tabsize=4, fontsize=\small]{java}{IataCode-3.java}
\end{frame}

\subsection{Praxis: Bilder}
\begin{frame}{Praxis: Bilder}
    Gegeben ist ein Schwarz-Weiß Bild \myCode{int[][] image}.
    \begin{enumerate}
        \item Was repräsentiert \myCode{image[2][42]}?
        \item Schreiben Sie eine Methode, die die durchschnittliche
              Farbe von \myCode{image} berechnet und zurückgibt.
    \end{enumerate}
\end{frame}

\begin{frame}{Praxis: Bilder - Lösung}
    \inputminted[linenos=true, numbersep=5pt, tabsize=4, fontsize=\tiny]{java}{Bilder.java}

    Mögliche Probleme:
    \begin{itemize}
        \item Ein \myCode{int} für \myCode{sum} ist eventuell zu klein
        \item 255 Grauwerte, \myCode{int} hat maximalwert von 2.147.483.64
        \item[$\Rightarrow$] $\frac{2147483647}{255} \approx 8.4 \cdot 10^6$
        \item[$\Rightarrow$] maximal 8.4 Millionen Pixel
        \item[$\Rightarrow$] $\sqrt{8.4 \cdot 10^6} \approx 2900$
        \item[$\Rightarrow$] maximale Auflösung von $2900 \times 2900$ Pixel
        \item Eventuell \myCode{int[][] image = new int[0][0];}
    \end{itemize}
\end{frame}

\subsection{Praxis: Sieb des Erasthostenes}
\begin{frame}{Praxis: Sieb des Erasthostenes}
    Implementieren Sie das Sieb des Erasthostenes.
\end{frame}

\section{Allgemeine Hinweise zu ÜB}
\subsection{Praktomat}
\begin{frame}{Praktomat}
    \begin{itemize}
        \item Fragen frühzeitig stellen!
        \item Wenn ihr Pakete verwendet, am Anfang eurer Dateien
              also \myCode{package AB\_2;} oder ähnliches steht,
              müsst ihr die korrekte Ordnerstruktur in den Praktomat
              hochladen
        \item In Eclipse im Fenster "`Navigation"' den Ordner "`src"'
              auswählen
        \item \menu{File > Export\dots > File system}
        \item Zip-Datei aus dem exportiertem src-Ordner erstellen
        \item Zip-Datei hochladen
    \end{itemize}
\end{frame}

\framedgraphic{Projekt mit Eclipse exportieren}{eclipse-export-project.png}

\begin{frame}{Allgemeines}
    \begin{itemize}
        \item Eclipse-Warnungen beherzigen und die Ursache dafür beseitigen
        \item Formattieren: \keys{Strg + A} und \keys{Strg + \shift + F}
        \item Encoding: \menu{Project > Properties > Ressource > Text file encoding > UTF-8}
    \end{itemize}
\end{frame}

\section{Hinweise zum ÜB 3}
\subsection{ÜB 3}
\begin{frame}{ÜB 3}
    \begin{itemize}
        \item A1:
        \begin{itemize}
            \item Pro Aufgabe jeweils eine Textdatei(\myCode{A1.txt}, \myCode{B1.txt})
            \item Schaut euch meinen Artikel \href{http://martin-thoma.com/java-puzzle-1-pre-and-postincrement/}{Java Puzzle \#1: Pre- and Postincrement} an
        \end{itemize}
        \item A2: Buchstaben "`d"' und "`f"' nicht verwenden
    \end{itemize}
\end{frame}

\section{Abspann}
\subsection{Comic}
\framedgraphic{Abgaben um 11:55 Uhr}{ub-2-abgabge.png}
\framedgraphic{Vielen Dank für eure Aufmerksamkeit!}{../images/geekandpoke-2010-01-commenting.jpg}

\end{document}
