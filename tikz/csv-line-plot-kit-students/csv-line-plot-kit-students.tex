\documentclass[varwidth=true, border=2pt]{standalone}
\usepackage[utf8]{inputenc} % this is needed for umlauts
\usepackage[ngerman]{babel} % this is needed for umlauts
\usepackage[T1]{fontenc}    % this is needed for correct output of umlauts in pdf
\usepackage[margin=2.5cm]{geometry} %layout

\usepackage{tikz}
\usepackage{pgfplots}
\pgfplotsset{compat=1.12}

\begin{document}
\begin{tikzpicture}
    \begin{axis}[
            axis x line=middle,
            axis y line=middle,
            enlarge y limits=true,
            xmin=1986, xmax=2014,
            ymin=0, ymax=26000,
            width=15cm, height=8cm,     % size of the image
            grid = major,
            grid style={dashed, gray!30},
            ylabel=Anzahl,
            ylabel style={at={(-0.1,1.0)}},
            xlabel=Jahr,
            legend style={at={(0.5,1.0)}, anchor=north},
            xticklabel style=
            {/pgf/number format/1000 sep=,rotate=60,anchor=east,font=\scriptsize},
            yticklabel style=
            {scaled ticks=false,/pgf/number format/1000 sep=\,,rotate=0,anchor=east,font=\scriptsize},
            legend style={draw=none, legend columns=-1}
         ]
          \addplot[thick, blue, mark=x] table [x=year, y=totalStudents, col sep=comma] {students.csv};
          \addplot[thick, red, mark=x] table [x=year, y=female, col sep=comma] {students.csv};
          \addplot[thick, orange, mark=x] table [x=year, y=foreigners, col sep=comma] {students.csv};
          \legend{Eingeschriebene Studenten, davon Frauen, davon Ausländer}
    \end{axis}
\end{tikzpicture}
\end{document}
