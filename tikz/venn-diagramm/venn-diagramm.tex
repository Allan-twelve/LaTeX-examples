\documentclass{article}
\usepackage[pdftex,active,tightpage]{preview}
\setlength\PreviewBorder{2mm}
\usepackage{tikz}
\usetikzlibrary{shapes,decorations,calc}
\usepackage{amsmath,amssymb}
\begin{document}
\begin{preview}
\begin{tikzpicture}[%
    auto,
    example/.style={
      rectangle,
      draw=blue,
      thick,
      fill=blue!20,
      text width=4.5em,
      align=center,
      rounded corners,
      minimum height=2em
    },
    algebraicName/.style={
      text width=7em,
      align=center,
      minimum height=2em
    },
    explanation/.style={
      text width=10em,
      align=left,
      minimum height=3em
    }
  ]
    \draw[fill=yellow!20,yellow!20, rounded corners] (-1.85, 0.55) rectangle (13.4,-6.85);
    \draw[fill=black!20,black!20, rounded corners]   ( 7.53,-1.40) rectangle (13.0,-6.45);
    \draw[fill=lime!20,lime!20, rounded corners]     (-1.75, 0.45) rectangle (7.3,-6.75);
    \draw[fill=purple!20,purple!20, rounded corners] (-1.65,-1.40) rectangle (7.2,-6.65);
    \draw[fill=blue!20,blue!20, rounded corners]     (-1.55,-3.55) rectangle (7.1,-6.55);
    \draw[fill=red!20,red!20, rounded corners]       (-1.45,-4.65) rectangle (7.0,-6.45);
    \draw (0, 0) node[algebraicName] (A) {Gruppe}
          (2, 0) node[explanation]   (B) {
            \begin{minipage}{0.90\textwidth}
                \tiny
                \begin{itemize}
                \itemsep -0.3em
                \item Assoziativit\"at
                \item Neutrales Element
                \item Inverse Elemente
                \end{itemize}
            \end{minipage}
          }
          (6, 0) node[example, draw=lime, fill=lime!15] (X) {$\text{GL}(n, \mathbb{K})$}
          (6,-1) node[example, draw=lime, fill=lime!15] (X) {$\text{O}(n)$}
          (0,-2) node[algebraicName, purple] (C) {abelsche Gruppe}
          (2,-2) node[explanation]   (X) {
            \begin{minipage}{150\textwidth}
                \tiny
                \begin{itemize}
                \itemsep -0.3em
                \item kommutativ
                \end{itemize}
            \end{minipage}
          }
          (2, -3) node[example, draw=purple, fill=purple!15] (D) {$(\mathbb{Z}, +)$}
          (4, -3) node[example, draw=purple, fill=purple!15] (E) {$(\mathbb{Q} \setminus \{0\}, \cdot)$}
          (6, -3) node[example, draw=purple, fill=purple!15] (X) {$\mathbb{Z}_1$}

          (10,-6) node[example, draw=black, fill=black!15] (F) {$(\mathbb{N}_0, +)$}
          (12,-6) node[example, draw=black, fill=black!15] (G) {$(\mathbb{N}_0, \cdot)$}

          (0,-4) node[algebraicName, blue] (H) {Ring}
          (2,-4.1) node[explanation]   (X) {
            \begin{minipage}{150\textwidth}
                \tiny
                \begin{itemize}
                \itemsep -0.3em
                \item Zwei Verkn\"upfungen
                \item $(R, +)$ ist abelsche Gruppe
                \item $(R, \cdot)$ ist Halbgruppe
                \item Distributivgesetze
                \end{itemize}
            \end{minipage}
          }
          (6,-4) node[example, draw=blue, fill=blue!15] (I) {$(\mathbb{Z}, +, \cdot)$}

          (0,-5) node[algebraicName, red] (J) {K\"orper}
          (2,-5) node[explanation]   (X) {
            \begin{minipage}{150\textwidth}
                \tiny
                \begin{itemize}
                \itemsep -0.3em
                \item $(\mathbb{K} \setminus \{0\}, \cdot)$ ist abelsche Gruppe
                \end{itemize}
            \end{minipage}
          }
          (0,-6) node[example, draw=red, fill=red!15] (K) {$(\mathbb{Q}, +, \cdot)$}
          (2,-6) node[example, draw=red, fill=red!15] (L) {$(\mathbb{R}, +, \cdot)$}
          (4,-6) node[example, draw=red, fill=red!15] (M) {$(\mathbb{C}, +, \cdot)$}
          (6,-6) node[example, draw=red, fill=red!15] (N) {$\mathbb{Z} / p \mathbb{Z}$}


          (9, 0) node[algebraicName] (O) {Halbgruppe}
          (12,0) node[explanation]   (X) {
            \begin{minipage}{150\textwidth}
                \tiny
                \begin{itemize}
                \itemsep -0.3em
                \item Eine Verkn\"upfung
                \item Abgeschlossenheit
                \end{itemize}
            \end{minipage}
          }
          (12,-1) node[example, draw=yellow, fill=yellow!15] (P) {$(\emptyset, \emptyset)$}
          (9, -2) node[algebraicName] (Q) {kommutative Halbgruppe}
          (12,-2) node[explanation]   (X) {
            \begin{minipage}{150\textwidth}
                \tiny
                \begin{itemize}
                \itemsep -0.3em
                \item kommutativ
                \end{itemize}
            \end{minipage}
          };

    % Körper
    \draw[red,thick, rounded corners]     ($(J.north west)+(-0.1,0.0)$) rectangle ($(N.south east)+(0.1,-0.1)$);
    % Ring
    \draw[blue, thick, rounded corners]   ($(H.north west)+(-0.2,0.1)$) rectangle ($(N.south east)+(0.2,-0.2)$);
    % abelsche Gruppe
    \draw[purple, thick, rounded corners] ($(C.north west)+(-0.3,0.1)$) rectangle ($(N.south east)+(0.3,-0.3)$);
    % Gruppe
    \draw[lime, thick, rounded corners]   ($(A.north west)+(-0.4,0.1)$) rectangle ($(N.south east)+(0.4,-0.4)$);
    % Halbgruppe
    \draw[yellow, thick, rounded corners] ($(A.north west)+(-0.5,0.2)$) rectangle ($(G.south east)+(0.5,-0.5)$);
    % Halbgruppe
    \draw[black, thick, rounded corners] ($(Q.north west)+(-0.1,0.1)$) rectangle ($(G.south east)+(0.1,-0.1)$);
\end{tikzpicture}
\end{preview}
\end{document}
