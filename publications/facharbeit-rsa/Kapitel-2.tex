\section{Einwegfunktionen}
Eine Einwegfunktion ist in der Mathematik eine Beziehung zwischen
zwei Mengen, die "`komplexitätstheoretisch "`schwer"' umzukehren ist"'\footnote{[Beutelspacher], S. 114}.
Ein Beispiel für eine Einwegfunktion ist die Multiplikation zweier
Zahlen. Die Laufzeit des Schönhage-Strassen-Algorithmus zur
Multiplikation zweier $n$-stelliger ganzer Zahlen ist mit
$\mathcal{O}(n \cdot \log(n) \cdot \log(log(n)))$\footnote{[Pethö], S. 25}
deutlich kleiner als die Laufzeit von  des Zahlkörpersiebs
$\mathcal{O}(e^{(1,92+o(1)) \sqrt[3]{\ln n} \sqrt[3]{(\ln \ln n)^2}})$\footnote{[Rothe], S. 384},
das der Faktorisierung dient.

Die Sicherheit des RSA-Verfahrens zur asymmetrischen
Verschlüsselung basiert auf der Annahme, dass die Faktorisierung
einer großen Zahl deutlich länger dauert als das Multiplizieren der
Primfaktoren. Falls es keinen besseren Algorithmus zur Faktorisierung
als zur Multiplikation gibt, ist diese Annahme korrekt. Nach dem
Stand von 2009 ist dies der Fall.

Weitere Hinweise zur Sicherheit des RSA-Kryptosystems sind in \cref{sec:Security} zu finden.
