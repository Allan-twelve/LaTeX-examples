\documentclass[technote,a4paper,leqno]{IEEEtran}
\pdfoutput=1

\usepackage[utf8]{inputenc} % this is needed for umlauts
\usepackage[ngerman]{babel} % this is needed for umlauts
\usepackage[T1]{fontenc}    % this is needed for correct output of umlauts in pdf
\usepackage{graphicx} % Standardpaket zur Grafikeinbindung
\usepackage{amsmath,amssymb} % Erweiterung des Mathematik-Modus
\usepackage[colorinlistoftodos, german]{todonotes} % Option 'disable' entfernt alle ToDos
\usepackage[absolute,overlay]{textpos}
\usepackage{vmargin}          % Adjust margins in a simple way
\usepackage{tikz}
\usepackage{csquotes}
\usepackage[binary-units=true]{siunitx}
\usepackage{minted} % needed for the inclusion of source code

\usepackage{url}
\usepackage{breakurl}
\usepackage[raiselinks=true,
            bookmarks=true,
            bookmarksopenlevel=1,
            bookmarksopen=true,
            bookmarksnumbered=true,
            breaklinks,
            hyperindex=true,
            plainpages=false,
            pdfpagelabels=true,
            pdfborder={0 0 0.5}]{hyperref}
\def\UrlBreaks{\do\/\do-}

\usepackage{xspace}
\newcommand*\elide{\textup{[\,\dots]}\xspace}

\usepackage[german,nameinlink, noabbrev,capitalise]{cleveref}

\title{Sicherheit in Kognitiven Automobilien}
\author{%
    \IEEEauthorblockN{Martin Thoma}\\
    \IEEEauthorblockA{E-Mail: info@martin-thoma.de} % ORCID: http://orcid.org/0000-0002-6517-1690
}

\hypersetup{
    pdfauthor   = {Martin Thoma},
    pdfkeywords = {security, Sicherheit, Automobile, Hacking},
    pdfsubject  = {security},
    pdftitle    = {Sicherheit in Kognitiven Automobilien},
}

\usepackage{microtype}

\begin{document}
\maketitle
In dieser Arbeit wird der DYCOS-Algorithmus, wie er in \cite{aggarwal2011} vorgestellt wurde, erklärt.
Er arbeitet auf Graphen, deren Knoten teilweise mit
Beschriftungen versehen sind und ergänzt automatisch Beschriftungen
für Knoten, die bisher noch keine Beschriftung haben. Dieser Vorgang
wird \enquote{Klassifizierung} genannt. Dazu verwendet er die
Struktur des Graphen sowie textuelle Informationen, die den Knoten
zugeordnet sind. Die in \cite{aggarwal2011} beschriebene experimentelle
Analyse ergab, dass er auch auf dynamischen Graphen mit $\num{19396}$
bzw. $\num{806635}$ Knoten, von denen nur $\num{14814}$ bzw. $\num{18999}$
beschriftet waren, innerhalb von weniger als einer Minute auf einem
Kern einer Intel Xeon 2.5GHz CPU mit 32G RAM ausgeführt werden kann.\\
Zusätzlich wird \cite{aggarwal2011} kritisch Erörtert und
und es werden mögliche Erweiterungen des DYCOS-Algorithmus vorgeschlagen.

\textbf{Keywords:} DYCOS, Label Propagation, Knotenklassifizierung

%!TEX root = ../booka4.tex

\chapter{Einleitung}
Kognitive Automobile sind, im Gegensatz zu klassischen Automobilen, in der Lage
ihre Umwelt und sich selbst wahrzunehmen und dem Fahrer zu assistieren oder
auch teil- bzw. vollautonom zu fahren. Diese Systeme benötigen Zugriff auf
Sensoren und Aktoren, um ihre Aufgabe zu erfüllen. So benötigt ein Auto mit
Antiblockiersystem beispielsweise die Drehzahl an jedem Reifen und die
Möglichkeit die Bremsen zu beeinflussen; für Einparkhilfen werden Sensoren
benötigt, welche die Distanz zu Hindernissen wahrnehmen sowie Aktoren, die das
Auto lenken und beschleunigen können. Weitere dieser Systeme sind
Spurhalteassistenz, Spurwechselassistenz und Fernlichtassistenz.

Als immer mehr elektronische Systeme in Autos verbaut wurden, die teilweise
sich überschneidende Aufgaben erledigt haben, wurde der CAN-Bus
entwickelt~\cite{Kiencke1986}. Über ihn kommunizieren elektronische
Steuergeräte, sog. \textit{ECUs} (engl. \textit{electronic control units}).
Diese werden beispielsweise für ABS und ESP eingesetzt.

Der folgende Kapitel geht auf Standards wie den CAN-Bus und Verordnungen, die
in der Europäischen Union gültig sind, ein. In \cref{ch:attack} werden
Angriffsziele und Grundlagen zu den Angriffen erklärt, sodass in
\cref{ch:defense} mögliche Verteidigungsmaßnahmen erläutert werden können.

%!TEX root = ../booka4.tex
\section{Standards und Verordnungen}\label{ch:standards}
Für den Automobilbereich existieren viele Standards und Verordnungen. In diesem
Abschnitt wird eine Auswahl vorgestellt, die Fahrzeuge der Klassen M$_1$ und
N$_1$ betrifft. Das sind Fahrzeuge zur Personenbeförderung
\enquote{mit mindestens vier Rädern und höchstens acht Sitzplätzen außer dem Fahrersitz}
sowie \enquote{für die Güterbeförderung ausgelegte und gebaute Kraftfahrzeuge
mit einer zulässigen Gesamtmasse von
3,5~Tonnen}~\cite{Richtlinie70/156/EWG:Fahrzeugklassen}.

In der EU wurde mit~\cite{EUDirective98/69/EC} die OBD-Schnittstelle
verpflichtend für Fahrzeuge der Klasse M$_1$ und N$_1$ mit Fremdzündungsmotor
ab 1.~Januar 2004. Die EU-Direktive führt weiter die in der ISO~DIS~15031-6
Norm aufgeführten Fehlercodes als Minimalstandard ein. Diese müssen
\enquote{für genormte Diagnosegeräte \elide uneingeschränkt zugänglich sein}.
Außerdem muss die Schnittstelle im Auto so verbaut werden, dass sie
\enquote{für das Servicepersonal leicht zugänglich \elide ist}.

Der Software-Zugang ist durch J2534 der Society of Automotive Engineers
standardisiert~\cite{SAE2004}. Dieser Standard stellt sicher, dass unabhängig
vom OBD-Reader Diagnosen über das Auto erstellt und die ECUs umprogrammiert
bzw. mit Aktualisierungen versorgt werden können.

Um die Daten bereitzustellen, werden verschiedene elektronische Komponenten
über den CAN-Bus vernetzt. Dieser ist durch ISO~11898 genormt.

Weiterhin wurde in der EU mit \cite{EURegulation661/2009} beschlossen, dass ab
1.~November 2012 alle PKWs für Neuzulassungen ein System zur
Reifen\-druck\-über\-wachung (engl. \textit{tire pressure monitoring system}, kurz
\textit{TPMS}) besitzen müssen. Seit 1.~November 2014 müssen alle Neuwagen ein
solches System besitzen. Da sich die Räder schnell drehen ist eine
kabelgebundene Übertragung der Druckmesswerte nicht durchführbar. Daher sendet
jeder Reifen kabellos ein Signal, welches von einem oder mehreren Sensoren im
Auto aufgenommen wird.

Mit \cite{EURegulation2015/ecall} wird für Fahrzeuge, die ab dem 31.~März 2018
gebaut werden das eCall-System, ein elektronisches Notrufsystem, verpflichtend.
Dabei müssen dem eCall-System \enquote{präzise\mbox{[-]} und verlässliche\mbox{[-]}
Positionsdaten} zur Verfügung stehen, welche über das globales
Satelliten\-navigations\-system Galileo und dem Erweiterungssystem EGNOS geschehen
soll. eCall soll über öffentliche Mobilfunknetze eine \enquote{Tonverbindung
zwischen den Fahrzeug\-insassen und einer eCall-Notruf\-abfrage\-stelle} herstellen
können. Außerdem muss ein Mindestdatensatz übermittelt werden, welcher in
DIN EN 15722:2011 geregelt ist. Diese Funktionen müssen im Fall eines schweren
Unfalls automatisch durchgeführt werden können.

%!TEX root = ../booka4.tex
\section{Angriffe}\label{ch:attack}

Eine Reihe von elektronischen Systemen wurde zum Diebstahlschutz entwickelt
\cite{Song2008,Turner1999,Hwang1997}. Allerdings passen sich auch Diebe an die
modernen Gegebenheiten, insbesondere Funk\-schlüssel \cite{Lee2014}, an. Außerdem
gehen diese Systeme von einem klassischen Angreifer aus, der sich
ausschließlich auf der Hardware-Ebene bewegt.

Im Folgenden werden in \cref{subsec:can-intern-attackers} zunächst
Möglichkeiten von netzwerk-internen Angreifer, d.h.~Angreifern welche
physischen Zugang zum CAN-Bus haben, aufgelistet. Es folgt in
\cref{subsec:can-extern-attackers} eine Beschreibung wie Angreifer ohne
direkten physischen Zugriff auf das Auto sich mit dem CAN-Bus verbinden können.
Viele Angriffe nutzen sogenannte Buffer Overflows aus. Diese werden
in~\cref{sec:Buffer-Overflow} erklärt. Abschließend folgt in
\cref{sec:sicherheitslage} eine Liste konkreter, öffentlich bekannt gewordener
Sicherheitslücken.

Es ist nicht notwendigerweise der Fall, dass alle ECUs am selben CAN-Bus
angeschlossen sind. Allerdings müssen einige der Geräte Daten an die OBD-II
Schnittstelle senden. Außerdem liegt es aus wirtschaftlichen Gründen nahe
möglichst wenige Leitungen zur Datenübertragung zu verlegen.


\subsection{CAN-interne Angriffe}\label{subsec:can-intern-attackers}
Koscher~et~al.~haben in \cite{Koscher2010} zwei nicht näher spezifizierte Autos
der selben Marke und des selben Modells untersucht. Sie waren in der Lage über
den CAN-Bus etliche Funktionen des Autos, unabhängig vom Fahrer, zu
manipulieren. Das beinhaltet das Deaktivieren und Aktivieren der Bremsen, das
Stoppen des Motors, das Steuern der Klimaanlage, Heizung, Lichter, Manipulation
der Anzeigen im Kombiinstrument sowie das Öffnen und Schließen der Schlösser.
Durch moderne Systeme wie eCall kann der Angreifer sich sogar einen
Kommunikationskanal zu dem Auto aufbauen. Dies setzt allerdings voraus, dass
der Angreifer sich bereits im auto-internen Netzwerk befindet.

\subsection{CAN-externe Angriffe}\label{subsec:can-extern-attackers}
In~\cite{Checkoway2011} wurde an einer Mittel\-klasse\-limosine mit
Standard\-komponenten gezeigt, dass der Zugang zum auto-internen Netzwerk über
eine Vielzahl an Komponenten erfolgen kann. So haben Checkoway~et~al.
CD-Spieler, Bluetooth und den OBD-Zugang als mögliche Angriffs\-vektoren
identifiziert.

Bei dem Angriff über den Media Player haben Checkoway~et~al. die Tatsache
genutzt, dass dieser am CAN-Bus angeschlossen ist und die Software des
Mediaplayers über eine CD mit einem bestimmten Namen und Dateiformat
aktualisiert werden kann. Außerdem wurde ein Fehler beim Abspielen der
Audio-Dateien genutzt um einen Buffer Overflow zu erzeugen. Es wurde gezeigt,
dass dieser genutzt werden kann um die Software des Media-Players zu
aktualisieren. Dafür muss nur eine modifizierte Audio-Datei, welche immer noch
abspielbar ist, auf der CD sein.

Der von Checkoway~et~al. durchgeführte Angriff via Bluetooth benötigt ein
mit dem Media Player verbundenes Gerät. Dieses nutzt dann
\verb+strcpy+-Befehle, bei denen die Puffergrößen nicht überprüft wurden um bei
der Bluetooth-Konfiguration aus um beliebigen Code auf der Telematik-Einheit
des Autos ausführen zu können. Daher ist das Smartphone des Autobesitzers ein
Angriffsvektor.

Die Bluetooth-Verbindung kann jedoch auch ohne ein verbundenes Gerät für
Angriffe genutzt werden. Dazu muss der Angreifer genügend Zeit in der Nähe des
Autos verbringen um die Bluetooth-MAC-Adresse zu erfahren. Damit kann er ein
Anfrage zum Verbindungsaufbau starten. Diese müsste der Benutzer normalerweise
mit der Eingabe einer PIN bestätigen. Checkoway~et~al. haben für ein Auto
gezeigt, dass die Benutzerinteraktion nicht nötig ist und die PIN via
Brute-Force, also das Ausprobieren aller Möglichkeiten, innerhalb von
10~Stunden gefunden werden kann. Allerdings kann dieser Angriff parallel
ausgeführt werden. Es ist also beispielsweise möglich diesen Angriff für alle
Autos in einem Parkhaus durchzuführen. Bei einem parallel durchgeführten
Brute-Force-Angriff ist der erste Erfolg deutlich schneller zu erwarten.

Die standardisierte und von Automechanikern zu Diagnosezwecken genutzte
OBD-Schnittstelle stellt einen weiteren Angriffs\-vektor dar. Für die
verschiedenen Marken gibt es Diagnose\-werkzeuge, wie z.B. NGS für Ford,
Consult~II für Nissan und der Diagnostic~Tester von Toyota. Diese dedizierten
Diagnose\-geräte werden allerdings über PCs mit Aktualisierungen versorgt.
Modernere Diagnose\-werkzeuge sind nicht mehr bei der Diagnose vom PC getrennt,
sondern werden direkt, über ein Kabel, \mbox{W-LAN} oder Bluetooth, mit einem PC
verbunden. Daher stellt die Diagnose- und Aktualisierungs\-tätigkeit von
Automechanikern einen weiteren Angriffs\-vektor dar. Wenn der Mechaniker ein
Diagnose\-gerät benutzt, welches ein \mbox{W-LAN} aufbaut, so können Angreifer
sich mit diesem verbinden und selbst Aktualisierungen durchführen. Außerdem
wurde von Checkoway~et~al. gezeigt, dass auch das Diagnose\-gerät selbst so
manipuliert werden kann, dass es automatisch die gewünschten Angriffe ausführt.

Wie in \cref{ch:standards} beschrieben wird eCall-System ab 2018 in Europa
verpflichtend eingeführt. Dieses nutzt das Mobilfunknetz zur Kommunikation.
Checkoway~et~al. haben gezeigt, dass Telematik-Einheiten von außerhalb des
Autos angewählt werden und die Software auf diese Weise aus beliebigen
Entfernungen aktualisiert werden kann. Dazu wurden zahlreiche Schwachstellen
der Telematik-Einheit von Airbiquitys Modem aqLink genutzt. Dieses Modem wird
unter anderem von BMW und Ford eingesetzt \cite{AirbiquityBMW,AirbiquityFord}.

Ein Angriff auf die Privatsphäre ist mit TPMS möglich. In~\cite{Rouf2010} wurde
gezeigt, dass TPMS-Signale zur Identifikation von Autos genutzt werden können.
Die Identifikation eines vorbeifahrenden Autos ist aus bis zu \SI{40}{\meter}
Entfernung möglich.

Genauso stellt das Mikrofon, welches wegen eCall ab 2018 in jedem Auto sein
muss, eine Möglichkeit zum Angriff auf die Privatsphäre dar.


\subsection{Buffer Overflow Angriffe}\label{sec:Buffer-Overflow}
Dieser Abschnitt erklärt anhand eines einfachen Beispiels wie Buffer Overflow
Angriffe durchgeführt werden.

Um Buffer Overflow Angriffe zu verstehen, müssen Grundlagen der Struktur eines
Prozesses im Speicher bekannt sein. Diese werden im Detail
in~\cite{Silberschatz2005} erklärt.

Buffer Overflow Angriffe nutzen die Tatsache aus, dass bestimmte Befehle wie
beispielsweise \verb+gets+ Zeichenketten in einen Puffer schreiben, ohne die
Größe des Puffers zu beachten. \verb+gets+ erhält als Parameter einen Zeiger
auf die Startadresse des Puffers. Wenn der Benutzer eine längere Eingabe macht
als der Puffer erlaubt, so wird in nach\-folgende Speicher\-bereiche
geschrieben. Dies kann an folgendem, aus~\cite{Arora2013} entnommenem und
leicht modifiziertem Beispiel beobachtet werden:

\lstinputlisting[language=C,title=simple.c]{simple.c}

Kompiliert man dieses Programm mit
\texttt{gcc -O0 -fno-stack-protector -g simple.c -o simple}, so kann mit der
Eingabe von 16~Zeichen die Variable \texttt{pass} überschrieben werden. In diesem
Fall wird deshalb der passwortgeschützte Code-Abschnitt ausgeführt, selbst
wenn das eingegebene Passwort fehlerhaft ist.

Allerdings ist es nicht nur möglich interne Variablen zu überschreiben, sondern
sogar beliebigen Code auszuführen. Dies wird in \cite{Mixter} mit einem sehr
ähnlichem Beispiel gezeigt und im Detail erklärt. Dabei wird nicht beliebiger
Text in den Puffer geschrieben, sondern sogenannter \textit{Shellcode}. Unter
Shellcode versteht man Assemblerbefehle, welche in Opcodes umgewandelt wurden.


\subsection{Sicherheitslage in Automobilen bis August 2015}\label{sec:sicherheitslage}
Heutzutage ist nicht nur die Hardware von Automobilen durch Diebe und andere
Angreifer gefährdet, sondern auch die Software. Die folgenden Beispiele zeigen,
dass Angriffe auf die IT in Automobilen nicht nur im akademischen Rahmen
auf einige wenige spezielle Modelle durchgeführt werden, sondern dass auch
modellübergreifende Angriffe möglich sind.

Es gibt etliche Automobilhersteller, -marken und -modelle. Für viele Modelle
gibt es unterschiedliche Konfigurationen und wiederum zahlreiche Optionen für
Zubehör wie beispielsweise das Autoradio oder Navigationssysteme. Dies macht
allgemeine Aussagen über konkrete Angriffe auf Automobile schwierig. Allerdings
stellen Standards und Verordnungen (vgl. \cref{ch:standards}) sicher, dass
Teile der relevanten Infrastruktur in Automobilen gleich sind, sodass Angreifer
diese fahrzeugübergreifend nutzen können.

Die Art der Angriffe ist nicht neu. So sind wurde mit dem Morris-Wurm bereits
1988 ein Stack Overflow Angriff durchgeführt~\cite{Seltzer2013},
Replay-Angriffe wurden 1994 beschrieben~\cite{Syverson1994} und seit Beginn
der Entwicklung von Viren werden bekannte Lücken in veralteter Software
ausgenutzt. Allerdings ist die Software in kognitiven Automobilen komplexer
als in herkömmlichen Automobilen, die Menge der eingesetzten Software ist
größer und die Möglichkeiten zur Einflussnahme durch Aktoren sind gewachsen.
Daher bieten kognitive Automobile eine größere Angriffsfläche als
herkömmliche Automobile.

\begin{itemize}
    \item 2010 hat ein ehemaliger Angestellter mehr als 100~Autos über ein
          Fernsteuersystem, welches Kunden an fällige Zahlungen erinnern soll,
          die Hupen aktiviert~\cite{Poulsen2010}.
    \item 2010 wurde mit~\cite{Koscher2010} auf mögliche Probleme in kognitiven
          Automobilen hingewiesen. Mit~\cite{Checkoway2011} wurde 2011 gezeigt,
          dass mindestens ein Modell in einer bestimmten Konfiguration unsicher
          ist.
    \item 2015 wurde von Charlie Miller und Chris Valasek gezeigt, dass das
          Unterhaltungssystem Uconnect von Fiat Crysler benutzt werden kann um
          Autos aus der Ferne zu übernehmen. Wegen dieses Softwarefehlers hat
          Fiat Chrysler 1,4~Millionen Autos zurückgerufen~\cite{Gallagher2015}.
    \item 2015 wurde eine Sicherheitslücke in BMW's ConnectedDrive bekannt.
          Diese hat es dem Angreifer erlaubt, das Auto zu
          öffnen~\cite{Spaar2015}. Dieter Spaar hat dabei mehrere
          Sicherheitslücken aufgedeckt: BMW hat in allen Fahrzeugen die selben
          symmetrischen Schlüssel eingesetzt, Teildienste haben keine
          Transportverschlüsselung verwendet und ConnectedDrive war nicht gegen
          Replay-Angriffe geschützt. Bei Replay-Angriffen nimmt der Angreifer
          Teile der Kommunikation zwischen dem BMW-Server und dem Auto auf und
          spielt diese später wieder ab. Dies könnte beispielsweise eine
          Nachricht sein, die das Auto entriegelt.
    \item In \cite{Verdult2015} wurde 2015 gezeigt, dass einige Dutzend
          Automodelle von Audi, Ferrari, Fiat, Opel, VW und weiterer Marken
          eine Sicherheitslücke in den Schlüsseln haben, welche es Autodieben
          erlaubt nach nur zweimaligem Abhören der Kommunikation des
          Originalschlüssels mit dem Auto eine Kopie des Schlüssels
          anzufertigen.
    \item Die Sicherheitsfirma Lookout hat 2015 Fehler in der Software des
          Tesla Model~S gefunden, welche Root-Zugang zu internen Systemen
          erlaubt hat~\cite{Mahaffey2015}. Drei der gefundenen sechs
          Sicherheitslücken sind veraltete Softwarekomponenten.
    \item 2015 wurde auch ein Angriff auf das OnStar-System von GM bekannt,
          durch welchen der Angreifer beliebige Autos mit dem OnStar-System
          öffnen und den Motor starten konnte~\cite{Stevens2015}. Die Art des
          Angriffs, die bisher noch nicht detailliert beschrieben wurde,
          betrifft laut \cite{Greenberg2015} auch die iOS-Apps für Remote von
          BMW, mbrace von Mercedes-Benz, Uconnect von Chrysler und Smartstart
          von Viper.
\end{itemize}

%!TEX root = ../booka4.tex
\section{Verteidigungsmaßnahmen}\label{ch:defense}
Wie bereits in \cref{sec:sicherheitslage} beschrieben, sind die Arten der
Angriffe nicht neu. Daher sind auch die Verteidigungs\-maßnahmen nicht
spezifisch für den Automobil\-bereich, sondern allgemeiner software\-technischer
Art.

Alle von Checkoway~et~al. beschriebenen Angriffe basieren zum einen auf
Reverse-Engineering, also der Rekonstruktion der Software-Systeme und
Protokolle, zum anderen auf Fehlern in der Software. Das Reverse-Engineering
wurde in einigen Fällen laut Checkoway~et~al. stark vereinfacht, da
Debugging-Symbole in der Software waren. Diese können und sollten
entfernt werden.


\subsection{Datenvalidierung}\label{sec:validation}
Der CAN-Bus ist eine große Schwachstelle der IT-Sicherheit in Autos. Über ihn
müssen viele ECUs kommunizieren und einige, wie das Autoradio, werden nicht als
sicherheits\-kritisch wahrgenommen. Gleichzeitig sind sicherheitskritische ECUs
an dem selben CAN-Bus angeschlossen. Daher ist es wichtig die Nachrichten,
welche über den CAN-Bus empfangen werden, zu filtern. Die Informationen müssen
auf Plausibilität geprüft werden. Insbesondere bei Software-Updates sollte
anhand einer kryptographischen Signatur überprüft werden, ob das Update vom
Hersteller stammt.

Außerdem sollten laut Checkoway~et~al. die Diagnose\-geräte Authentifizierung
und Verschlüsselung wie beispielsweise OpenSSL nutzen.


\subsection{Buffer Overflows}
Gegen Buffer-Overflow-Angriffe können zum einen Sprachen wie Java oder Rust
verwendet werden, welche die Einhaltung der Bereichs\-grenzen automatisch
überprüfen. Des Weiteren kann anstelle der C-Funktion \verb+strcpy()+ die
Funktion \verb+strncpy()+ verwendet werden, welche die Anzahl der zu
schreibenden Zeichen begrenzt~\cite{Eckert2012}.

Ein weiteres Konzept zum Schutz vor Buffer-Overflow-Angriffen sind Stack
Cookies~\cite{Bray2002}. Stack Cookies sind Werte die auf den Stack, direkt
nach den Puffer geschrieben werden. Bevor der Sprung zurück in
die aufrufende Funktion durchgeführt wird, wird die \verb+XOR+ Operation auf
den Stack Cookie und die Rücksprung\-adresse ausgeführt. Der so errechnete Wert
wird mit dem erwarteten Wert verglichen. Falls es eine Abweichung gibt wird
nicht die \verb+RET+ Operation ausgeführt, sondern in eine Sicherheits\-routine
gesprungen, die diesen Fall behandelt.


\subsection{Code-Qualität}
Code Reviews und IT-Sicherheits\-audits können solche Sicherheits\-lücken
aufdecken~\cite{Howard2006}. Code Reviews können teilweise automatisch mit
Werkzeugen zur statischen Code Analyse durchgeführt werden~\cite{McGraw2008}.
Insbesondere Fahrzeuge welche typischerweise von der Feuerwehr, der Polizei
oder von Rettungsdiensten eingesetzt werden sollten --- unabhängig vom Hersteller
--- überprüft werden.

Eine weiterer wichtiger Stützpfeiler für sichere Software sind schnell
ausgelieferte Sicherheits\-aktualisierungen. Dazu gehört
laut~\cite{Mahaffey2015} unter anderem ein System zum mobilen versenden von
Aktualisierungen an Autos mit Mobilfunk\-verbindung.


\bibliographystyle{IEEEtranSA}
\bibliography{literatur}
\end{document}
