\documentclass[technote,a4paper,leqno]{IEEEtran}
\pdfoutput=1
%---- Sonderzeichen-------%
\usepackage[utf8]{inputenc} % this is needed for umlauts
\usepackage[ngerman]{babel} % this is needed for umlauts
\usepackage[T1]{fontenc}    % this is needed for correct output of umlauts in pdf
%---- Codierung----%
\usepackage{graphicx}
\usepackage{url}
%----- Mathematischer Zeichenvorrat---%
\usepackage{amsmath}
\usepackage{amssymb}
\usepackage{enumerate}
% fuer die aktuelle Zeit
\usepackage{scrtime}
\usepackage{listings}
\usepackage{hyperref}
\usepackage{cite}
\usepackage{parskip}
\usepackage[framed,amsmath,thmmarks,hyperref]{ntheorem}
\usepackage{algorithm}
\usepackage[noend]{algpseudocode}
\usepackage{csquotes}
\usepackage{subfig}         % multiple figures in one
\usepackage{caption}
\usepackage{tikz}
\let\labelindent\relax
\usepackage{enumitem}
\usepackage[german]{cleveref} % nameinlink - arxiv has too old LaTeX :-(
\usepackage{braket}
\allowdisplaybreaks
\usetikzlibrary{backgrounds}
\usepackage[binary-units=true]{siunitx}
\sisetup{range-phrase=--}
\usepackage{mystyle}
\usepackage{microtype}

\setcounter{tocdepth}{3}
\setcounter{secnumdepth}{3}

\hypersetup{
  pdftitle    = {Über die Klassifizierung von Knoten in dynamischen Netzwerken mit textuellen Inhalten},
  pdfauthor   = {Martin Thoma}, % ORCID: http://orcid.org/0000-0002-6517-1690
  pdfkeywords = {DYCOS}
}

\begin{document}

\title{Über die Klassifizierung von Knoten in dynamischen Netzwerken mit Inhalt}
\author{Martin Thoma}
\date{17.01.2014}
\maketitle

\begin{abstract}%
In dieser Arbeit wird der DYCOS-Algorithmus, wie er in \cite{aggarwal2011} vorgestellt wurde, erklärt.
Er arbeitet auf Graphen, deren Knoten teilweise mit
Beschriftungen versehen sind und ergänzt automatisch Beschriftungen
für Knoten, die bisher noch keine Beschriftung haben. Dieser Vorgang
wird \enquote{Klassifizierung} genannt. Dazu verwendet er die
Struktur des Graphen sowie textuelle Informationen, die den Knoten
zugeordnet sind. Die in \cite{aggarwal2011} beschriebene experimentelle
Analyse ergab, dass er auch auf dynamischen Graphen mit $\num{19396}$
bzw. $\num{806635}$ Knoten, von denen nur $\num{14814}$ bzw. $\num{18999}$
beschriftet waren, innerhalb von weniger als einer Minute auf einem
Kern einer Intel Xeon 2.5GHz CPU mit 32G RAM ausgeführt werden kann.\\
Zusätzlich wird \cite{aggarwal2011} kritisch Erörtert und
und es werden mögliche Erweiterungen des DYCOS-Algorithmus vorgeschlagen.

\textbf{Keywords:} DYCOS, Label Propagation, Knotenklassifizierung

\end{abstract}

\section{Einleitung}
%!TEX root = ../booka4.tex

\chapter{Einleitung}
Kognitive Automobile sind, im Gegensatz zu klassischen Automobilen, in der Lage
ihre Umwelt und sich selbst wahrzunehmen und dem Fahrer zu assistieren oder
auch teil- bzw. vollautonom zu fahren. Diese Systeme benötigen Zugriff auf
Sensoren und Aktoren, um ihre Aufgabe zu erfüllen. So benötigt ein Auto mit
Antiblockiersystem beispielsweise die Drehzahl an jedem Reifen und die
Möglichkeit die Bremsen zu beeinflussen; für Einparkhilfen werden Sensoren
benötigt, welche die Distanz zu Hindernissen wahrnehmen sowie Aktoren, die das
Auto lenken und beschleunigen können. Weitere dieser Systeme sind
Spurhalteassistenz, Spurwechselassistenz und Fernlichtassistenz.

Als immer mehr elektronische Systeme in Autos verbaut wurden, die teilweise
sich überschneidende Aufgaben erledigt haben, wurde der CAN-Bus
entwickelt~\cite{Kiencke1986}. Über ihn kommunizieren elektronische
Steuergeräte, sog. \textit{ECUs} (engl. \textit{electronic control units}).
Diese werden beispielsweise für ABS und ESP eingesetzt.

Der folgende Kapitel geht auf Standards wie den CAN-Bus und Verordnungen, die
in der Europäischen Union gültig sind, ein. In \cref{ch:attack} werden
Angriffsziele und Grundlagen zu den Angriffen erklärt, sodass in
\cref{ch:defense} mögliche Verteidigungsmaßnahmen erläutert werden können.


\section{Related Work}
%!TEX root = Ausarbeitung-Thoma.tex
Sowohl das Problem der Knotenklassifikation, als auch das der Textklassifikation,
wurde bereits in verschiedenen Kontexten. Jedoch scheien bisher entweder nur die Struktur des zugrundeliegenden Graphen oder nur Eigenschaften der Texte verwendet worden zu sein.

So werden in \cite{bhagat,szummer} unter anderem Verfahren zur Knotenklassifikation
beschrieben, die wie der in \cite{aggarwal2011} vorgestellte DYCOS-Algorithmus,
um den es in dieser Ausarbeitung geht, auch auf Random Walks basieren.

Obwohl es auch zur Textklassifikation einige Paper gibt \cite{Zhu02learningfrom,Jiang2010302}, geht doch keines davon auf den Spezialfall der Textklassifikation
mit einem zugrundeliegenden Graphen ein.

Die vorgestellten Methoden zur Textklassifikation variieren außerdem sehr stark.
Es gibt Verfahren, die auf dem bag-of-words-Modell basieren \cite{Ko:2012:STW:2348283.2348453}
wie es auch im DYCOS-Algorithmus verwendet wird. Aber es gibt auch Verfahren,
die auf dem Expectation-Maximization-Algorithmus basieren \cite{Nigam99textclassification}
oder Support Vector Machines nutzen \cite{Joachims98textcategorization}.

Es wäre also gut Vorstellbar, die Art und Weise wie die Texte in die Klassifikation
des DYCOS-Algorithmus einfließen zu variieren. Allerdings ist dabei darauf hinzuweisen,
dass die im Folgeden vorgestellte Verwendung der Texte sowohl einfach zu implementieren
ist und nur lineare Vorverarbeitungszeit in Anzahl der Wörter des Textes hat,
als auch es erlaubt einzelne
Knoten zu klassifizieren, wobei der Graph nur lokal um den zu klassifizerenden
Knoten betrachten werden muss.

\section{DYCOS}
\subsection{Überblick}
DYCOS (\underline{DY}namic \underline{C}lassification algorithm with
c\underline{O}ntent and \underline{S}tructure) ist ein
Knotenklassifizierungsalgorithmus, der Ursprünglich in \cite{aggarwal2011}
vorgestellt wurde.

Ein zentrales Element des DYCOS-Algorithmus ist der sog. {\it Random Walk}:

\begin{definition}[Random Walk, Sprung]
    Sei $G = (V, E)$ mit $E \subseteq V \times V$ ein Graph und
    $v_0 \in V$ ein Knoten des Graphen.

    Ein Random Walk der Länge $l$ auf $G$, startend bei $v_0$ ist nun der
    zeitdiskrete stochastische Prozess, der $v_i$ auf einen zufällig gewählten
    Nachbarn $v_{i+1}$ abbildet (für $i \in 0, \dots, l-1$). Die Abbildung $v_i
    \mapsto v_{i+1}$ heißt ein Sprung.
\end{definition}

Der DYCOS-Algorithmus klassifiziert einzelne Knoten, indem $r$ Random Walks der
Länge $l$, startend bei dem zu klassifizierenden Knoten $v$ gemacht werden.
Dabei werden die Beschriftungen der besuchten Knoten gezählt. Die Beschriftung,
die am häufigsten vorgekommen ist, wird als Beschriftung für $v$ gewählt. DYCOS
nutzt also die sog. Homophilie, d.~h. die Eigenschaft, dass Knoten, die nur
wenige Hops von einander entfernt sind, häufig auch ähnlich sind \cite{bhagat}.
Der DYCOS-Algorithmus arbeitet jedoch nicht direkt auf dem Graphen, sondern
erweitert ihn mit Hilfe der zur Verfügung stehenden Texte. Wie diese
Erweiterung erstellt wird, wird im Folgenden erklärt.\\
Für diese Erweiterung wird zuerst wird Vokabular $W_t$ bestimmt, das
charakteristisch für eine Knotengruppe ist. Wie das gemacht werden kann und
warum nicht einfach jedes Wort in das Vokabular aufgenommen wird, wird in
\cref{sec:vokabularbestimmung} erläutert.\\
Nach der Bestimmung des Vokabulars wird für jedes Wort im Vokabular ein
Wortknoten zum Graphen hinzugefügt. Alle Knoten, die der Graph zuvor hatte,
werden nun \enquote{Strukturknoten} genannt.
Ein Strukturknoten $v$ wird genau dann mit einem Wortknoten $w \in W_t$
verbunden, wenn $w$ in einem Text von $v$ vorkommt. \Cref{fig:erweiterter-graph}
zeigt beispielhaft den so entstehenden, semi-bipartiten Graphen.
Der DYCOS-Algorithmus betrachtet also die Texte, die einem Knoten
zugeordnet sind, als eine Multimenge von Wörtern. Das heißt, zum einen
wird nicht auf die Reihenfolge der Wörter geachtet, zum anderen wird
bei Texten eines Knotens nicht zwischen verschiedenen
Texten unterschieden. Jedoch wird die Anzahl der Vorkommen
jedes Wortes berücksichtigt.

\begin{figure}[htp]
    \centering
    \tikzstyle{vertex}=[draw,black,circle,minimum size=10pt,inner sep=0pt]
\tikzstyle{edge}=[very thick]
\begin{tikzpicture}[scale=1.3]
    \node (a)[vertex] at (0,0) {};
    \node (b)[vertex]  at (0,1) {};
    \node (c)[vertex] at (0,2) {};
    \node (d)[vertex] at (1,0) {};
    \node (e)[vertex]  at (1,1) {};
    \node (f)[vertex] at (1,2) {};
    \node (g)[vertex] at (2,0) {};
    \node (h)[vertex] at (2,1) {};
    \node (i)[vertex] at (2,2) {};

    \node (x)[vertex] at (4,0) {};
    \node (y)[vertex] at (4,1) {};
    \node (z)[vertex] at (4,2) {};

    \draw[edge] (a) -- (d);
    \draw[edge] (b) -- (d);
    \draw[edge] (b) -- (c);
    \draw[edge] (c) -- (d);
    \draw[edge] (d) -- (e);
    \draw[edge] (d) edge[bend left] (f);
    \draw[edge] (d) edge[bend right] (x);
    \draw[edge] (g) edge (x);
    \draw[edge] (h) edge (x);
    \draw[edge] (h) edge (y);
    \draw[edge] (h) edge (e);
    \draw[edge] (e) edge (z);
    \draw[edge] (i) edge (y);

    \draw [dashed] (-0.3,-0.3) rectangle (2.3,2.3);
    \draw [dashed] (2.5,2.3) rectangle (5, -0.3);

    \node (struktur)[label={[label distance=0cm]0:Sturkturknoten $V_t$}] at (-0.1,2.5) {};
    \node (struktur)[label={[label distance=0cm]0:Wortknoten $W_t$}] at (2.7,2.5) {};
\end{tikzpicture}

    \caption{Erweiterter Graph}
    \label{fig:erweiterter-graph}
\end{figure}

Entsprechend werden zwei unterschiedliche Sprungtypen unterschieden, die
strukturellen Sprünge und inhaltliche Zweifachsprünge:

\begin{definition}[struktureller Sprung]
    Sei $G_{E,t} = (V_t, E_{S,t} \cup E_{W,t}, V_{L,t}, W_{t})$ der
    um die Wortknoten $W_{t}$ erweiterte Graph.

    Dann heißt das zufällige wechseln des aktuell betrachteten
    Knoten $v \in V_t$ zu einem benachbartem Knoten $w \in V_t$
    ein \textit{struktureller Sprung}.
\end{definition}
\goodbreak
Im Gegensatz dazu benutzten inhaltliche Zweifachsprünge tatsächlich die
Grapherweiterung:
\begin{definition}[inhaltlicher Zweifachsprung]
    Sei $G_t = (V_t, E_{S,t} \cup E_{W,t}, V_{L,t}, W_{t})$ der um die
    Wortknoten $W_{t}$ erweiterte Graph.

    Dann heißt das zufällige wechseln des aktuell betrachteten Knoten $v \in
    V_t$ zu einem benachbartem Knoten $w \in W_t$ und weiter zu einem
    zufälligem Nachbar $v' \in V_t$ von $w$ ein inhaltlicher Zweifachsprung.
\end{definition}

Jeder inhaltliche Zweifachsprung beginnt und endet also in einem
Strukturknoten, springt über einen Wortknoten und ist ein Pfad der Länge~2.

Ob in einem Sprung der Random Walks ein struktureller Sprung oder ein
inhaltlicher Zweifachsprung gemacht wird, wird jedes mal zufällig neu
entschieden. Dafür wird der Parameter $0 \leq p_S \leq 1$ für den Algorithmus
gewählt. Mit einer Wahrscheinlichkeit von $p_S$ wird ein struktureller Sprung
durchgeführt und mit einer Wahrscheinlichkeit von $(1-p_S)$ ein modifizierter
inhaltlicher Zweifachsprung, wie er in \cref{sec:sprungtypen} erklärt wird,
gemacht. Der Parameter $p_S$ gibt an, wie wichtig die Struktur des Graphen im
Verhältnis zu den textuellen Inhalten ist. Bei $p_S = 0$ werden ausschließlich
die Texte betrachtet, bei $p_S = 1$ ausschließlich die Struktur des Graphen.

Die Vokabularbestimmung kann zu jedem Zeitpunkt $t$ durchgeführt werden, muss
es aber nicht.

In \cref{alg:DYCOS} steht der DYCOS-Algorithmus in Form von Pseudocode:
In \cref{alg1:l8} wird für jeden unbeschrifteten Knoten
durch die folgenden Zeilen eine Beschriftung gewählt.

\Cref{alg1:l10} führt $r$ Random Walks durch. In \cref{alg1:l11} wird eine
temporäre Variable für den aktuell betrachteten Knoten angelegt.

In \cref{alg1:l12} bis \cref{alg1:l21} werden einzelne Random Walks der Länge
$l$ durchgeführt, wobei die beobachteten Beschriftungen gezählt werden und mit
einer Wahrscheinlichkeit von $p_S$ ein struktureller Sprung durchgeführt wird.

\begin{algorithm}[ht]
    \begin{algorithmic}[1]
        \Require \\$G_{E,t} = (V_t, E_{S,t} \cup E_{W,t}, V_{L,t}, W_t)$ (Erweiterter Graph),\\
                 $r$ (Anzahl der Random Walks),\\
                 $l$ (Länge eines Random Walks),\\
                 $p_s$ (Wahrscheinlichkeit eines strukturellen Sprungs),\\
                 $q$ (Anzahl der betrachteten Knoten in der Clusteranalyse)
        \Ensure  Klassifikation von $V_t \setminus V_{L,t}$\\
        \\

        \ForAll{Knoten $v \in V_t \setminus V_{L,t}$}\label{alg1:l8}
            \State $d \gets $ leeres assoziatives Array
            \For{$i = 1, \dots,r$}\label{alg1:l10}
                \State $w \gets v$\label{alg1:l11}
                \For{$j= 1, \dots, l$}\label{alg1:l12}
                    \State $sprungTyp \gets \Call{random}{0, 1}$
                    \If{$sprungTyp \leq p_S$}
                        \State $w \gets$ \Call{SturkturellerSprung}{$w$}
                    \Else
                        \State $w \gets$ \Call{InhaltlicherZweifachsprung}{$w$}
                    \EndIf
                    \State $beschriftung \gets w.\Call{GetLabel}{ }$
                    \If{$!d.\Call{hasKey}{beschriftung}$}
                        \State $d[beschriftung] \gets 0$
                    \EndIf
                    \State $d[beschriftung] \gets d[beschriftung] + 1$
                \EndFor\label{alg1:l21}
            \EndFor

            \If{$d$.\Call{isEmpty}{ }} \Comment{Es wurde kein beschrifteter Knoten gesehen}
                \State $M_H \gets \Call{HäufigsteLabelImGraph}{ }$
            \Else
                \State $M_H \gets \Call{max}{d}$
            \EndIf
            \\
            \State \textit{//Wähle aus der Menge der häufigsten Beschriftungen $M_H$ zufällig eine aus}
            \State $label \gets \Call{Random}{M_H}$
            \State $v.\Call{AddLabel}{label}$ \Comment{und weise dieses $v$ zu}
        \EndFor
        \State \Return Beschriftungen für $V_t \setminus V_{L,t}$
    \end{algorithmic}
\caption{DYCOS-Algorithmus}
\label{alg:DYCOS}
\end{algorithm}

\subsection{Datenstrukturen}
Zusätzlich zu dem gerichteten Graphen $G_t = (V_t, E_t, V_{L,t})$ verwaltet der
DYCOS-Algorithmus zwei weitere Datenstrukturen:
\begin{itemize}
    \item Für jeden Knoten $v \in V_t$ werden die vorkommenden Wörter,
          die auch im Vokabular $W_t$ sind,
          und deren Anzahl gespeichert. Das könnte z.~B. über ein
          assoziatives Array (auch \enquote{dictionary} oder
            \enquote{map} genannt) geschehen. Wörter, die nicht in
          Texten von $v$ vorkommen, sind nicht im Array. Für
          alle vorkommenden Wörter ist der gespeicherte Wert zum
          Schlüssel $w \in W_t$ die Anzahl der Vorkommen von
          $w$ in den Texten von $v$.
    \item Für jedes Wort des Vokabulars $W_t$ wird eine Liste von
          Knoten verwaltet, in deren Texten das Wort vorkommt.
          Diese Liste wird bei den inhaltlichen Zweifachsprung,
          der in \cref{sec:sprungtypen} erklärt wird,
          verwendet.
\end{itemize}

\subsection{Sprungtypen}\label{sec:sprungtypen}
Die beiden bereits definierten Sprungtypen, der strukturelle Sprung
sowie der inhaltliche Zweifachsprung werden im folgenden erklärt.
\goodbreak
Der strukturelle Sprung entspricht einer zufälligen Wahl eines
Nachbarknotens, wie es in \cref{alg:DYCOS-structural-hop}
gezeigt wird.
\begin{algorithm}[H]
    \begin{algorithmic}[1]
        \Procedure{SturkturellerSprung}{Knoten $v$, Anzahl $q$}
            \State $n \gets v.\Call{NeighborCount}{}$ \Comment{Wähle aus der Liste der Nachbarknoten}
            \State $r \gets \Call{RandomInt}{0, n-1}$ \Comment{einen zufällig aus}
            \State $v \gets v.\Call{Next}{r}$ \Comment{Gehe zu diesem Knoten}
            \State \Return $v$
        \EndProcedure
    \end{algorithmic}
\caption{Struktureller Sprung}
\label{alg:DYCOS-structural-hop}
\end{algorithm}

Bei inhaltlichen Zweifachsprüngen ist jedoch nicht sinnvoll so strikt
nach der Definition vorzugehen,  also
direkt von einem strukturellem Knoten
$v \in V_t$ zu einem mit $v$ verbundenen Wortknoten $w \in W_t$ zu springen
und von diesem wieder zu einem verbundenem strukturellem Knoten
$v' \in V_t$. Würde man dies machen, wäre zu befürchten, dass
aufgrund von Homonymen die Qualität der Klassifizierung verringert
wird. So hat \enquote{Brücke} im Deutschen viele Bedeutungen.
Gemeint sein können z.~B. das Bauwerk, das Entwurfsmuster der
objektorientierten Programmierung oder ein Teil des Gehirns.

Deshalb wird für jeden Knoten $v$, von dem aus man einen inhaltlichen
Zweifachsprung machen will folgende Textanalyse durchgeführt:
\begin{enumerate}[label=C\arabic*,ref=C\arabic*]
    \item \label{step:c1} Gehe alle in $v$ startenden Random Walks der Länge $2$ durch
          und erstelle eine Liste $L$ der erreichbaren Knoten $v'$. Speichere
          außerdem, durch wie viele Pfade diese Knoten $v'$ jeweils erreichbar sind.
    \item \label{step:c2} Betrachte im folgenden nur die Top-$q$ Knoten bzgl. der
          Anzahl der Pfade von $v$ nach $v'$, wobei $q \in \mathbb{N}$
          eine zu wählende Konstante des DYCOS-Algorithmus ist.\footnote{Sowohl für den DBLP, als auch für den
CORA-Datensatz wurde in \cite[S. 364]{aggarwal2011} $q=10$ gewählt.}
          Diese Knotenmenge heiße im Folgenden $T(v)$ und $p(v, v')$
          sei die Anzahl der Pfade von $v$ über einen Wortknoten nach $v'$.
    \item \label{step:c3} Wähle mit Wahrscheinlichkeit $\frac{p(v, v')}{\sum_{w \in T(v)} p(v, w)}$
          den Knoten $v' \in T(v)$ als Ziel des Zweifachsprungs.
\end{enumerate}

Konkret könnte also ein inhaltlicher Zweifachsprung sowie wie in
\cref{alg:DYCOS-content-multihop} beschrieben umgesetzt werden.
Der Algorithmus bekommt einen Startknoten $v \in V_T$ und
einen $q \in \mathbb{N}$ als Parameter. $q$ ist ein Parameter der
für den DYCOS-Algorithmus zu wählen ist. Dieser Parameter beschränkt
die Anzahl der möglichen Zielknoten $v' \in V_T$ auf diejenigen
$q$ Knoten, die $v$ bzgl. der Textanalyse am ähnlichsten sind.

In \cref{alg:l2} bis \cref{alg:l5} wird \cref{step:c1} durchgeführt
und alle erreichbaren Knoten in $reachableNodes$ mit der Anzahl
der Pfade, durch die sie erreicht werden können, gespeichert.

In \cref{alg:l6} wird \cref{step:c2} durchgeführt.
Ab hier gilt
\[ |T| = \begin{cases}q               &\text{falls } |reachableNodes|\geq q\\
                     |reachableNodes| &\text{sonst }\end{cases}\]

Bei der Wahl der Datenstruktur von $T$ ist zu beachten, dass man in
\cref{alg:21} über Indizes auf Elemente aus $T$ zugreifen können muss.

In \cref{alg:l8} bis \ref{alg:l13} wird ein assoziatives Array erstellt,
das von $v' \in T(v)$ auf die relative
Häufigkeit bzgl. aller Pfade von $v$ zu Knoten aus den Top-$q$ abbildet.

In allen folgenden Zeilen wird \cref{step:c3} durchgeführt.
In \cref{alg:15} bis \cref{alg:22} wird ein Knoten $v' \in T(v)$ mit
einer Wahrscheinlichkeit, die seiner relativen Häufigkeit am Anteil
der Pfaden der Länge 2 von $v$ nach $v'$ über einen beliebigen
Wortknoten entspricht ausgewählt und schließlich zurückgegeben.

\begin{algorithm}
  \caption{Inhaltlicher Zweifachsprung}
  \label{alg:DYCOS-content-multihop}
    \begin{algorithmic}[1]
        \Procedure{InhaltlicherZweifachsprung}{Knoten $v \in V_T$, $q \in \mathbb{N}$}
            \State $erreichbareKnoten \gets$ leeres assoziatives Array\label{alg:l2}
            \ForAll{Wortknoten $w$ in $v.\Call{getWordNodes}{ }$}
                \ForAll{Strukturknoten $x$ in $w.\Call{getStructuralNodes}{ }$}
                    \If{$!erreichbareKnoten.\Call{hasKey}{x}$}
                        \State $erreichbareKnoten[x] \gets 0$
                    \EndIf
                    \State $erreichbareKnoten[x] \gets erreichbareKnoten[x] + 1$
                \EndFor
            \EndFor\label{alg:l5}
            \State \label{alg:l6} $T \gets \Call{max}{erreichbareKnoten, q}$
            \\
            \State \label{alg:l8} $s \gets 0$
            \ForAll{Knoten $x \in T$}
                \State $s \gets s + erreichbareKnoten[x]$
            \EndFor
            \State $relativeHaeufigkeit \gets $ leeres assoziatives Array
            \ForAll{Knoten $x \in T$}
                \State $relativeHaeufigkeit \gets \frac{erreichbareKnoten[x]}{s}$
            \EndFor\label{alg:l13}
            \\
            \State \label{alg:15} $random \gets \Call{random}{0, 1}$
            \State $r \gets 0.0$
            \State $i \gets 0$
            \While{$s < random$}
                \State $r \gets r + relativeHaeufigkeit[i]$
                \State $i \gets i + 1$
            \EndWhile

            \State $v \gets T[i-1]$ \label{alg:21}
            \State \Return $v$ \label{alg:22}
        \EndProcedure
    \end{algorithmic}
\end{algorithm}

\subsection{Vokabularbestimmung}\label{sec:vokabularbestimmung}
Da die Größe des Vokabulars die Datenmenge signifikant beeinflusst,
liegt es in unserem Interesse so wenig Wörter wie möglich ins
Vokabular aufzunehmen. Insbesondere sind Wörter nicht von Interesse,
die in fast allen Texten vorkommen, wie im Deutschen z.~B.
\enquote{und}, \enquote{mit} und die Pronomen. Es ist wünschenswert Wörter zu
wählen, die die Texte möglichst stark voneinander Unterscheiden. Der
DYCOS-Algorithmus wählt die Top-$m$ dieser Wörter als Vokabular, wobei
$m \in \mathbb{N}$ eine festzulegende Konstante ist. In \cite[S. 365]{aggarwal2011}
wird der Einfluss von $m \in \Set{5,10, 15,20}$ auf die Klassifikationsgüte
untersucht und festgestellt, dass die Klassifikationsgüte mit größerem $m$
sinkt, sie also für $m=5$ für den DBLP-Datensatz am höchsten ist. Für den
CORA-Datensatz wurde mit $m \in \set{3,4,5,6}$ getestet und kein signifikanter
Unterschied festgestellt.

Nun kann man manuell eine Liste von zu beachtenden Wörtern erstellen
oder mit Hilfe des Gini-Koeffizienten automatisch ein Vokabular erstellen.
Der Gini-Koeffizient ist ein statistisches Maß, das die Ungleichverteilung
bewertet. Er ist immer im Intervall $[0,1]$, wobei $0$ einer
Gleichverteilung entspricht und $1$ der größtmöglichen Ungleichverteilung.

Sei nun $n_i(w)$ die Häufigkeit des Wortes $w$ in allen Texten mit der $i$-ten
Knotenbeschriftung.
\begin{align}
    p_i(w) &:= \frac{n_i(w)}{\sum_{j=1}^{|\L_t|} n_j(w)} &\text{(Relative Häufigkeit des Wortes $w$)}\\
    G(w)   &:= \sum_{j=1}^{|\L_t|} p_j(w)^2              &\text{(Gini-Koeffizient von $w$)}
\end{align}
In diesem Fall ist $G(w)=0$ nicht möglich, da zur Vokabularbestimmung nur
Wörter betrachtet werden, die auch vorkommen.

Ein Vorschlag, wie die Vokabularbestimmung implementiert werden kann, ist als
Pseudocode mit \cref{alg:vokabularbestimmung} gegeben. In \cref{alg4:l6} wird
eine Teilmenge $S_t \subseteq V_{L,t}$ zum Generieren des Vokabulars gewählt.
In \cref{alg4:l8} wird ein Array $cLabelWords$ erstellt, das $(|\L_t|+1)$
Felder hat. Die Elemente dieser Felder sind jeweils assoziative Arrays, deren
Schlüssel Wörter und deren Werte natürliche Zahlen sind. Die ersten $|\L_t|$
Elemente von $cLabelWords$ dienen dem Zählen der Häufigkeit der Wörter von
Texten aus $S_t$, wobei für jede Beschriftung die Häufigkeit einzeln gezählt
wird. Das letzte Element aus $cLabelWords$ zählt die Summe der Wörter. Diese
Datenstruktur wird in \cref{alg4:l10} bis \ref{alg4:l12} gefüllt.

In \cref{alg4:l17} bis \ref{alg4:l19} wird die relative Häufigkeit der Wörter
bzgl. der Beschriftungen bestimmt. Daraus wird in \cref{alg4:l20} bis
\ref{alg4:l22} der Gini-Koeffizient berechnet. Schließlich werden in
\cref{alg4:l23} bis \ref{alg4:l24} die Top-$q$ Wörter mit den
höchsten Gini-Koeffizienten zurückgegeben.

\begin{algorithm}[ht]
    \begin{algorithmic}[1]
        \Require \\
                 $V_{L,t}$ (beschriftete Knoten),\\
                 $\L_t$ (Menge der Beschriftungen),\\
                 $f:V_{L,t} \rightarrow \L_t$ (Beschriftungsfunktion),\\
                 $m$ (Gewünschte Vokabulargröße)
        \Ensure  $\M_t$ (Vokabular)\\
        \State $S_t \gets \Call{Sample}{V_{L,t}}$\label{alg4:l6} \Comment{Wähle $S_t \subseteq V_{L,t}$ aus}
        \State $\M_t \gets \emptyset$ \Comment{Menge aller Wörter}
        \State $cLabelWords \gets$ Array aus $(|\L_t|+1)$ assoziativen Arrays\label{alg4:l8}
        \ForAll{$v \in S_t$} \label{alg4:l10}
            \State $i \gets \Call{getLabel}{v}$
            \State \Comment{$w$ ist das Wort, $c$ ist die Häufigkeit}
            \ForAll{$(w, c) \in \Call{getTextAsMultiset}{v}$}
                \State $cLabelWords[i][w] \gets cLabelWords[i][w] + c$
                \State $cLabelWords[|\L_t|][w] \gets cLabelWords[i][|\L_t|] + c$
                \State $\M_t = \M_t \cup \Set{w}$
            \EndFor
        \EndFor\label{alg4:l12}
		\\
        \ForAll{Wort $w \in \M_t$}
            \State $p \gets $ Array aus $|\L_t|$ Zahlen in $[0, 1]$\label{alg4:l17}
            \ForAll{Label $i \in \L_t$}
                \State $p[i] \gets \frac{cLabelWords[i][w]}{cLabelWords[i][|\L_t|]}$
            \EndFor\label{alg4:l19}

            \State $w$.gini $\gets 0$ \label{alg4:l20}
            \ForAll{$i \in 1, \dots, |\L_t|$}
                \State $w$.gini $\gets$ $w$.gini + $p[i]^2$
            \EndFor\label{alg4:l22}
        \EndFor

        \State $\M_t \gets \Call{SortDescendingByGini}{\M_t}$\label{alg4:l23}
        \State \Return $\Call{Top}{\M_t, m}$\label{alg4:l24}
    \end{algorithmic}
\caption{Vokabularbestimmung}
\label{alg:vokabularbestimmung}
\end{algorithm}

Die Menge $S_t$ kann aus der Menge aller Dokumente, deren Knoten beschriftet
sind, mithilfe des in \cite{Vitter} vorgestellten Algorithmus bestimmt werden.



\section{Analyse des DYCOS-Algorithmus}
Für den DYCOS-Algorithmus wurde in \cite{aggarwal2011} bewiesen, dass sich nach
Ausführung von DYCOS für einen unbeschrifteten Knoten mit einer
Wahrscheinlichkeit von höchstens $(|\L_t|-1)\cdot e^{-l \cdot b^2 / 2}$ eine
Knotenbeschriftung ergibt, deren relative Häufigkeit weniger als $b$ der
häufigsten Beschriftung ist. Dabei ist $|\L_t|$ die Anzahl der Beschriftungen
und $l$ die Länge der Random-Walks.

Außerdem wurde experimentell anhand des DBLP-Datensatzes\footnote{http://dblp.uni-trier.de/}
und des CORA-Datensatzes\footnote{http://people.cs.umass.edu/~mccallum/data/cora-classify.tar.gz}
gezeigt (vgl. \cref{tab:datasets}), dass die Klassifikationsgüte nicht wesentlich von der Anzahl der Wörter mit
höchstem Gini-Koeffizient $m$ abhängt. Des Weiteren betrug die Ausführungszeit
auf einem Kern eines Intel Xeon $\SI{2.5}{\GHz}$ Servers mit
$\SI{32}{\giga\byte}$ RAM für den DBLP-Datensatz unter $\SI{25}{\second}$,
für den CORA-Datensatz sogar unter $\SI{5}{\second}$. Dabei wurde eine
für CORA eine Klassifikationsgüte von 82\% - 84\% und auf den DBLP-Daten
von 61\% - 66\% erreicht.

\begin{table}[htp]
    \centering
    \begin{tabular}{|l||r|r|r|r|}\hline
    \textbf{Name} & \textbf{Knoten} & \textbf{davon beschriftet} & \textbf{Kanten}  & \textbf{Beschriftungen} \\ \hline\hline
    \textbf{CORA} & \num{19396}  & \num{14814}             & \num{75021}   & 5              \\
    \textbf{DBLP} & \num{806635} & \num{18999 }            & \num{4414135} & 5              \\\hline
    \end{tabular}
    \caption{Datensätze, die für die experimentelle analyse benutzt wurden}
    \label{tab:datasets}
\end{table}

Obwohl es sich nicht sagen lässt, wie genau die Ergebnisse aus
\cite{aggarwal2011} zustande gekommen sind, eignet sich das
Kreuzvalidierungsverfahren zur Bestimmung der Klassifikationsgüte wie es in
\cite{Lavesson,Stone1974} vorgestellt wird:
\begin{enumerate}
    \item Betrachte nur $V_{L,T}$.
    \item Unterteile $V_{L,T}$ zufällig in $k$ disjunkte Mengen $M_1, \dots, M_k$.
    \item \label{schritt3} Teste die Klassifikationsgüte, wenn die Knotenbeschriftungen
          aller Knoten in $M_i$ für DYCOS verborgen werden für $i=1,\dots, k$.
    \item Bilde den Durchschnitt der Klassifikationsgüten aus \cref{schritt3}.
\end{enumerate}

Es wird $k=10$ vorgeschlagen.


\section{Probleme des DYCOS-Algorithmus}
Bei der Anwendung des in \cite{aggarwal2011} vorgestellten Algorithmus
auf reale Datensätze könnten zwei Probleme auftreten,
die im Folgenden erläutert werden. Außerdem werden Verbesserungen
vorgeschlagen, die es allerdings noch zu untersuchen gilt.

\subsection{Anzahl der Knotenbeschriftungen}
So, wie der DYCOS-Algorithmus vorgestellt wurde, können nur Graphen bearbeitet werden,
deren Knoten jeweils höchstens eine Beschriftung haben. In vielen Fällen, wie z.~B.
Wikipedia mit Kategorien als Knotenbeschriftungen haben Knoten jedoch viele Beschriftungen.

Auf einen ersten Blick ist diese Schwäche einfach zu beheben, indem
man beim zählen der Knotenbeschriftungen für jeden Knoten jedes Beschriftung zählt. Dann
wäre noch die Frage zu klären, mit wie vielen Beschriftungen der betrachtete
Knoten beschriftet werden soll.

Jedoch ist z.~B. bei Wikipedia-Artikeln auf den Knoten eine
Hierarchie definiert. So ist die Kategorie \enquote{Klassifikationsverfahren}
eine Unterkategorie von \enquote{Klassifikation}. Bei dem Kategorisieren
von Artikeln sind möglichst spezifische Kategorien vorzuziehen, also
kann man nicht einfach bei dem Auftreten der Kategorie \enquote{Klassifikationsverfahren}
sowohl für diese Kategorie als auch für die Kategorie \enquote{Klassifikation}
zählen.


\subsection{Überanpassung und Reklassifizierung}
Aggarwal und Li beschreiben in \cite{aggarwal2011} nicht, auf welche
Knoten der Klassifizierungsalgorithmus angewendet werden soll. Jedoch
ist die Reihenfolge der Klassifizierung relevant. Dazu folgendes
Minimalbeispiel:

Gegeben sei ein dynamischer Graph ohne textuelle Inhalte. Zum Zeitpunkt
$t=1$ habe dieser Graph genau einen Knoten $v_1$ und $v_1$  sei
mit dem $A$ beschriftet. Zum Zeitpunkt $t=2$ komme ein nicht beschrifteter
Knoten $v_2$ sowie die Kante $(v_2, v_1)$ hinzu.\\
Nun wird der DYCOS-Algorithmus auf diesen Knoten angewendet und
$v_2$ mit $A$ beschriftet.\\
Zum Zeitpunkt $t=3$ komme ein Knoten $v_3$, der mit $B$ beschriftet ist,
und die Kante $(v_2, v_3)$ hinzu.

\begin{figure}[ht]
    \centering
    \subfloat[$t=1$]{
        \tikzstyle{vertex}=[draw,black,circle,minimum size=10pt,inner sep=0pt]
\tikzstyle{edge}=[very thick]
\begin{tikzpicture}[scale=1,framed]
    \node (a)[vertex,label=$A$] at (0,0) {$v_1$};
    \node (b)[vertex, white] at (1,0) {$v_2$};
    \node (struktur)[label={[label distance=-0.2cm]0:$t=1$}] at (-2,1) {};
\end{tikzpicture}

        \label{fig:graph-t1}
    }%
    \subfloat[$t=2$]{
        \tikzstyle{vertex}=[draw,black,circle,minimum size=10pt,inner sep=0pt]
\tikzstyle{edge}=[very thick]
\begin{tikzpicture}[scale=1,framed]
    \node (a)[vertex,label=$A$] at (0,0) {$v_1$};
    \node (b)[vertex,label={\color{blue}$A$}] at (1,0) {$v_2$};
    \draw[->] (b) -- (a);
    \node (struktur)[label={[label distance=-0.2cm]0:$t=2$}] at (-2,1) {};
\end{tikzpicture}

        \label{fig:graph-t2}
    }

    \subfloat[$t=3$]{
        \tikzstyle{vertex}=[draw,black,circle,minimum size=10pt,inner sep=0pt]
\tikzstyle{edge}=[very thick]
\begin{tikzpicture}[scale=1,framed]
    \node (a)[vertex,label=$A$] at (0,0) {$v_1$};
    \node (b)[vertex,label={\color{blue}$A$}] at (1,0) {$v_2$};
    \node (c)[vertex,label=$B$] at (2,0) {$v_3$};
    \draw[->] (b) -- (a);
    \draw[->] (b) -- (c);
    \node (struktur)[label={[label distance=-0.2cm]0:$t=3$}] at (-1,1) {};
\end{tikzpicture}

        \label{fig:graph-t3}
    }%
    \subfloat[$t=4$]{
        \tikzstyle{vertex}=[draw,black,circle,minimum size=10pt,inner sep=0pt]
\tikzstyle{edge}=[very thick]
\begin{tikzpicture}[scale=1,framed]
    \node (a)[vertex,label=$A$] at (0,0) {$v_1$};
    \node (b)[vertex,label=45:{\color{blue}$A$}] at (1,0) {$v_2$};
    \node (c)[vertex,label=$B$] at (2,0) {$v_3$};
    \node (d)[vertex] at (1,1) {$v_4$};
    \draw[->] (b) -- (a);
    \draw[->] (b) -- (c);

    \draw[->] (d) -- (a);
    \draw[->] (d) -- (b);
    \draw[->] (d) -- (c);
    \node (struktur)[label={[label distance=-0.2cm]0:$t=3$}] at (-1,1) {};
\end{tikzpicture}

        \label{fig:graph-t4}
    }%
    \label{Formen}
    \caption{Minimalbeispiel für den Einfluss früherer DYCOS-Anwendungen}
\end{figure}

Würde man nun den DYCOS-Algorithmus erst jetzt, also anstelle von
Zeitpunkt $t=2$ zum Zeitpunkt $t=3$ auf den Knoten $v_2$ anwenden, so
würde eine $50\%$-Wahrscheinlichkeit bestehen, dass dieser mit $B$
beschriftet wird. Aber in diesem Beispiel wurde der Knoten schon
zum Zeitpunkt $t=2$ beschriftet. Obwohl es in diesem kleinem Beispiel
noch keine Rolle spielt, wird das Problem klar, wenn man weitere
Knoten einfügt:

Wird zum Zeitpunkt $t=4$ ein unbeschrifteter Knoten $v_4$ und die Kanten
$(v_1, v_4)$, $(v_2, v_4)$, $(v_3, v_4)$ hinzugefügt, so ist die
Wahrscheinlichkeit, dass $v_4$ mit $A$ beschriftet wird bei $\frac{2}{3}$.
Werden die unbeschrifteten Knoten jedoch erst jetzt und alle gemeinsam
beschriftet, so ist die Wahrscheinlichkeit für $A$ als Beschriftung bei nur $50\%$.
Bei dem DYCOS-Algorithmus findet also eine Überanpassung an vergangene
Beschriftungen statt.

Das Reklassifizieren von Knoten könnte eine mögliche Lösung für dieses
Problem sein. Knoten, die durch den DYCOS-Algorithmus beschriftet wurden
könnten eine Lebenszeit bekommen (TTL, Time to Live). Ist diese
abgelaufen, wird der DYCOS-Algorithmus erneut auf den Knoten angewendet.




\section{Ausblick}
Den DYCOS-Algorithmus kann in einigen Aspekten erweitert werden. So könnte man
vor der Auswahl des Vokabulars jedes Wort auf den Wortstamm zurückführen. Dafür
könnte zum Beispiel der in \cite{porter} vorgestellte Porter-Stemming-Algorithmus verwendet werden. Durch diese Maßnahme wird das Vokabular kleiner
gehalten wodurch mehr Artikel mit einander durch Vokabular verbunden werden
können. Außerdem könnte so der Gini-Koeffizient ein besseres Maß für die
Gleichheit von Texten werden.

Eine weitere Verbesserungsmöglichkeit besteht in der Textanalyse. Momentan ist
diese noch sehr einfach gestrickt und ignoriert die Reihenfolge von Wörtern
beziehungsweise Wertungen davon. So könnte man den DYCOS-Algorithmus in einem
sozialem Netzwerk verwenden wollen, in dem politische Parteiaffinität von
einigen Mitgliedern angegeben wird um die Parteiaffinität der restlichen
Mitglieder zu bestimmen. In diesem Fall macht es jedoch einen wichtigen
Unterschied, ob jemand über eine Partei gutes oder schlechtes schreibt.

Eine einfache Erweiterung des DYCOS-Algorithmus wäre der Umgang mit mehreren
Beschriftungen.

DYCOS beschränkt sich bei inhaltlichen Zweifachsprüngen auf die
Top-$q$-Wortknoten, also die $q$ ähnlichsten Knoten gemessen mit der
Aggregatanalyse, allerdings wurde bisher noch nicht untersucht, wie der
Einfluss von $q \in \mathbb{N}$ auf die Klassifikationsgüte ist.


\bibliographystyle{IEEEtranSA}
\bibliography{literatur}

\end{document}

